% 5
\begin{node}[Normal forms]\label{prop-000P}%
A common trick in logic is to take some sort of ``normal form'' of
formulas. That is to say, we restrict attention to a sublanguage of
propositional logic such that (a) it is constructed using an adequate
set of connectives~\zref{prop-000E}, and (b) it simplifies proving if a
proposition is satisfiable or unsatisfiable. (This is far too applied a
topic for Mendelson~\cite{mendelson2015mathematical}, and I am relying on
Harrison~\cite{harrison2009handbook} for much of the material here.)

\begin{definition}\label{prop-000Q}%
In propositional logic, we define a \define{Literal} to be a
propositional variable or its negation. We say that a literal is
\define{Negative} if it is of the form $\neg A$; otherwise we say it is
\define{Positive}. 
\end{definition}

\begin{definition}\label{prop-000O}%
We say a proposition is in \define{Negation Normal Form} if it is
constructed using conjunctions and disjunctions of
literals. Inductively:
\begin{enumerate}
\item A literal~\zref{prop-000Q} is in negation normal form;
\item If $\varphi$ and $\psi$ are in negation normal form, then so are
  $\varphi\land\psi$ and $\varphi\lor\psi$;
\item We nominally accept $\verum$ and $\falsum$ as ``degenerate'' cases
  of negation normal form.
\end{enumerate}

\begin{node}[History of term NNF]\label{prop-normal-form-0003}%
The term ``negation normal form'' appears to have been coined by the
1960s. It appears in Raymond Smullyan's 1968 book \textit{First-Order Logic}
in exercise 3 (on page 131), and in John Barwise's 1967 doctoral thesis
``Infinitary logic and admissible sets''. But it is unclear to me if
either of these men actually \emph{coined} the term. Certainly, it seems
safe to speculate that it appears \emph{after} 1928, i.e., when Hilbert
coined the terms ``disjunctive normal form'' and ``conjunctive normal form''~\zref{prop-normal-form-0004}.
\end{node}
\end{definition}

\begin{node}[Transforming to NNF]\label{prop-000R}%
We can transform a formula into negation normal form by eliminating
``$\implies$'' and ``$\iff$'' using
\begin{subequations}
\begin{equation}
(A\implies B) \iff ((\neg A)\lor B),
\end{equation}
and
\begin{equation}
(A\iff B)\iff((A\lor\neg B)\land((\neg A)\lor B)).
\end{equation}
\end{subequations}
We can use De Morgan laws to simplify negated conjunctions and
disjunctions, and ``pull negations down to the literals'':
\begin{subequations}
  \begin{align}
    \neg(A\land B) &\iff (\neg A)\lor(\neg B)\\
    \neg(A\lor B) &\iff (\neg A)\land(\neg B)\\
    \neg(\neg A) &\iff A
\end{align}
\end{subequations}
In \SML, we could write a function transforming any \verb|Prop|
instance~\zref{prop-000G} into negation normal form (using some
optimizations) as something like:
\begin{sml}
(* nnf : Prop -> Prop *)
fun nnf (And(A,B)) = And (nnf A, nnf B)
  | nnf (Or(A,B)) = Or(nnf A, nnf B)
  | nnf (Implies(A,B)) = Or(nnf(Not A), nnf B)
  | nnf (Iff(A,B)) = Or(And(nnf A, nnf B),
                        And(nnf(Not A), nnf(Not B)))
  | nnf (Not (Not A)) = nnf A
  | nnf (Not (And (A, B))) = Or(nnf(Not A),nnf(Not B))
  | nnf (Not (Implies (A,B))) = And(nnf A, nnf(Not B))
  | nnf (Not (Iff (A,B))) = Or(And(nnf A, nnf(Not B)),
                               And(nnf(Not A), nnf B))
  | nnf (fm as _) = fm;
\end{sml}
\end{node}

\begin{definition}\label{prop-normal-form-0000}%
A proposition is in \define{Negation Equivalence Normal Form} (NENF) if
it comprises of conjunctions, disjunctions, bi-implications, and literals.
We give an explicit inductive definition:
\begin{enumerate}
\item A literal~\zref{prop-000Q} is in NENF;
\item If $\varphi$ and $\psi$ are in NENF, then so are
  $\varphi\land\psi$, $\varphi\lor\psi$, and $\varphi\iff\psi$;
\item We nominally accept $\verum$ and $\falsum$ as ``degenerate'' cases
  of NENF.
\end{enumerate}

\begin{node}[Remark]\label{prop-normal-form-0001}%
Harrison~\cite{harrison2009handbook} simply refers to this normal form
as ``NENF'' without explaining what this acronym means. I am guessing
based on Barry Watson's notes\footnote{\url{http://barrywatson.se/cl/cl_nenf.html}}
that it refers to ``negation equational normal form'' and on the index
page to ``negation equivalence normal form''. I do not know who
first coined the acronym ``NENF'', and I do not honestly know what
``NENF'' stands for.
\end{node}
\end{definition}

\begin{node}[Transforming to NENF]\label{prop-normal-form-0002}%
The transformation to NENF more or less follows the rules for
transformation to NNF, except it preserves biconditionals. Its rules may
be coded up quickly (with precomposing with a simplification function,
which will simplify the proposition with various identities):
\begin{sml}
local
  (* nenf1 : Prop -> Prop *)
  fun nenf1 (And (A,B)) = And (nenf1 A, nenf1 B)
    | nenf1 (Or (A,B)) = Or (nenf1 A, nenf1 B)
    | nenf1 (Implies (A, B)) = Or (nenf1 (Not A), nenf1 B)
    | nenf1 (Iff (A, B)) = Iff (nenf1 A, nenf1 B)
    | nenf1 (Not (Not A)) = nenf1 A
    | nenf1 (Not (And (A, B))) = Or (nenf1(Not A), nenf1(Not B))
    | nenf1 (Not (Or (A, B))) = And (nenf1(Not A), nenf1(Not B))
    | nenf1 (Not (Implies (A, B))) = And (nenf A, nenf1(Not B))
    | nenf1 (Not (Iff (A, B))) = Iff (nenf1(Not A), nenf1 B)
    | nenf1 (fm as _) = fm
in
(* nenf : Prop -> Prop *)
  fun nenf fm = nenf1 (simplify fm)
end;
\end{sml}
\end{node}

\begin{definition}\label{prop-000S}%
A proposition is in \define{Disjunctive Normal Form} if it is the
disjunction of conjuncts, i.e., when it is of the form
\begin{subequations}
\begin{equation}
\varphi = D_{1}\lor D_{2}\lor\dots\lor D_{m}
\end{equation}
where each disjunct $D_{i}$ is of the form
\begin{equation}
D_{i} = \ell_{i_{1}}\land\ell_{i_{2}}\land\dots\land\ell_{i_{n_{i}}}
\end{equation}
\end{subequations}
(where each $\ell_{\star}$ is a literal~\zref{prop-000Q}).
\begin{theorem}\label{prop-000T}%
If a proposition is in disjunctive normal form, then it is in negation
normal form.
\end{theorem}
\end{definition}

\begin{definition}\label{prop-000U}%
A proposition is in \define{Conjunctive Normal Form} if it is the
conjunction of disjunctions of literals, i.e., if it is of the form
\begin{subequations}
\begin{equation}
C_{1}\land C_{2}\land\dots\land C_{m}
\end{equation}
where each conjunct $C_{i}$ is of the form
\begin{equation}
\ell_{i_{1}}\lor\ell_{i_{2}}\lor\dots\lor\ell_{i_{n_{i}}},
\end{equation}
and each $\ell_{i_{j}}$ is a literal~\zref{prop-000Q}.
\end{subequations}

\begin{node}[History of terms CNF and DNF]\label{prop-normal-form-0004}%
Carnap's ``Testability and Meaning'' (\textit{Philosophy of Science}
\textbf{3} no.4 (1936) 419--471) cites Hilbert (\textit{Logik} [?],
p.63) and separately Carnap's earlier book \textit{Logical Syntax of Language}
(calling it  ``conjunctive standard form'') as the source of conjunctive
normal form. Coincidentally, ``Testability and Meaning'' appears to coin
the term ``disjunctive normal form'' on page 437, citing page 13 of some work by Hilbert (there is no bibliography for ``Testability and Meaning'').
The 1950 English translation of Hilbert and Ackermann's
\textit{Principles of Mathematical Logic} uses the term ``disjunctive
normal form'' (ch.~1, \S6) and ``conjunctive normal form'' (ch~1, \S7).
It seems safe to credit David Hilbert for inventing these notions, and
may be found in this German book published 1928.
\end{node}

\begin{theorem}\label{prop-000V}%
If a proposition is in conjunctive normal form, then it is in negation
normal form.
\end{theorem}
\end{definition}

\begin{definition}\label{prop-000W}%
Let $\varphi$ be a proposition.
We say a proposition $\psi$ is $\varphi$ in \define{Definitional Conjunctive Normal Form}
(or ``\textit{Definitional CNF}\,'' for short) if
$\psi$ is equisatisfiable as $\varphi$ (i.e., $\psi$ is satisfiable if
and only if $\varphi$ is satisfiable) and $\psi$ is obtained by finitely
many transformations, each of which introduce a ``fresh'' propositional
variable $X$ which replaces all instances of some subproposition
$\alpha$, and we conjunct onto it $X\iff\alpha$. There are a great many
variations of the procedure.

\begin{node}\label{prop-000X}%
It appears the first time the \emph{phrase} ``definitional CNF'' was
used may be found in Beckert and Posegga~\cite{beckert1994lean}. The
idea may be traced back to Tsetin~\cite{Tseitin1983}.
\end{node}
\end{definition}
\end{node}
