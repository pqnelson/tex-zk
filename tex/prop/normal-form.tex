% 24
\begin{node}[Normal forms]\label{prop-000P}%
A common trick in logic is to take some sort of ``normal form'' of
formulas. That is to say, we restrict attention to a sublanguage of
propositional logic such that (a) it is constructed using an adequate
set of connectives~\zref{prop-000E}, and (b) it simplifies proving if a
proposition is satisfiable or unsatisfiable. (This is far too applied a
topic for Mendelson~\cite{mendelson2015mathematical}, and I am relying on
Harrison~\cite{harrison2009handbook} for much of the material here.)

\begin{definition}\label{prop-000Q}%
In propositional logic, we define a \define{Literal} to be a
propositional variable or its negation. We say that a literal is
\define{Negative} if it is of the form $\neg A$; otherwise we say it is
\define{Positive}.

\begin{node}\label{prop:normal-form-000F}%
We can give some code to test if a formula is a literal, to check if it
is positive or negative:

\begin{sml}
fun is_literal (Atom _) = true
 |  is_literal (Not A) = is_literal A
 |  is_literal _ = false;

fun is_negative (Not _) = true
 |  is_negative _ = false;

fun is_positive lit = not (is_neative lit);
\end{sml}
\end{node}

\begin{node}[Negation]\label{prop:normal-form-000G}%
We speak of \emph{negating} a literal, which simply takes $\neg A$ and
produces $A$ (and vice-versa). We will implement more generous code, to
work with \emph{any} proposition $A$.

\begin{sml}
(* negate : Prop -> Prop *)
fun negate (Not A) = A
 |  negate A = Not A;
\end{sml}
\end{node}
\end{definition}

\begin{definition}\label{prop-000O}%
We say a proposition is in \define{Negation Normal Form} if it is
constructed using conjunctions and disjunctions of
literals. Inductively:
\begin{enumerate}
\item A literal~\zref{prop-000Q} is in negation normal form;
\item If $\varphi$ and $\psi$ are in negation normal form, then so are
  $\varphi\land\psi$ and $\varphi\lor\psi$;
\item We nominally accept $\verum$ and $\falsum$ as ``degenerate'' cases
  of negation normal form.
\end{enumerate}
\end{definition}

\begin{node}[Transforming to NNF]\label{prop-000R}%
We can transform a formula into negation normal form by eliminating
``$\implies$'' and ``$\iff$'' using
\begin{subequations}
\begin{equation}
(A\implies B) \iff ((\neg A)\lor B),
\end{equation}
and
\begin{equation}
(A\iff B)\iff((A\lor\neg B)\land((\neg A)\lor B)).
\end{equation}
\end{subequations}
We can use De Morgan laws to simplify negated conjunctions and
disjunctions, and ``pull negations down to the literals'':
\begin{subequations}
  \begin{align}
    \neg(A\land B) &\iff (\neg A)\lor(\neg B)\\
    \neg(A\lor B) &\iff (\neg A)\land(\neg B)\\
    \neg(\neg A) &\iff A
\end{align}
\end{subequations}
In \SML, we could write a function transforming any \verb|Prop|
instance~\zref{prop-000G} into negation normal form (using some
simplifications~\zref{prop-000Z}) as something like:
\begin{sml}
(* in struct Prop *)
local
  fun nnf1 (And(A,B)) = And (nnf1 A, nnf1 B)
    | nnf1 (Or(A,B)) = Or(nnf1 A, nnf1 B)
    | nnf1 (Implies(A,B)) = Or(nnf1(Not A), nnf1 B)
    | nnf1 (Iff(A,B)) = Or(And(nnf1 A, nnf1 B),
                           And(nnf1(Not A), nnf1(Not B)))
    | nnf1 (Not (Not A)) = nnf1 A
    | nnf1 (Not (And (A, B))) = Or(nnf1(Not A),nnf1(Not B))
    | nnf1 (Not (Implies (A,B))) = And(nnf1 A, nnf1(Not B))
    | nnf1 (Not (Iff (A,B))) = Or(And(nnf1 A, nnf1(Not B)),
                                 And(nnf1(Not A), nnf1 B))
    | nnf1 (fm as _) = fm;
in
  fun nnf fm = nnf1 (simplify fm)
end;
\end{sml}
\end{node}

\begin{definition}\label{prop-normal-form-0000}%
A proposition is in \define{Negation Equivalence Normal Form} (NENF) if
it comprises of conjunctions, disjunctions, bi-implications, and literals.
We give an explicit inductive definition:
\begin{enumerate}
\item A literal~\zref{prop-000Q} is in NENF;
\item If $\varphi$ and $\psi$ are in NENF, then so are
  $\varphi\land\psi$, $\varphi\lor\psi$, and $\varphi\iff\psi$;
\item We nominally accept $\verum$ and $\falsum$ as ``degenerate'' cases
  of NENF.
\end{enumerate}

\begin{node}[Remark]\label{prop-normal-form-0001}%
Harrison~\cite{harrison2009handbook} simply refers to this normal form
as ``NENF'' without explaining what this acronym means. I am guessing
based on Barry Watson's notes\footnote{\url{http://barrywatson.se/cl/cl_nenf.html}}
that it refers to ``negation equational normal form'' and on the index
page to ``negation equivalence normal form''. I do not know who
first coined the acronym ``NENF'', and I do not honestly know what
``NENF'' stands for.
\end{node}
\end{definition}

\begin{node}[Transforming to NENF]\label{prop-normal-form-0002}%
The transformation to NENF more or less follows the rules for
transformation to NNF, except it preserves biconditionals. Its rules may
be coded up quickly (with precomposing with a simplification function,
which will simplify the proposition with various identities~\zref{prop-000Z}):
\begin{sml}
(* in struct Prop *)
local
  (* nenf1 : Prop -> Prop *)
  fun nenf1 (And (A,B)) = And (nenf1 A, nenf1 B)
    | nenf1 (Or (A,B)) = Or (nenf1 A, nenf1 B)
    | nenf1 (Implies (A, B)) = Or (nenf1 (Not A), nenf1 B)
    | nenf1 (Iff (A, B)) = Iff (nenf1 A, nenf1 B)
    | nenf1 (Not (Not A)) = nenf1 A
    | nenf1 (Not (And (A, B))) = Or (nenf1(Not A), nenf1(Not B))
    | nenf1 (Not (Or (A, B))) = And (nenf1(Not A), nenf1(Not B))
    | nenf1 (Not (Implies (A, B))) = And (nenf A, nenf1(Not B))
    | nenf1 (Not (Iff (A, B))) = Iff (nenf1(Not A), nenf1 B)
    | nenf1 (fm as _) = fm
in
(* nenf : Prop -> Prop *)
  fun nenf fm = nenf1 (simplify fm)
end;
\end{sml}
\end{node}

\begin{definition}\label{prop-000S}%
A proposition is in \define{Disjunctive Normal Form} if it is the
disjunction of conjuncts, i.e., when it is of the form
\begin{subequations}
\begin{equation}
\varphi = D_{1}\lor D_{2}\lor\dots\lor D_{m}
\end{equation}
where each disjunct $D_{i}$ is of the form
\begin{equation}
D_{i} = \ell_{i_{1}}\land\ell_{i_{2}}\land\dots\land\ell_{i_{n_{i}}}
\end{equation}
\end{subequations}
(where each $\ell_{\star}$ is a literal~\zref{prop-000Q}).

\begin{theorem}\label{prop-000T}%
If a proposition is in disjunctive normal form, then it is in negation
normal form.
\end{theorem}

\begin{node}[Construction using truth tables]\label{prop:normal-form-0005}%
Given a proposition $\varphi$, we can form a logically equivalent
proposition $\psi$ by writing down the truth table for $\varphi$. Each
row of the truth table defines a valuation~\zref{prop-0004} $v$,
which we use to form a disjunct $\ell_{1}\land\dots\land\ell_{k}$
where for each propositional variable $p_{j}$ gives us the literal
$\ell_{j}=p_{j}$ if $v(p_{j})$ is true, and $\ell_{j}=\neg p_{j}$
if $v(p_{j})$ is false. Then taking the disjunction of all these
disjuncts gives us the proposition $\psi$ logically equivalent to
$\varphi$. This requires all $2^{k}$ valuations.
\end{node}

\begin{node}[Construction via transformation]\label{prop:normal-form-0006}%
If we start with a proposition $\varphi$ in negation normal
form~\zref{prop-000O}, then we can repeatedly apply the following
tautologies to transform it to DNF:
\begin{subequations}
\begin{equation}
A\land (B\lor C)\iff (A\land B)\lor(A\land C)
\end{equation}
\begin{equation}
(A\lor B)\land C\iff (A\land C)\lor(B\land C)
\end{equation}
\end{subequations}
The implementation in \SML\ requires a bit of care, to avoid
transforming the syntax tree needlessly ``too many times''. We separate
out applying the distributive laws (assuming the subpropositions are
already in DNF) from the recursive transformations of subpropositions
into DNF, so in particular only on conjunctions will we have to think
about distributivity. Then the transformation to DNF will pass over the
syntax tree once.
\begin{sml}
local
  fun distrib (And (p,(Or(q,r)))) = Or(distrib(And(p,q)), distrib(And(p,r)))
   |  distrib (And (Or(p,q),r)) = Or(distrib(And(p,r)), distrib(And(q,r)))
   |  distrib (fm as _) = fm;
in
  fun to_dnf (And (p,q)) = distrib(And(to_dnf p, to_dnf q))
   |  to_dnf (Or (p,q)) = Or(to_dnf p, to_dnf q)
   |  to_dnf (fm as _) = fm
end;
\end{sml}
\end{node}
\begin{node}[Set-based representation]\label{prop:normal-form-0007}%
\begin{node}\label{prop:normal-form-0008}%
The basic idea is to represent a proposition in DNF using sets, so that
we can quickly distinguish $A\land\neg B\lor\neg A\land B$ disjuncts
from conjuncted literals. We encode this particular example as
$\{\{A,\neg B\},\{\neg A,B\}\}$
a set of conjunctions (where the conjunctions are themselves sets of
literals). 
\end{node}
\begin{node}\label{prop:normal-form-0009}%
Distributivity~\zref{prop:normal-form-0006} needs to be phrased using
this set representation. We would expect something like (supposing we
had access to a \verb|Set| library):
\begin{sml}
(* pseudocode *)
fun set_dnf (And (A,B)) = distrib (set_dnf A) (set_dnf B)
 |  set_dnf (Or (A,B)) = Set.union (set_dnf A) (set_dnf B)
 |  set_dnf (fm as _) = Set.singleton (Set.singleton fm);
\end{sml}
This would transform $A\land(B\lor C)$ into $dnf(\{\{A\}\}\land\{\{B\},\{C\}\})$
which should coincide with $(A\land B)\lor(A\land C)$. This would require
$distrib(\{\{A\}\},\{\{B\},\{C\}\})=\{\{A,B\},\{A,C\}\}$.
Similarly, $(A\lor B)\land C$ would be expected to transform into $\{\{A,C\},\{B,C\}\}$,
which tells us how $distrib(\{\{A\},\{B\}\}, \{\{C\}\})$ would behave.
This gives us the basis for the general situation
\begin{equation}
(A_{1}\lor\dots\lor A_{m})\land(B_{1}\lor\dots\lor B_{n})
=\bigvee_{i=1}^{m}\bigvee_{j=1}^{n}(A_{i}\land B_{j}).
\end{equation}
This gives us the code for distributivity:
\begin{sml}
(* distrib : (Prop Set.t) -> (Prop Set.t) -> (Prop Set.t) Set.t *)
fun distrib c1 c2 = Set.map (fn a =>
                               Set.map (fn b => 
                                         Set.union (Set.singleton a)
                                                   (Set.singleton b))
                                       c2)
                            c1;
\end{sml}
\end{node}
\begin{node}\label{prop:normal-form-000A}%
Now we may follow suite with using transformations, and write the code
for using sets to transform a proposition in NNF to a set representation
of DNF:
\begin{sml}
fun set_dnf (And (A, B)) = distrib (set_dnf A) (set_dnf B)
 |  set_dnf (Or (A, B)) = Set.union (set_dnf A) (set_dnf B)
 |  set_dnf (fm as _) = Set.singleton (Set.singleton fm);
\end{sml}
\end{node}
\begin{node}[Optimization: use noncontradictory disjuncts]\label{prop:normal-form-000B}%
We can remove the trivial disjuncts, which are the ones which involve
both a literal and its negation.
\begin{sml}
(* trivial : Prop Set.t -> Prop Set.t *)
fun trivial literals =
let
  val (pos,neg) = Set.partition positive literals
in
  not (Set.null (Set.intersection pos (Set.map negate neg)))
end;

(* Example usage: Set.filter (non trivial) (set_dnf prop)
   produces a logically equivalent but smaller DNF representation of the
   proposition. *)
\end{sml}
\end{node}
\begin{node}[Optimization: Subsumption]\label{prop:normal-form-000C}%
If we have two disjuncts $D_{1}$ and $D_{2}$ which appear in the DNF for
our formula $D_{1}\lor D_{2}\lor\dots$ which happen to be such that
$D_{1}\subset D_{2}$, then we can discard $D_{1}$. In this case, we say
$D_{2}$ \define{Subsumes} $D_{1}$.

We use this to transform a generic formula (possibly in NNF, possibly not)
into a set-representation of DNF.
\begin{sml}
fun simp_dnf False = Set.Empty
 |  simp_dnf True = Set.singleton (Set.Empty)
 |  simp_dnf A =
let
  val djs = Set.filter (non trivial) (set_dnf (nnf A))
in
  Set.filter (fun d =>
                not (Set.exists (fn d' => Set.isSubsetOf d' d) djs))
             djs
end;
\end{sml}
\end{node}
\begin{node}\label{prop:normal-form-000D}%
We can now transform back from the set-representation to a \lstinline[basicstyle=\color{sbase03}\ttfamily\small,language=SML]{Prop}
object using the preceding function. Care must be taken to transform
$\{\emptyset\}\mapsto\verum$ and $\{\}\mapsto\falsum$, which is carved
out explicitly in the helper functions.
\begin{sml}
(* dnf : Prop -> Prop *)
local
  fun set_disj s = if Set.null s
                   then False
                   else if Set.isSingleton s
                   then Set.elemAt 0 s
                   else Set.foldr Or (Set.elemAt 0 s) (Set.deleteAt 0 s);
  fun set_conj s = if Set.null s
                   then True
                   else if Set.isSingleton s
                   then Set.elemAt 0 s
                   else Set.foldr And (Set.elemAt 0 s) (Set.deleteAt 0 s);
in
  fun dnf fm = list_disj (Set.map list_conj (simpdnf fm))
end;
\end{sml}
\end{node}
\end{node}
\end{definition}

\begin{definition}\label{prop-000U}%
A proposition is in \define{Conjunctive Normal Form} if it is the
conjunction of disjunctions of literals, i.e., if it is of the form
\begin{subequations}
\begin{equation}
C_{1}\land C_{2}\land\dots\land C_{m}
\end{equation}
where each conjunct $C_{i}$ is of the form
\begin{equation}
\ell_{i_{1}}\lor\ell_{i_{2}}\lor\dots\lor\ell_{i_{n_{i}}},
\end{equation}
and each $\ell_{i_{j}}$ is a literal~\zref{prop-000Q}.
\end{subequations}

\begin{theorem}\label{prop-000V}%
If a proposition is in conjunctive normal form, then it is in negation
normal form.
\end{theorem}

\begin{node}[Transformation to CNF]\label{prop:normal-form-000E}%
\begin{node}[Set-representation]\label{prop:normal-form-000H}%
We can take advantage of the fact that, if we have a proposition $A$,
and if we can write $\neg A$ in DNF as
\begin{subequations}
\begin{equation}
\neg A\iff\bigvee^{m}_{i=1}\bigwedge^{n}_{j=1}\ell_{i,j},
\end{equation}
then de Morgan's laws give us:
\begin{equation}
A \iff\bigwedge^{m}_{i=1}\bigvee^{n}_{j=1}\neg\ell_{i,j}.
\end{equation}
\end{subequations}
This is precisely the CNF for $A$. We can do this using our earlier code
to give the set-representation for the DNF of a proposition~\zref{prop:normal-form-000A},
negating literals~\zref{prop:normal-form-000G}, and de Morgan's laws:
\begin{sml}
(* set_cnf : Prop -> (Prop Set.t) Set.t *)
fun set_cnf A = Set.map (Set.map negate) (set_dnf (nnf (Not A)));
\end{sml}
\end{node}

\begin{node}[Simplification]\label{prop:normal-form-000I}%
We can simplify the set-representation for the CNF of a formula, though
the representations for $\verum$ and $\falsum$ change.
\begin{sml}
fun simp_cnf False = Set.singleton (Set.Empty)
 |  simp_cnf True = Set.Empty
 |  simp_cnf A = let val cs = Set.filter (non trivial) (set_cnf A)
                 in Set.filter (fn c =>
                                  not(Set.exists (fn c' => Set.isSubsetOf c' c)
                                                 cs))
                               cs
                 end;
\end{sml}
\end{node}

\begin{node}[Transformation to CNF Formula]\label{prop:normal-form-000J}%
We can now compose the previous functions with transforming the
set-representation back to a \lstinline[basicstyle=\color{sbase03}\ttfamily\small,language=SML]{Prop}
instance, just as we did for DNF~\zref{prop:normal-form-000D} with
similar helper functions:
\begin{sml}
(* cnf : Prop -> Prop *)
local
  fun set_disj s = if Set.null s
                   then False
                   else if Set.isSingleton s
                   then Set.elemAt 0 s
                   else Set.foldr Or (Set.elemAt 0 s) (Set.deleteAt 0 s);
  fun set_conj s = if Set.null s
                   then True
                   else if Set.isSingleton s
                   then Set.elemAt 0 s
                   else Set.foldr And (Set.elemAt 0 s) (Set.deleteAt 0 s);
in
  cnf A = set_conj (Set.map set_disj (simp_cnf A))
end;
\end{sml}
\end{node}

\end{node}

\end{definition}


\begin{theorem}
The proposition $A_{1}\iff(A_{2}\iff(\dots\iff A_{n}))$, when converted
to CNF, will have $\bigO(2^{n})$ leaves in its syntax tree.

\begin{node}\label{prop:normal-form-000K}%
This is the ``worst case'' scenario for conversion to CNF, so we'd like
to avoid this particular problem. Instead of using ``CNF on the nose'',
we'll convert to a proposition which is \emph{equisatisfiable} (i.e.,
propositions $A$ and $B$ are equisatisfiable meaning $A$ is satisfiable
if and only if $B$ is satisfiable). This is a weaker notion than
\emph{equivalid} ($A$ is a tautology if and only if $B$ is a tautology).
\end{node}
\end{theorem}


\begin{definition}\label{prop-000W}%
Let $\varphi$ be a proposition.
We say a proposition $\psi$ is $\varphi$ in \define{Definitional Conjunctive Normal Form}
(or ``\textit{Definitional CNF}\,'' for short) if
$\psi$ is equisatisfiable as $\varphi$ (i.e., $\psi$ is satisfiable if
and only if $\varphi$ is satisfiable) and $\psi$ is obtained by finitely
many transformations, each of which introduce a ``fresh'' propositional
variable $X$ which replaces all instances of some subproposition
$\alpha$, and we conjunct onto it $X\iff\alpha$. There are a great many
variations of the procedure.
\end{definition}

\begin{node}[Satisfiability Problem]\label{prop:normal-form-000L}%
Given a proposition $A$, is it a tautology? Can we determine
``mechanically'' that it follows from propositions $B_{1}$, \dots, $B_{n}$?
Equivalently, can we determine $B_{1}\land\dots\land B_{n}\implies A$
is a tautology?

\begin{node}\label{prop:normal-form-000M}%
This is a ``toy problem'' of the more interesting case, where we have a
mathematical proof presented in first-order logic, and we would like the
computer to verify that step $n+1$ really follows logically from steps
$1$, \dots, $n$ (and possibly cited theorems or definitions).
\end{node}
\end{node}

\begin{node}[History]\label{prop:normal-form-000N}%
\begin{node}[History of term NNF]\label{prop-normal-form-0003}%
The term ``negation normal form'' appears to have been coined by the
1960s. It appears in Raymond Smullyan's 1968 book \textit{First-Order Logic}
in exercise 3 (on page 131), and in John Barwise's 1967 doctoral thesis
``Infinitary logic and admissible sets''. But it is unclear to me if
either of these men actually \emph{coined} the term. Certainly, it seems
safe to speculate that it appears \emph{after} 1928, i.e., when Hilbert
coined the terms ``disjunctive normal form'' and ``conjunctive normal form''~\zref{prop-normal-form-0004}.
\end{node}

\begin{node}[History of terms CNF and DNF]\label{prop-normal-form-0004}%
Carnap's ``Testability and Meaning'' (\textit{Philosophy of Science}
\textbf{3} no.4 (1936) 419--471) cites Hilbert (\textit{Logik} [?],
p.63) and separately Carnap's earlier book \textit{Logical Syntax of Language}
(calling it  ``conjunctive standard form'') as the source of conjunctive
normal form. Coincidentally, ``Testability and Meaning'' appears to coin
the term ``disjunctive normal form'' on page 437, citing page 13 of some work by Hilbert (there is no bibliography for ``Testability and Meaning'').
The 1950 English translation of Hilbert and Ackermann's
\textit{Principles of Mathematical Logic} uses the term ``disjunctive
normal form'' (ch.~1, \S6) and ``conjunctive normal form'' (ch~1, \S7).
It seems safe to credit David Hilbert for inventing these notions, and
may be found in this German book published 1928, and probably may be
found in Hilbert's lectures on logic earlier in the 1920s.
\end{node}

\begin{node}[History of definitional CNF]\label{prop-000X}%
It appears the first time the \emph{phrase} ``definitional CNF'' was
used may be found in Beckert and Posegga~\cite{beckert1994lean}. The
idea may be traced back to de la Tour~\cite{de1990minimizing} who, in
turn, improves upon Tsetin~\cite{Tseitin1983}.
\end{node}
\end{node}
\end{node}
