% 0
\begin{node}[Syntax]\label{prop-000J}%
\begin{node}\label{prop-000I}%
The BNF grammar for propositional logic may be written down explicitly
as a half-dozen production rules with variables treated as a
``primitive''. These rules tell us that variables are propositions,
so are negations of propositions, conjunction, disjunction, implication,
and bi-implication of propositions form propositions (respectively).
\begin{center}
\begin{tabular}{rcl}
$\langle$\textit{proposition}$\rangle$ & ::= & $\langle$\textit{variable}$\rangle$\\
& $|$ & $\neg\langle$\textit{proposition}$\rangle$\\
& $|$ & $\langle$\textit{proposition}$\rangle\land\langle$\textit{proposition}$\rangle$\\
& $|$ & $\langle$\textit{proposition}$\rangle\lor\langle$\textit{proposition}$\rangle$\\
& $|$ & $\langle$\textit{proposition}$\rangle\implies\langle$\textit{proposition}$\rangle$\\
& $|$ & $\langle$\textit{proposition}$\rangle\iff\langle$\textit{proposition}$\rangle$\\
\end{tabular}
\end{center}
\end{node}

\begin{definition}\label{prop-0002}%
Let $V$ be a (nonempty, countably infinite) set of propositional variables. Then
we inductively define a \define{Proposition} according to the following
rules:
\begin{enumerate}
\item Any propositional variable $A\in V$ is a proposition;
\item If $A$ and $B$ are propositions, then so are $\neg A$, $A\land B$,
  $A\lor B$, $A\implies B$, and $A\iff B$.
\end{enumerate}
Only those expressions constructed by these rules are propositions.

\begin{node}\label{prop-0003}%
We usually keep the set of variables $V$ implicit throughout. So we will
not explicitly refer to it unless absolutely necessary. We will
understand that ``a proposition'' really means ``a proposition over the
set of propositional variables $V$'', and we keep it in the back of our minds.
\end{node}

\begin{node}\label{prop-000G}%
We can encode the syntax of propositional logic using \SML\ as a
recursive data structure. This reflects the tree-like structure to the
grammar of well-formed formulas \zref{prop-000I}.

If we were to do this in production, we would probably place this in a
structure with the various other utility functions.

\begin{sml}
datatype Prop = Atom of string
              | Not of Prop
              | And of Prop * Prop
              | Or of Prop * Prop
              | Implies of Prop * Prop
              | Iff of Prop * Prop;
\end{sml}
\end{node}
\end{definition}

\begin{node}[Precedence]\label{prop-000D}%
We should note that logical connectives have the following binding
strength (from weakest to strongest): $\neg$, $\land$, $\lor$,
$\implies$, $\iff$. These are ``left associative''. We may insert
parentheses to alter it. Some examples:
\begin{enumerate}
\item $A\iff (\neg B)\lor C\implies D$ is parsed as $A\iff(((\neg B)\lor C)\implies D)$.
\item $A\implies B\implies C$ is parsed as $(A\implies B)\implies C$.
\end{enumerate}
\end{node}
\end{node}
