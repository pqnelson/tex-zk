% 5

\begin{node}[Semantics]\label{prop-000K}%
\begin{node}[Connectives]\label{prop-0001}%
  The usual connectives are enumerated. We begin with negation.
\begin{enumerate}
\item If $A$ is
a proposition, then its \define{negation} is denoted $\neg A$. Its truth table is
\[\begin{array}{c|c}
A & \neg A\\ \hline
\verum & \falsum\\
\falsum & \verum\\
\end{array}\]
where $\verum$ and $\falsum$ are the truth values ``true'' and
``false'', respectively.
\item Let $A$ and $B$ be propositions. Their \define{conjunction} is the
  proposition denoted $A\land B$ with truth table
\[\begin{array}{cc|c}
A & B & A\land B\\\hline
\verum  & \verum  & \verum\\
\falsum & \verum  & \falsum\\
\verum  & \falsum & \falsum\\
\falsum & \falsum & \falsum\\
\end{array}\]
We refer to the subpropositions $A$ and $B$ as \define{Conjuncts} of
$A\land B$.
\item Let $A$ and $B$ be propositions. Their \define{Disjunction} is the
  proposition denoted $A\lor B$ and is interpreted as ``Either $A$ or
  $B$ or both''. Its truth table is
\[\begin{array}{cc|c}
A & B & A\lor B\\\hline
\verum  & \verum  & \verum\\
\falsum & \verum  & \verum\\
\verum  & \falsum & \verum\\
\falsum & \falsum & \falsum\\
\end{array}\]
Observe $A\lor B$ is false only when both $A$ and $B$ are false.
The subpropositions $A$ and $B$ are called the \define{Disjuncts} of
$A\lor B$.
\item The conditional ``If $A$, then $B$'' requires some
  justification. Surely ``If $\verum$, then $\falsum$'' should be
  false. But what of the other three possible cases? We should hope ``If
  $\verum$, then $\verum$'' is true. We will adopt the convention that
  ``If $\falsum$, then $B$'' is always true. The truth table may be
  written down as
\[\begin{array}{cc|c}
A & B & A\implies B\\\hline
\verum  & \verum  & \verum\\
\falsum & \verum  & \verum\\
\verum  & \falsum & \falsum\\
\falsum & \falsum & \verum\\
\end{array}\]
We call the subformula $A$ the \define{Antecedent} and the subformula
$B$ the \define{Consequent} (or sometimes \textit{Succedent}) of the
formula $A\implies B$.

\textsc{Caution:} The implication should not be used when considering
counterfactuals or causal laws. As odd as it sounds, neither
counterfactuals nor causal laws are needed in mathematics.
\item Let $A$ and $B$ be propositions. The logical equivalence of $A$
  and $B$, or \define{Biconditional} of $A$ and $B$, is the proposition
  denoted $A\iff B$ and is true only when $A$ and $B$ have the same
  truth values:
\[\begin{array}{cc|c}
A & B & A\iff B\\\hline
\verum  & \verum  & \verum\\
\falsum & \verum  & \falsum\\
\verum  & \falsum & \falsum\\
\falsum & \falsum & \verum\\
\end{array}\]
\end{enumerate}
\end{node}

\begin{node}\label{prop-0004}%
Let $V$ be our set of variables. A \define{Valuation} (or
\textit{Truth Function}) is a function $\rho\colon V\to\{\verum,\falsum\}$
assigning a truth value to each propositional variable. Intuitively,
this corresponds to a row in the truth table.

We will abuse notation and write $\rho(A)$ for any proposition $A$ over
the propositional variables $V$, and understand this to mean it is the
result of looking at the relevant row of the truth table for $A$.

Valuations describe the \emph{semantics} of logic, in the sense
that we are describing a possible configuration of the ``world''.
\end{node}


\begin{node}\label{prop:semantics-0000}%
We can define the semantics of propositional logic using an \lstinline{eval}
function, which is defined using a valuation (i.e., a function
\texttt{Prop -> bool}):

\begin{sml}
(* Prop.eval : 'a Formula -> ('a -> bool) -> bool *)
fun eval False v = false
  | eval True v = true
  | eval (Atom p) v = v(p)
  | eval (Not fm) v = not (eval fm v)
  | eval (And (A,B)) v = (eval A v) andalso (eval B v)
  | eval (Or (A,B)) v = (eval A v) orelse (eval B v)
  | eval (Implies (A,B)) v = not(eval A v) orelse (eval B v)
  | eval (Iff (A,B)) v = (eval A v) = (eval B v)
  | eval _ _ = raise Fail "Invalid propositional formula"; 
\end{sml}
\end{node}

\begin{definition}\label{prop-000L}%
We call a proposition $A$ a \define{Tautology} (or \textit{Logically Valid})
if it evaluates to $\verum$ under every valuation $\rho(A)=\verum$.
That is to say, its truth-table value is ``true'' for \emph{all} rows.

``Validity'' is the term used in first-order logic, but Wittgenstein's
\textit{Tractatus} popularized the term ``Tautology''.
\end{definition}

\begin{node}\label{prop:semantics-0001}%
The implementation of testing a propositional formula as a tautology or
not involves checking every valuation is satisfied:

\begin{sml}
(* Prop.onallvaluations : ((Prop -> bool) -> bool)
                          -> (Prop -> bool)
                          -> Prop list
                          -> bool *)
fun onallvaluations subfn v ([] : Prop list) = subfn v
  | onallvaluations subfn v (((p as (P x))::ps) : Prop list) =
    let fun v' t ((q as (P y)) : Prop) = if (x = y) then t else v(q)
    in (onallvaluations subfn (v' false) ps) andalso
       (onallvaluations subfn (v' true) ps)
    end;

(* Prop.tautology : Prop Formula -> bool *)
fun tautology fm = onallvaluations (eval fm) (fn _ => false) (vars fm);
\end{sml}
\begin{node}\label{prop:semantics-0002}%
We can test this code on Bourbaki's axioms found in their propositional
logic, as stated in the \textit{Theory of Sets}~\cite[(I~\S3.1)]{bourbaki1968theory}, as:
\begin{sml}
(* Bourbaki's axioms *)
local
    val A = Atom (P "A")
    val B = Atom (P "B")
    val C = Atom (P "C")
in
    val s1 = Implies (Or(A,A), A);
    val s2 = Implies (A, Or(A,B));
    val s3 = Implies (Or(A,B), Or(B,A));
    val s4 = Implies (Implies(A,B),
                      Implies(Or(C,A),
                              Or(C,B)));
    val bourbaki_axioms_are_tautologies = Prop.tautology s1 andalso
                                          Prop.tautology s2 andalso
                                          Prop.tautology s3 andalso
                                          Prop.tautology s4;
end;
\end{sml}
\end{node}
\end{node}

\begin{definition}\label{prop-000M}%
We will call a proposition $A$ \define{Satisfiable} if there exists at
least one valuation $\rho$ for which $\rho(A)=\verum$. 

\begin{node}\label{prop-000N}%
All tautologies are also satisfiable.
\end{node}
\end{definition}

\begin{definition}\label{prop-0005}%
We say a proposition $C$ is a \define{Logical Consequence} of
proposition $B$ if every valuation $\rho$ for which $\rho(B)$ is true
also makes $\rho(C)$ true.
\end{definition}

\begin{definition}\label{prop-0006}%
Let $B$ and $C$ be propositions. We will say $B$ and $C$ are
\define{Logically Equivalent} if for every valuation $\rho$ the
truth value of $\rho(B)$ is identical to the truth value of $\rho(C)$.
\end{definition}

\begin{node}\label{prop-0007}%
Let $B$ and $C$ be propositions.
\begin{enumerate}
\item $C$ is a logical consequence of $B$ if and only if $B\implies C$
  is a tautology;
\item $B$ and $C$ are logically equivalent if and only if $B\iff C$ is a tautology.
\end{enumerate}
\end{node}

\begin{definition}\label{prop-0008}%
We will call a proposition a \define{Contradiction} (or
\define{Unsatisfiable}) if there is no valuation for which it is
true. Equivalently, if a proposition is false under every valuation,
then it is a contradiction. 
\begin{node}\label{prop-0009}%
Observe, a proposition $A$ is a contradiction if and only if $\neg A$
is a tautology.
\end{node}

\begin{node}\label{prop:semantics-0003}%
A proposition $A$ is a tautology if and only if $\neg A$ is
unsatisfiable (i.e., $\neg A$ is always false in every valuation).
\end{node}
\end{definition}

\begin{node}\label{prop-000A}%
Let $\tau(x_{1},\dots,x_{n})$ be a tautology involving the $n$
metavariables $x_{1}$, \dots, $x_{n}$ ranging over propositions. Let
$A_{1}$, \dots, $A_{n}$ be any $n$ propositions. Then $\tau(A_{1},\dots,
A_{n})$ is a tautology, which Mendelson calls \textit{Logically True}.

Conversely, given any contradiction $\gamma(x_{1},\dots, x_{m})$
parametrized by $m$ metavariables $x_{1}$, \dots, $x_{m}$ ranging over
propositions, and given any $m$ propositions $B_{1}$, \dots, $B_{m}$, we
see that $\gamma(B_{1},\dots,B_{m})$ is a contradiction (which Mendelson
calls \textit{Logically False}).
\end{node}

\begin{node}\label{prop-000B}%
Let $B$ and $C$ be propositions.
If both $B$ and $B\implies C$ are tautologies, then so is $C$.
\end{node}

\begin{node}\label{prop-000C}%
Let $A(x)$ be a proposition parametrized by a metavariable $x$ ranging
over propositions.
Let $B$ and $C$ be propositions. Then $(B\iff C)\implies(A(B)\iff A(C))$
is a tautology.
\end{node}

\begin{node}[Adequate sets of connectives]\label{prop-000E}%
Every proposition is logically equivalent to a proposition involving the
connectives $\neg$, $\land$, $\lor$. That is to say, the set of
connectives $\{\neg,\land,\lor\}$ is adequate for propositional logic
and the others may be expressed in terms of them.

Furthermore, the sets $\{\neg,\land\}$, $\{\neg,\lor\}$, and $\{\neg,\implies\}$
are adequate for propositional logic (in the sense that any proposition
is logically equivalent to one involving only these connectives). This
is because $A\land B$ is logically equivalent to $\neg(\neg A\lor\neg B)$,
$A\implies B$ is logically equivalent to $(\neg A)\lor B$, and $A\iff B$
is logically equivalent to $(A\implies B)\land(B\implies A)$.

\begin{node}\label{prop:semantics-0004}%
Bourbaki~\cite{bourbaki1968theory}, Hilbert and Ackermann, and Russell
and Whitehead use $\{\neg,\lor\}$ as their primitives and the axioms
\begin{enumerate}
\item $A\lor A\implies A$
\item $A\implies(A\lor B)$
\item $A\lor B\implies B\lor A$
\item $(A\implies B)\implies((C\lor A)\implies(C\lor B))$.
\end{enumerate}
Russell and Whitehead had a fifth axiom proving disjunction is
associative, but Bernays proved (in 1926) it is a consequence of the
other four axioms. 
\end{node}
\end{node}
\end{node}
