% 20
\begin{node}\label{stlc-0000}%
The basic idea to simply-typed $\lambda$ calculus is to introduce a
``label'' we affix to expressions and [bound] variables which tracks the
``type'' of the expression. We restrict application to situations when
then rator ``accepts'' the label of the rand expression. As before, we
partition our study of simply-typed $\lambda$-calculus into two
concerns: its syntax and semantics.
\end{node}

\begin{node}[Syntax]\label{stlc-0001}%
\begin{definition}\label{stlc-0002}%
Let $\mathbb{V}$ be a (nonempty, countably infinite) set of
variables. We define the set of \define{Types} for simply-typed
$\lambda$-calculus inductively using the rules:
\begin{enumerate}
\item For all variables $\alpha\in\mathbb{V}$, we have $\alpha$ be a type;
\item If $T_{1}$ and $T_{2}$ are types, then the arrow type $T_{1}\to T_{2}$
  is a type.
\end{enumerate}
And that's all!

\begin{node}[Remark: type variables]\label{stlc-0004}%
Note that ``type variables'' are disjoint from ``term variables'', so
the set $\mathbb{V}$ of type variables is completely different than the
set of variables $V$ we use with terms. We adopt the informal convention
that lowercase Greek variables are type variables, lowercase Latin
variables are term variables.
\end{node}

\begin{node}[BNF Grammar]\label{stlc-0003}%
The BNF grammar for types may be written down rather quickly as:
\begin{center}
\begin{tabular}{rcl}
$\langle$\textit{type}$\rangle$ & ::= & $\langle$\textit{type variable}$\rangle$\\
& $|$ & $\langle$\textit{type}$\rangle\to\langle$\textit{type}$\rangle$
\end{tabular}
\end{center}
\end{node}

\begin{node}[Base types]\label{stlc-0005}%
Usually we have some base types in simply-typed lambda calculus (like
the type of Booleans, or the type of natural numbers), because we want
to do some computation with them. If we have base types, the grammar for
types is extended to be:
\begin{center}
\begin{tabular}{rcl}
$\langle$\textit{type}$\rangle$ & ::= & $\langle$\textit{type variable}$\rangle$\\
& $|$ & $\langle$\textit{base type}$\rangle$\\
& $|$ & $\langle$\textit{type}$\rangle\to\langle$\textit{type}$\rangle$
\end{tabular}
\end{center}
\end{node}
\end{definition}

\begin{node}\label{stlc-0008}%
Provisionally, we have a type declaration convey the intuition that a
term or expression may be labelled as having a certain type. This is
written down as ``$e\esti T$'' to assert ``Expression $e$ has type
$T$''. The colon (``$\esti$'') is called ``\textit{esti}'' but people
usually read it as ``has type''.
\end{node}

\begin{definition}\label{stlc-0006}%
Let $V$ be a (nonempty countably-infinite) set of variables, let $\mathbb{V}$ be a (nonempty countably-infinite)
set of type variables disjoint from $V$. We define the
\define{Simply-Typed Terms} in the Simplify-Typed $\lambda$-calculus inductively by
the rules
\begin{enumerate}
\item For any variable $x\in V$, we have $x$ be a simply-typed term;
\item For any type $T$, variable $x\in V$, and simply-typed term $e$, we
  have $\lambda x\esti T.e$ be a simply-typed term;
\item For any simply-typed terms $e_{1}$ and $e_{2}$, we have
  $e_{1}~e_{2}$ be a simply-typed term.
\end{enumerate}
\end{definition}

\begin{definition}\label{stlc-0009}%
A \define{Typing Context} is a finite ordered list of type declarations,
which we may inductively define by the rules:
\begin{enumerate}
\item The empty list $\emptylist$ is a typing context;
\item For any variable $x\in V$ and any type $T$ and any typing context
  $\Gamma$ which does not contain $x\notin\Gamma$,
  we have ``$x\esti T,\Gamma$'' be a typing context.
\end{enumerate}
Note that it is important to keep in mind $\Gamma$ is an \emph{ordered}
finite list without duplicate variables. In future type theories, this
will be a critical assumption.

\begin{node}\label{stlc-000E}%
We will write $\dom(\Gamma)$ for the set of variables. Inductively, this
is defined by $\dom(\emptylist)=\emptyset$ and $\dom(x\esti T,\Gamma)=\{x\}\cup\dom(\Gamma)$.
\end{node}

\begin{node}\label{stlc-000F}%
Let $\Gamma$, $\Gamma'$ be two typing contexts. We will write
$\Gamma\subset\Gamma'$ to define the predicate defined inductively by:
\begin{enumerate}
\item $x\esti T,\Gamma\subset \Gamma'$ if $\Gamma'=\Gamma_{0}',x\esti T,\Gamma_{1}'$
  and $\Gamma\subset\Gamma_{1}'$;
\item $\emptylist\subset\Gamma'$ for any typing context $\Gamma'$.
\end{enumerate}
\end{node}

\begin{node}[Exercise]\label{stlc-000G}%
Prove or find a counter-example: $\Gamma\subset\Gamma'$ and
$\Gamma'\subset\Gamma$ if and only if $\Gamma=\Gamma'$. In other words,
the familiar criteria of extensionality works for typing contexts.
\end{node}
\end{definition}

\begin{node}[Typing Judgement]\label{stlc-0007}%
We have a judgement to assert that an expression belongs to a type,
relative to a typing context telling us how to interpret the free
variables appearing in the expression. This is written as
$\Gamma\vdash e\esti T$ where $\Gamma$ is a typing context, $e$ is a
simply-typed expression, and $T$ is a type. We now need to introduce
inference rules to derive typing judgements. When $\Gamma=\emptyset$, we
just write $\vdash e\esti T$.

\begin{node}\label{stlc-000A}%
We would have, if $x\esti T$ appears somewhere in a typing context
$\Gamma$, then $\Gamma\vdash x\esti T$ be well-typed. This is a
tautology, like ``$A$ is $A$''.
\end{node}

\begin{node}\label{stlc-000B}%
The abstraction rule tells us, if we have established that
$\Gamma,x\esti T_{1}\vdash e\esti T_{2}$, then the expression $e$ is
parametrized by a variable of type $T_{1}$. Whenever a value $v_{1}$ of type
$T_{1}$ is substituted into $e[v_{1}/x]$, we should obtain an expression
of type $T_{2}$. This means we can form the abstraction $\lambda x\esti T_{1}.e$
to be a term of a function type $T_{1}\to T_{2}$ relative to the context
$\Gamma$. As an inference rule,
\begin{equation}
\infer[\Rule{T-Abs}]{\Gamma\vdash(\lambda x\esti T_{1}.e)\esti T_{1}\to T_{2}}{%
\Gamma,x\esti T_{1}\vdash e\esti T_{2}}
\end{equation}
\end{node}

\begin{node}\label{stlc-000C}%
The application rule tells us, if we have a function $f\esti T_{1}\to T_{2}$
and an argument $e\esti T_{1}$, then we should expect the application
$f~e\esti T_{2}$. This has become so commonplace in modern mathematics,
it hardly deserves justification.
\end{node}

\begin{node}[Inference rules]\label{stlc-000D}%
The inference rules for type judgements takes the form
\begin{subequations}
\begin{equation}
\infer[\Rule{T-Var}]{\Gamma\vdash x\esti T}{x\esti T\in\Gamma}
\end{equation}
\begin{equation}
\infer[\Rule{T-Abs}]{\Gamma\vdash(\lambda x\esti T_{1}.e)\esti T_{1}\to T_{2}}{%
\Gamma,x\esti T_{1}\vdash e\esti T_{2}}
\end{equation}
\begin{equation}
\infer[\Rule{T-App}]{\Gamma\vdash e_{1}~e_{2}\esti T_{2}}{\Gamma\vdash e_{1}\esti T_{1}\to T_{2}
& \Gamma e_{2}\esti T_{2}}
\end{equation}
%% We can also add arbitrary bindings of free variables without changing
%% the typing judgement:
%% \begin{equation}
%% \infer[\Rule{T-Weak}]{\Gamma'\vdash e\esti T}{\Gamma\subset\Gamma' & \Gamma\vdash e\esti T}
%% \end{equation}
\end{subequations}
\end{node}
\end{node}

\begin{node}\label{stlc-000H}%
Observe that typing judgements are \emph{syntactic} in concern and
nature. It is telling us that some string (some word or phrase) may be,
in some coherent way, attached with a ``label''. In practice, this is
how programming languages track context-sensitive information (like
we're adding two numbers together, as opposed to concatenating lists).
\end{node}

\begin{theorem}[Uniqueness of type]
For any typing context $\Gamma$, any simply-typed expression $e$ has at
most one type $T$; i.e., there is at most one $T$ such that
$\Gamma\vdash e\esti T$ is derivable, if any derivation exists at all.
\end{theorem}

\begin{proof}[Proof sketch]
Either there is a derivation of $\Gamma\vdash e\esti T$ or not. If not,
then there are zero types $T$ describing $e$, and that proves the claim
in one case.

The other case, assume that there is a derivation of $\Gamma\vdash e\esti T$.
Then we can prove by induction on derivations that there is at most one
type $T$ such that $\Gamma\vdash e\esti T$ is derivable.
\end{proof}

\begin{theorem}[Weakening]
If $\Gamma\vdash e\esti T$ is derivable and $x\notin\Gamma$, then
for any type $T'$ we have $\Gamma,x\esti T'\vdash e\esti T$ be derivable too.
\end{theorem}

\end{node}

\begin{node}[Semantics]\label{stlc-000I}%

\begin{node}\label{stlc-000J}%
The semantics of a simply-typed $\lambda$-calculus amounts to the same
semantics of $\lambda$-calculus (using $\beta$-reduction to give meaning
to the language), except we typecheck expressions \emph{before}
evaluating them. That is to say, we need to make sure that terms are
well-typed, and then we can perform $\beta$-reduction. So, in
particular, the interesting thing is how typing judgements behave under
$\beta$-reduction. 
\end{node}

\begin{theorem}[Progress]
If $t$ is a closed, well-typed term (i.e., $\vdash t\esti T$ for some
type $T$), then either $t$ is a value or there is some term $t'$ such
that $t\betareduce t'$.
\end{theorem}

\begin{theorem}[Preservation]
If $\Gamma\vdash e\esti T$ and $e\betareduce e'$, then we have
$\beta$-reduction preserve the typing judgement, i.e., $\Gamma\vdash e'\esti T$.
\end{theorem}
  
\end{node}
