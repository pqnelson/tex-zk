% 5
\chapter{Isabelle}

\begin{node}\label{isabelle-0000}%
The basic idea of Isabelle is that it is a proof assistant based on a
fragment of intuitionistic higher-order logic (called the ``meta-logic''
and denoted $\mathcal{M}$), which serves as a logical
framework for implementing other deductive systems. The most developed
deductive system appears to be Isabelle/HOL.

\begin{node}[Terminology]\label{isabelle-0002}%
The convention is to refer to the Isabelle implementation of deductive
system $X$ as ``Isabelle/$X$''. For example, Isabelle/HOL implements
higher-order logic, Isabelle/FOL for first-order logic, Isabelle/ZF for
Zermelo--Fraenkel set theory, and so on. People call the system $X$ the
``object logic'' (analogous to the ``metalanguage'' and ``object
language'' distinction).

Confusingly, Isabelle/Pure is the name given to the code implementing
$\mathcal{M}$ in Isabelle.
\end{node}

\begin{node}[References]\label{isabelle-0001}%
Paulson~\cite{paulson1988foundation} first sketched out $\mathcal{M}$
and implemented Isabelle. (Really, Paulson has been the heart and dynamo
behind Isabelle.)
Ro{\ss}kopf and Nipkow~\cite{Rosskopf2022formalization} have formalized
$\mathcal{M}$ inside Isabelle/HOL, producing a validated version of
Isabelle (which runs extremely slowly). 
\end{node}
\end{node}

\begin{node}[Isabelle/ZF]\label{isabelle-0003}%
As best as I can determine, Isabelle/ZF builds on No\"{e}l's
``proof-of-concept'' work~\cite{Noel1993experimenting}. Specifically,
using Isabelle/FOL, the \ZF\ axioms are implemented. Details may be
found in Paulson's reviews~\cite{Paulson1993set,Paulson1995set}. Paulson
later~\cite{Paulson2003ac} mechanized the proof of the relative
consistency of the axiom of choice. Gunther and friends~\cite{gunther2020formalization}
have recently formalized results concerning forcing using Isabelle/ZF.

Paulson notes~\cite[\S2.2]{Paulson1993set} that in Isabelle's set
theory, ``\texttt{i}'' is the type of sets and ``\texttt{o}'' is the
type of formulas (I think this is a nod to Church's usage of $\iota$ for
individuals and $o$ for propositions). 
\end{node}

\begin{node}[Getting started]\label{isabelle-0004}%
When we formalize something in Isabelle, they are organized into files
ending with ``\texttt{.thy}'' (they're called ``theory files''). The
basic template for such files look like (in a file named ``\texttt{example.thy}''):
\begin{Isabelle}
theory example
imports Theory_1 ... Theory\_n
begin

...

abbreviation mizeq :: "Set \<Rightarrow> Set \<Rightarrow> o" (infixl "=" 50)
  where "mizeq \<equiv> IFOL.eq"


ML \<open>
val basic_ss = FOL_basic_ss;
val main_ss = FOL_ss;
val mk_Trueprop = \<^make_judgment>;
val dest_Trueprop = \<^dest_judgment>;
\<close>
  
end
\end{Isabelle}
The ``code'' goes in between the ``begin'' and ``end'' segment of the file.

We need to use Isabelle's jEdit \textsc{ide}, though there is a small
community which still use Emacs's proof general.
\end{node}

\transclude{isabelle/metalogic}
\transclude{isabelle/fol}
\transclude{isabelle/hol}
