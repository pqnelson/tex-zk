% 7

\chapter{Interactive Theorem Prover}

\section{Consistency}

\begin{node}[Pollack consistency]\label{itp-0001}%
I am following Wiedijk~\cite{wiedijk2012pollack}. However, I will
renaming Wiedijk's $\mathtt{print}_{*}$ function using Haskell convention
\begin{subequations}
\begin{equation}
\mathsf{show}_{*}\colon *\to\mathtt{string}.
\end{equation}
We also have
\begin{equation}
\mathsf{parse}_{t}\colon\mathtt{string}\to\mathtt{term}\cup\{\mathrm{FAIL}\}
\end{equation}
which returns the ``FAIL'' value when the input string is invalid, and
(when formulas are distinguished in the system)
\begin{equation}
\mathsf{parse}_{f}\colon\mathtt{string}\to\mathtt{formula}\cup\{\mathrm{FAIL}\}
\end{equation}
\end{subequations}
which parses strings for terms or formulas.


\begin{definition}\label{itp-0002}%
The functions $\mathsf{parse}_{t}$ and $\mathsf{show}_{t}$ are
\define{compatible} if the output of $\mathsf{show}_{t}$ is always
contained in the domain of $\mathsf{parse}_{t}$: for all terms $t$, we
have $\mathsf{parse}_{t}\bigl(\mathsf{show}_{t}(t)\bigr)$ is defined.
\end{definition}

\begin{definition}\label{itp-0003}%
The functions $\mathsf{parse}_{t}$ and $\mathsf{show}_{t}$ are
\define{well-behaved} if parse is the left-inverse of print, i.e., if 
for all terms $t$ we have 
$\mathsf{parse}_{t}\bigl(\mathsf{show}_{t}(t)\bigr)=t$.

\begin{node}\label{itp-0004}%
Note that $\mathsf{parse}_{t}$ is generally not the right-inverse of
$\mathsf{show}_{t}$ because, e.g.,
\begin{equation}
\mathsf{parse}_{t}(\texttt{"x+y"}) = \mathsf{parse}_{t}(\texttt{"x + y"})
=\mathsf{parse}_{t}(\texttt{"(x + y)"})
\end{equation}
despite the fact that $\texttt{"x+y"}\neq\texttt{"x + y"}\neq\texttt{"(x + y)"}$.
\end{node}
\end{definition}

\begin{definition}\label{itp-0005}%
We call the function $\mathsf{parse}_{t}$ \define{input-complete} if for
all terms $t$ there exists at least one string $s$ such that
$\mathsf{parse}_{t}(s)=t$; i.e., if every term of the system can be
written in the input language.
\end{definition}


\begin{definition}\label{itp-0000}%
Generically, for an interactive proof assistant, let
\[\mathsf{show}_{f}\colon\mathsf{formula}\to\mathsf{string}\]
(assuming the system can distinguish formulas from terms) and 
\[\mathsf{show}_{t}\colon\mathsf{term}\to\mathsf{string}\]
be the functions used to print formulas and objects to a string (used right
before printing them to the screen). The \define{Pollack-axioms} are
\begin{enumerate}
\item For all formulas $\phi_{1}$ and $\phi_{2}$, if
  $\mathsf{show}_{f}(\phi_{1})=\mathsf{show}_{f}(\phi_{2})$, then we
  have $\phi_{1}\iff\phi_{2}$; and
\item For all terms $t_{1}$ and $t_{2}$, if $\mathsf{show}_{t}(t_{1})=\mathsf{show}_{t}(t_{2})$,
  then $t_{1}=t_{2}$ using the default equality for the system (e.g.,
  for Coq this would be Leibniz equality).
\end{enumerate}
\end{definition}

\begin{definition}\label{itp-0006}%
A proof assistant is called \define{Pollack-inconsistent} if from a
finite number of Pollack-axioms of the system it is possible to derive a
contradiction in the proof assistant.
\end{definition}
\end{node}
