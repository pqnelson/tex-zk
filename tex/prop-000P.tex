\begin{node}[Normal forms]\label{prop-000P}%
A common trick in logic is to take some sort of ``normal form'' of
formulas. That is to say, we restrict attention to a sublanguage of
propositional logic such that (a) it is constructed using an adequate
set of connectives~\zref{prop-000E}, and (b) it simplifies proving if a
proposition is satisfiable or unsatisfiable. (This is far too applied a
topic for Mendelson~\cite{mendelson2015mathematical}, and I am relying on
Harrison~\cite{harrison2009handbook} for much of the material here.)

\transclude{prop-000Q}
\transclude{prop-000O}
\transclude{prop-000R}
\transclude{prop-000S}
\transclude{prop-000U}
\transclude{prop-000W}
\end{node}
