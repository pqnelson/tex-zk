% 11
\section{Tarski--Grothendieck Axioms}

\begin{node}[Axioms]\label{tg-0003}%
The axioms to Tarski--Grothendieck set theory, besides the ontological
axiom ``For all objects $x$, we have $x$ is a set'', boil down to the
four axioms plus one axiom schema:

\begin{node}[Axiom of extensionality]\label{tg-0004}%
For any sets $X$ and $Y$, if every object $x$ satisfies $x\in X$ iff
$x\in Y$, then $X=Y$.
\end{node}

\begin{node}[Axiom of Pair]\label{tg-0005}%
For any objects $x$ and $y$, there exists a set $z$ such that for all
objects $a$ we have $a\in z$ if and only if $a=x$ or $a=y$.
\end{node}

\begin{node}[Axiom of union]\label{tg-0006}%
For any set $X$ there exists a set $Z$ such that for all objects $x$ we
have $x\in Z$ if and only if there exists a set $Y$ such that $x\in Y$
and $Y\in X$.
\end{node}

\begin{node}[Axiom of regularity]\label{tg-0007}%
For all objects $x$, for all sets $X$, if $x\in X$, then there exists a
set $Y$ such that $Y\in X$ and for all objects $z$ either $z\notin X$ or
$z\notin Y$.
\end{node}

\begin{node}[Fraenkel's scheme]\label{tg-0008}%
Let $\mathcal{A}$ be a set, let $\mathcal{P}[-,-]$ be a binary predicate
of objects. Suppose for all objects $x$, $y$, $z$ if $\mathcal{P}[x,y]$ and
$\mathcal{P}[x,z]$, then $y=z$.
Then there exists a set $X$ such that for all objects $x$ we have $x\in X$
if and only if there exists an object $y\in\mathcal{A}$ such that $\mathcal{P}[y,x]$.
\end{node}

\begin{node}[Axiom of universe]\label{tg-0009}%
For any set $x$, there exists a Tarski universe containing it.
\end{node}
\end{node}

\begin{node}\label{tg-0001}%
There are various different definitions for a Grothendieck universe.
\begin{definition}\label{set:tg-0000}%
A \define{Grothendieck universe}\footnote{Nlab definition accessed
December 30, 2023 \url{https://ncatlab.org/nlab/revision/Grothendieck+universe/57}} is a set $U$ such that
\begin{enumerate}
\item $U$ is transitive: for all $u\in U$ and $t\in u$, we have $t\in U$;
\item for all $u\in U$, we have $\powerset(u)\in U$;
\item $\emptyset\in U$ (some authors demand $\NN\in U$ instead);
\item for all $I\in U$ and functions $u\colon I\to U$, we have
  $\union_{i\in I}u_{i}\in U$.
\end{enumerate}
\end{definition}
\begin{definition}
A \define{Tarski Universe} is a set $U$ such that
\begin{enumerate}
\item for all sets $x$ and $y$, if $x\in U$ and $y\subset x$, then $y\in U$;
\item for any set $x\in U$, we have $\powerset(x)\in U$;
%% \item $U$ is transitive: for any set $x\in U$, every subset $y\subset x$
%%   is a member $y\in U$;
\item for any set $x$, if $x\subset U$ and $x\not\equipotent U$ are not equipotent, then $x\in U$.
\end{enumerate}
\begin{node}[Remarks]\label{tg-0002}%
It is not hard to prove that Grothendieck universes are Tarski universes.
Conversely, if we have a Tarski universe containing a given set $x$,
then it is contained in a Grothendieck universe containing $x$. This has
been formalized in Mizar articles \texttt{CLASSES1}, \texttt{CLASSES2},
and \texttt{CLASSES3}.

If we consider the ``smallest'' (as ordered by inclusion) transitive
Tarski universe containing $\emptyset$, then we can view this as
$\properclass{U}_{fin}$ the collection of finite sets. The ``smallest''
transitive Tarski universe containing $\properclass{U}_{fin}$ may be
viewed as the collection of all sets $\properclass{U}_{sets}$. We can
continue iterating this procedure, giving us a way to construct
models of ``classes'', ``conglomerates'', and so on. Everything we said
also works for the ``smallest'' Grothendieck universes satisfying the
given conditions. In particular, the smallest Grothendieck containing $\union\properclass{U}_{sets}$
is a model of \MK\ set theory (and thereby \NBG\ set theory).
\end{node}
\end{definition}
\end{node}

\begin{node}\label{tg-000A}%
The development of set theory using these axioms may be found in the
first 30 articles (or so) of the Mizar mathematical library. We will not
reproduce the work here.
\end{node}
