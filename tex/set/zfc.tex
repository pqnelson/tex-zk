% 22
\section{ZFC Set Theory}
\begin{node}[Axioms]\label{set:zfc-0000}%
Let me review the axioms as enumerated in the order Wikipedia gives us,
in a close approximation to the Mizar language:

\begin{node}[Extentionality]\label{set:zfc-0001}%
For all sets $X$ and $Y$, if every object $x$ satisfies $x\in X$ iff
$x\in Y$, then $X=Y$.
\end{node}

\begin{node}[Regularity]\label{set:zfc-0004}%
For any object $x$ and for any set $X$ such that $x\in X$,
there exists a set $Y$ such that $Y\in X$ and for all objects $z$ either
$z\notin X$ or $z\notin Y$.
\end{node}

\begin{node}[Schema of Specification]\label{set:zfc-0005}%
Let $\schematicvar{A}$ be a set, let $\schematicvar{P}[-]$ be a unary predicate of
objects. There exists a set $X$ such that for all objects $x$ we have
$x\in X$ if and only if $x\in\schematicvar{A}$ and $\schematicvar{P}[x]$.
\end{node}

\begin{node}[Pairing]\label{set:zfc-0002}%
For all objects $x$ and $y$, there exists a set $z$ such that for all
objects $a$ we have $a\in z$ if and only if $a=x$ or $a=y$.
\end{node}

\begin{node}[Union]\label{set:zfc-0003}%
For any set $X$, there exists a set $Z$ such that for all objects $x$ we
have $x\in Z$ if and only if there exists a set $Y$ such that $x\in Y$
and $Y\in X$.
\end{node}

\begin{node}[Schema of replacement]\label{set:zfc-0006}%
Let $\schematicvar{P}[-,-]$ be a binary predicate of sets, let $\schematicvar{A}$
be a set. There exists a set $X$ such that for all objects $x$ we have
$x\in X$ if and only if there exists an object $y\in\schematicvar{A}$ such
that $\schematicvar{P}[y,x]$; provided:
\begin{enumerate}
\item For all objects $x$, $y$, $z$, if $\schematicvar{P}[x,y]$ and
  $\schematicvar{P}[x,z]$, then $x=z$.
\end{enumerate}
\end{node}

\begin{node}[Infinity]\label{set:zfc-0007}%
There exists a set $X$ such that
\begin{enumerate}[label=(\roman*)]
\item there exists a set $e$ such that for all objects $z$ we have
  $z\notin e$ and $e\in X$; and
\item for all sets $y$ such that $y\in X$ there exists a set $s$ such
  that $s\in X$ and for all objects $z$ we have $z\in s$ iff $z=y$ or
  $z\in y$.
\end{enumerate}
Informally, in (ii), $s=\{y\}\cup y$ is the successor of $y$.
\end{node}

\begin{node}[Power set]\label{set:zfc-0008}%
For all sets $X$, there exists a set $P$ such that for all sets $Z$ we
have $Z\in P$ if and only if for all objects $z$ such that $z\in Z$
satisfies $z\in X$.
\end{node}

\begin{node}[Choice]\label{set:zfc-0009}%
For any set $X$ such that there exists an object $x\in X$,
there exists a function $f\colon X\to\union_{A\in X}A$ such that for
all sets $A$ such that $A\in X$ satisfies $f(A)\in A$.

\begin{node}\label{set:zfc-000A}%
This form of the axiom of choice requires defining the union of a family
of sets $\union_{A\in X}A$ and functions.
\end{node}

\begin{node}[Alternative version]\label{set:zfc-000B}%
If we stipulate the shorthand $\exists!x,P[x]$ for
$(\exists x)(P[x]\land((\forall y)(P[y]\implies x=y)))$, then we could
also borrow Freek Wiedijk's formalization\footnote{From his Automath
code for the paper ``Is ZFC a hack?''~\cite{wiedijk2006zf} \url{http://www.cs.ru.nl/~freek/zfc-etc/zfc.aut}}:

For all sets $X$, if every set $Y\in X$ is nonempty, 
and if for all sets $Y_{1}\in X$ and $Y_{2}\in X$ we have either
$Y_{1}$ misses $Y_{2}$ or $Y_{1}=Y_{2}$,
then there exists a set $Z$ such that for all sets $Y\in X$ there exists
a unique $Y'\in X'$ such that $Y'\in Y$.

This requires defining the predicate ``misses'' for sets.
\end{node}

\begin{node}[Well-ordering version]\label{set:zfc-000C}%
If we have the definition of ordered pairs $(x,y)=\{\{x\},\{x,y\}\}$
and relations, then we can use Zermelo's theorem to state an equivalent
form of the axiom of choice: for all sets $X$, there exists a relation
$R$ such that $R$ well-orders $X$.

We say $R$ well-orders $X$ if $R$ is reflexive, transitive,
antisymmetric, connected [i.e., for all $x,y\in X$ if $x\neq y$ and
  $(x,y)\notin R$, then $(y,x)\in R$], and is well-founded [i.e., for
  all sets $Y\subset X$ such that $Y\neq\emptyset$ there exists an
  object $a\in Y$ such that $\{x\in X\mid(x,y)\in R\}$ misses $Y$].

This requires quite a bit of more work than Wiedijk's version.
\end{node}
\end{node} % axiom of choice

\begin{node}[Remarks]\label{set:zfc-000D}%
\begin{node}\label{set:zfc-000E}%
The Tarski--Grothendieck axioms \zref{sec:set:tarski-grothendieck}
either coincide with the \ZFC\ axioms, or can recover the \ZFC\ axioms
as theorems. We could prove that $\ZF+(\mbox{Universe})$ (i.e., the
\ZF\ axioms plus the axiom ``For each set $X$, there exists a Grothendieck universe~\zref{tg-0001}
$\mathcal{U}$ containing $X\in\mathcal{U}$'') is equal in strength to
the Tarski--Grothendieck axioms.
\end{node} % universes

\begin{node}[Classes]\label{set:zfc-000F}%
We cannot form the ``collection of all sets'' in \ZFC\ since the only
sort of collection which exists is a set. This normally leads
mathematicians to consider ``classes'' as ``bigger collections''.
\ZFC\ cannot work directly with classes. There are two possible
remedies:
\begin{enumerate}
\item Although \ZFC\ doesn't treat classes as ``first class entities'',
we can still meaningfully work with classes as formulas. Jech~\cite{jech2003set}
does this, as does the Metamath proof assistant\footnote{See \url{https://us.metamath.org/mpegif/mmset.html\#class}}. (This seems to be
traced back to Quine's \textit{Set Theory and its Logic} (1969), but I
do not have access to the book to verify that claim.)
\item If we wanted to work with classes directly, we would either turn to
\NBG~\zref{sec:set:nbg} or \MK\ axioms. When working in category theory,
it turns out to be insufficient, and so \emph{conglomerates} are
introduced to formalize the notion of ``the collection of all classes''.
Unsurprisingly, we find we need to form ``the collection of all conglomerates'',
which leads to the notion of a \emph{cartel}\footnote{I actually have
not seen this term used too frequently, see Horst Herrlich and George
Strecker's \textit{Category Theory} (2007), chapter 2.}. The astute reader will
realize we are ``iterating the same construction'', and this is what
leads us to consider Tarski--Grothendieck set theory.\footnote{We
usually take ``the smallest'' Grothendieck universe $\mathcal{U}$
containing the natural numbers $\NN\in\mathcal{U}$ to be the universe of
sets, which corresponds to $\mathcal{U}=V_{\kappa}$ for some inaccessible
$\kappa$. Then $V_{\kappa+1}$ is the collection of all classes,
$V_{\kappa+2}$ is the collection of all conglomerates, and so
on. Although to be frank, we could weaken our demands to $\kappa$ being
a worldly cardinal rather than an inaccessible cardinal.}
\end{enumerate}
\end{node} % classes

\begin{node}[Independence of axioms]\label{set:zfc-000I}%
An axiom is ``independent'' of the others if we can negate the axiom
without producing an inconsistent theory.
The axiom of choice~\zref{set:zfc-0009} is independent of the other
axioms of \ZF\ set theory. Curiously, the axiom of
regularity~\zref{set:zfc-0004} is independent of the other axioms of
\ZF\ set theory.

Usually we prove claims of independence using forcing techniques\index{Forcing}. The
basic idea underlying forcing techniques is to start with a model $M$ of
the axioms, then adjoin an element $U$ to it forming a new model $M[U]$.
This is analogous to adjoining an element $X$ to a ring $R$ to form a
new ring $R[X]$, except it's considerably more complicated for models of
\ZF\ since the axioms of \ZF\ are considerably more tricky than
rings. We are forced to do this slowly and explicitly, constructing this
new element $U$ one step at a time, tracking what properties would be
``forced'' to hold at each step (hence the name
``forcing''). Chow~\cite{chow2008beginners} reviews the ``big picture''
of forcing.
\end{node}
\end{node} % remarks
\end{node} % axioms

\begin{node}[{Consistency of {\ZFC}}]\label{set:zfc-000G}%
A natural question, when given an axiomatic system, is to ask if it is
consistent. Unfortunately, G\"{o}del proved it is impossible for a
system to determine its own consistency (unless it is inconsistent). The
best we can hope for is \emph{relative consistency}. But there has been
some interesting developments lately.

\begin{node}[Using Turing machines]\label{set-0002}%
There is a 744-state Turing machine which halts if and only if \ZFC\ is
inconsistent. The explicit specification of the Turing
machine may be found on Github\footnote{\url{https://github.com/sorear/metamath-turing-machines/blob/master/zf2.nql}}
and follows Tarski's system of predicate logic with equality, and uses
just ZF minus regularity (and no axiom of choice) because this is
equiconsistent with \ZFC.

\begin{node}[Axiomatization]\label{set:zfc-000J}%
Basically, it's a Hilbert proof system whose inference rules are
``detachment'' (\textit{modus ponens}) and generalization. The axioms
are
\begin{itemize}
\item[(B1)] $(\varphi\implies\psi)\implies((\psi\implies\chi)\implies(\varphi\implies\chi))$
\item[(B2)] $(\neg\varphi\implies\varphi)\implies\varphi$
\item[(B3)] $\varphi\implies(\neg\varphi\implies\psi)$
\item[(B4)] $\bigl((\forall x)(\varphi\implies\psi)\bigr)\implies\bigl(((\forall x)(\varphi))\implies((\forall x)(\psi))\bigr)$
\item[(B6a)] $x=y\implies\bigl((\forall z)(x=y\land x\in y)\implies(\forall z)(x\in y)\bigr)$
  where $x\neq z$ and $y\neq z$
\item[(B6b)] $\bigl((\forall x)(\forall y)(\varphi)\bigr)\implies\bigl((\forall y)(\forall x)(\varphi)\bigr)$
\item[(B6c)] $((\exists x)(\forall x)(\varphi))\implies\varphi$
\item[(B7)] $(\exists x)(x = y)$
\item[(B8a)] $x=y\implies(x=z\implies y=z)$
\item[(B8b)] $x=y\implies(x\in z\implies y\in z)$
\item[(B8c)] $x=y\implies(z\in x\implies z\in y)$
\item[(EXT)] $(\forall v_{2})(v_{2}\in v_{0}\iff v_{2}\in v_{1})\implies v_{0}=v_{1}$
\item[(REP)] $(\forall v_{3})(\exists v_{1})(\forall v_{2})((\forall v_{1})(\varphi)\implies v_{2}=v_{1})\implies(\exists v_{1})(\forall v_{2})\bigl(v_{2}\in v_{1}\iff(\exists v_{3})(v_{3}\in v_{0}\land(\forall v_{1})(\varphi))\bigr)$
\item[(POW)] $(\exists v_{1})(\forall v_{2})\bigl((\forall v_{3})(v_{3}\in v_{2}\implies v_{3}\in v_{0})\implies v_{2}\in v_{1}\bigr)$
\item[(UNI)] $(\exists v_{1})(\forall v_{2})((\exists v_{3})(v_{2}\in v_{3}\land v_{3}\in v_{0})\implies v_{2}\in v_{1})$
\item[(INF)] $(\exists v_{1})(v_{0}\in v_{1}\land(\forall v_{0})(v_{0}\in v_{1}\implies(\exists v_{2})(v_{2}\in v_{1}\land(\forall v_{1})(v_{1}\in v_{2}\iff v_{1}=v_{0}))))$
\end{itemize}
\end{node}

\begin{node}[Encoding]\label{set:zfc-000K}%
The formulas are encoded as integers. We use the Cantor pairing
\begin{equation}
(X~.~Y):= X + \frac{1}{2}(X+Y)(X+Y+1),
\end{equation}
which is a bijection on non-negative integers. This notation is extended
as
\begin{subequations}
\begin{align}
(X~Y~Z) &:= (X~.~(Y~.~Z))\\
(Z) &:= (Z~.~0)\\
(X;~Y) &:= (X~.~(;~Y)) = (X~.~1+(Y~.~0)).
\end{align}
\end{subequations}
Well-formed formulas $\varphi$ are recursively numbered using
$|\varphi|$ which we define as:
\begin{subequations}
  \begin{align}
    |v_{i}=v_{j}| &= ((i~.~j)~.~0)\\
    |v_{i}\in v_{j}| &= ((i~.~j)~.~1)\\
    |\varphi\implies\psi| &= ((|\varphi|~.~|\psi|)~.~2)\\
    |\neg\varphi| &= (|\varphi|~.~3)\\
    |(\forall v_{i})(\varphi)| &= ((i~.~|\varphi|)~.~4).
  \end{align}
\end{subequations}
Observe that the well-formed formula with numeric value $0$ is $|v_{0}=v_{0}|$.
The well-formed formula with numeric value $1$ is $|v_{0}\in v_{0}|$
which is not a theorem of \ZF. This serves as the canonical
contradiction, and the Turing machine searches for a proof of $1$ (i.e.,
it halts when an inconsistency is found).
\end{node}

\begin{node}\label{set:zfc-000L}%
The architecture of the Turing machine itself, it keeps a stack of
theorems. A proof is a list of integers taken in groups of 4 --- we
could think of them as ``axiom codes'' [opcodes] and three parameters. 
Proofs manipulate the stack of theorems, inference rules pop one (or
two) theorems off and then push the inferred theorem back onto the stack.
Axioms push theorems onto the stack. There are 18 axiom codes for the 2
inference rules and 16 axioms of logic and set theory.

Well-formed formulas are constructed using a sequence of calls to
functions which manipulate the ``wff stack'', effectively using a
reverse-Polish notation:
\begin{itemize}
\item \verb|v_0()| manipulates the stack $s\mapsto (v_{0},s)$ pushing
  $v_{0}$ ontop of the wff-stack $s$;
\item \verb|v_1()| pushes $v_{1}$ to the top of the stack $s\mapsto(v_{1},s)$
\item \verb|v_2()| pushes $v_{2}$ to the top of the stack $s\mapsto(v_{2},s)$
\item \verb|v_3()| pushes $v_{3}$, $s\mapsto(v_{3},s)$
\item \verb|par1()| pushes a scheme parameter $s\mapsto(P_{1},s)$ ---
  these are the ``parameter'' arguments to the instruction
\item \verb|par2()| pushes a scheme parameter $s\mapsto(P_{2},s)$
\item \verb|par3()| pushes a scheme parameter $s\mapsto(P_{3},s)$
\item \verb|wim()| pops two formulas and pushes an implication $(\psi,\varphi,s)\mapsto(\varphi\implies\psi,s)$
\item \verb|wn()| pops the top of the stack, then pushes the negation
  back onto the stack $(\varphi,s)\mapsto(\neg\varphi,s)$
\item \verb|wal()| manipulates the stack by pushing a universal quantifier $(\varphi,x,s)\mapsto((\forall x)(\varphi),s)$
\item \verb|wex()| pushes an existential quantifier $(\varphi,x,s)\mapsto((\exists x)(\varphi),s)$
\item \verb|wa()| conjoins the top two formulas in the stack
  $(\psi,\varphi,s)\mapsto(\varphi\land\psi,x)$
\item \verb|wel()| $(y,x,s)\mapsto(x\in y,s)$
\item \verb|weq()| $(y,x,s)\mapsto(x=y,s)$
\end{itemize}
\end{node}

\begin{node}[Runtime of Turing machine too long]\label{set:zfc-000H}%
This requires something like $BB(744)$ steps to halt, if
it halts at all, where $BB(n)$ is the Busy Beaver function\footnote{For
a review, see Scott Aaronson's note \url{https://www.scottaaronson.com/papers/bb.pdf}}.
This would require something along the lines of
$S(744)\leq\Sigma(2238)$ steps, where $\Sigma(2k)>3\uparrow^{k-2}3$ is a
lower bound (there are no known upper bounds). That's
$\Sigma(2238)>3\uparrow^{1121}3$ steps. Compare this to known values, $BB(5)=47\,176\,870$
and $BB(6)\geq 7.4\times 10^{36543}$. It only gets worse as we consider
larger values of $BB(k)$. 

Note that the Poincar\'e recurrence theorem says that a mechanical
system would return to its original state in a time interval of
\begin{equation}
\Delta t\sim\exp(S)
\end{equation}
where $S$ is the entropy of the system. Using the Gibbons--Hawking
entropy for de Sitter space for our present universe, this would
give\footnote{The basic argument may be found in Don Page's article
\arXiv{hep-th/9411193}, but I also noticed Ropotenko \arXiv{0712.0993}}
\begin{equation}\label{eq:set:poincare-period-of-universe}
\Delta t_{\text{universe}}\sim\exp(10^{120})\;\mbox{seconds, years,
  millenia, whatever}.
\end{equation}
This is so huge a number, the units don't really matter. Indeed, in
base-10, we'd have $\Delta t\sim 10^{4.343\cdot 10^{119}}$ which might
as well be $10^{10^{120}}$.
\end{node}
\end{node}

\begin{puzzle}[Consistency of ZFC]
Can we improve the runtime for the Turing machine to determine the
consistency of \ZFC\ to be within a lifetime? Or, at least, before the
Poincar\'e recurrence period for the universe in Eq~\eqref{eq:set:poincare-period-of-universe}?
\end{puzzle}
\end{node} % consistency
