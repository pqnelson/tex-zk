% 28
\chapter{Standard ML}\label{chapter:sml}

\begin{node}\label{sml-0000}%
\SML\ reminds me of Scheme in its minimalism. We can ``morally'' use the
formula
\begin{equation*}
\mbox{\SML}=\begin{pmatrix}
\mbox{Typed}\\
\lambda\mbox{-calculus}
\end{pmatrix} + \begin{pmatrix}\mbox{Algebraic}\\
\mbox{datatypes}
\end{pmatrix} + (\mbox{modules}).
\end{equation*}
Towards that end, it makes sense to introduce these three components in turn.
\end{node}%

\begin{node}[Conventions]\label{sml-0001}%
We will adopt some conventions to simplify our
presentation. Specifically, we will not allow redefinitions: once a
value or function has been defined, we will not redefine
it. \SML\ allows the user to redefine functions and use static binding,
but this is too complicated.

We will also not worry about floating-point arithmetic. Although
floating-point is a critical component to numerical analysis, we will be
working with the goal of implementing proof assistants.
\end{node}

\begin{node}\label{sml-0002}%
We will treat \SML\ as ``just another deductive system''. This means we
will present the typing rules (``static semantics'') and evaluation
rules (``dynamic semantics'') as inference rules. Such a presentation
follows how the Definition of \SML~\cite[\S1]{milner1997definition}
describes execution of \SML\ programs:
\begin{quote}
In the execution of a declaration there are three phases:
\emph{parsing}, \emph{elaboration}, and \emph{evaluation}. Parsing
determines the grammatical form of a declaration. Elaboration, the
\emph{static} phase, determines whether it is well-typed and well-formed
in other ways, and records relevant type or form information in the
basis. Finally evaluation, the \emph{dynamic} phase, determines the
value of the declaration and records relevant value information in the
basis. Corresponding to these phases, our formal definition divides into
three parts: grammatical rules, elaboration rules, and evaluation
rules. Furthermore, the basis is divided into the \emph{static} basis
and the \emph{dynamic} basis; for example, a variable which has been
declared is associated with a type in the static basis and with a value
in the dynamic basis.
\end{quote}
\end{node}

\transclude{sml/core}
