% 28
\begin{node}\label{sml-0000}%
\SML\ reminds me of Scheme in its minimalism. We can ``morally'' use the
formula
\begin{equation*}
\mbox{\SML}=\begin{pmatrix}
\mbox{Typed}\\
\lambda\mbox{-calculus}
\end{pmatrix} + \begin{pmatrix}\mbox{Algebraic}\\
\mbox{datatypes}
\end{pmatrix} + (\mbox{modules}).
\end{equation*}
Towards that end, it makes sense to introduce these three components in turn.
\end{node}%

\begin{node}[Conventions]\label{sml-0001}%
We will adopt some conventions to simplify our
presentation. Specifically, we will not allow redefinitions: once a
value or function has been defined, we will not redefine
it. \SML\ allows the user to redefine functions and use static binding,
but this is too complicated.

We will also not worry about floating-point arithmetic. Although
floating-point is a critical component to numerical analysis, we will be
working with the goal of implementing proof assistants.
\end{node}

\begin{node}\label{sml-0002}%
We will treat \SML\ as ``just another deductive system''. This means we
will present the typing rules (``static semantics'') and evaluation
rules (``dynamic semantics'') as inference rules. Such a presentation
follows how the Definition of \SML~\cite[\S1]{milner1997definition}
describes execution of \SML\ programs:
\begin{quote}
In the execution of a declaration there are three phases:
\emph{parsing}, \emph{elaboration}, and \emph{evaluation}. Parsing
determines the grammatical form of a declaration. Elaboration, the
\emph{static} phase, determines whether it is well-typed and well-formed
in other ways, and records relevant type or form information in the
basis. Finally evaluation, the \emph{dynamic} phase, determines the
value of the declaration and records relevant value information in the
basis. Corresponding to these phases, our formal definition divides into
three parts: grammatical rules, elaboration rules, and evaluation
rules. Furthermore, the basis is divided into the \emph{static} basis
and the \emph{dynamic} basis; for example, a variable which has been
declared is associated with a type in the static basis and with a value
in the dynamic basis.
\end{quote}
\end{node}

\section{Judgements and Inference Rules for Core Language}

\begin{node}\label{sml-0003}%
We need to introduce some judgement forms if we wish to view \SML\ as
``just another deductive system''. Then we will introduce the base types
(Booleans, integers, functions, and so on). We will follow the basic
approach found in Erdmann's lecture notes~\cite{erdmann2023evaluation}.
It appears to be inspired by Harper and Stone's earlier work~\cite{harper2000:smltt-final}.
\end{node}

\begin{node}[Judgement Forms]\label{sml-0004}\index{Judgement!Standard ML}\index{Standard ML!Judgement}%
We will stipulate the user will be able to determine if an input string
is a grammatically well-formed \SML\ expression or not. Let us now
consider the sort of judgemental forms required for discussing \SML\ as
a deductive system. The basic judgement forms are sketched out here, but
we probably follow Geuvers and Nederpelt~\cite{nederpelt2014type} in
adding definitions later on.

\begin{node}\label{sml-0005}\index{Judgement!Reduction}%
We will define the ``dynamic semantics'' using a notion of
\define{Reduction}. Let $e$ and $e'$ be expressions. When $k$ is a
positive integer, we write $e\Reduces{k}e'$ for ``expression $e$ reduces
to $e'$ in $k$ steps.'' When we omit the $k$, writing $e\Reduces{}e'$,
we mean ``$e$ reduces to $e'$ in $0$ or more steps''.

Here ``steps'' refers to an abstract operational semantics, which we are
going to define. It may or may not be proportional to the number of CPU
cycles of an actual computer evaluating the \SML\ code.
\end{node}

\begin{node}\label{sml-0006}\index{Judgement!Value}%
We call an expression a \define{Value} if it evaluates to itself.
Let $e$ be an expression. The judgement ``Expression $e$ is a value'' is
denoted ``$e~\Value$''.
\end{node}

\begin{node}\label{sml-0007}\index{Judgement!Evaluation}%
Let $e$ be an expression and $v$ be a value. We introduce the judgement
``$e\evalto v$'' for ``Expression $e$ evaluates to value $v$''. 

This is related to reduction as follows: $e\evalto v$ if and only if either
$e$ is equal to $v$ already or $e\Reduces{1}e_{1}\Reduces{1}\dots\Reduces{1}v$.
\end{node}

\begin{node}\label{sml-000C}%
Let $e$ be an expression, let $T$ be a type. We have the usual typing
declaration ``$e\esti T$'' for the judgement ``Expression $e$ has type $T$''.
The colon here is called \textit{esti}\index{Esti}\index{Type declaration}\index{Judgement!Typing}.
\end{node}

\begin{node}[Raising exceptions]\label{sml-000F}%
Let $exn$ be a so-called ``exception packet'' (a ``value'' of some kind
of exception). When an expression $e$ raises an exception, we indicate
this by the judgement ``$e\Reduces{}\Raise{exn}$''. If we know this is done
in $k\in\NN$ steps, then we may write ``$e\Reduces{k}\Raise{exn}$''.

If, in the process of evaluating some expression $e[e']$ which has $e'$
as a subexpression, we have $e'\Reduces{}\Raise{exn}$, then
the exception propagates out $e[e']\Reduces{}\Raise{exn}$.
\end{node}
\end{node}

\begin{definition}[Extensional equivalence]\label{sml-0008}%
\begin{node}[For non-function expressions]\label{sml-0009}%
Let $e$ and $e'$ are two expressions of the same nonfunction type.
We say $e$ and $e'$ are \define{Extensionally Equivalent} and write
$e\exteq e'$ whenever one of the following is true:
\begin{enumerate}
\item Evaluation of $e$ produces the same value as does the evaluation
  of $e'$; or
\item Evaluation of $e$ raises ``the same'' exception as does the evaluation
  of $e'$; or
\item Evaluation of $e$ and $e'$ both loop forever.
\end{enumerate}
By ``raising `the same' exception'', we mean the exceptions are
extensionally equivalent in the following sense:
We call two exceptions ``extensionally equivalent'' if they are of
the same type and their payloads are extensionally equivalent.
\end{node}

\begin{node}[For function values]\label{sml-000A}%
Let $f$, $g$ be two function values of type $T\to T'$.
We say $f$ and $g$ are \define{Extensionally Equivalent}, written
$f\exteq g$, if for all values $v$ of type $T$, we have $f(v)\exteq g(v)$.
\end{node}

\begin{node}[For function expressions]\label{sml-000B}%
Let $e$, $e'$ be expressions of type $T\to T'$. We say $e$ and $e'$ are
\define{Extensionally Equivalent}, written $e\exteq e'$, if
\begin{enumerate}
\item They both evaluate to extensionally equivalent function values; or
\item Both raise extensionally equivalent exceptions; or
\item Both loop forever when evaluated.
\end{enumerate}
\end{node}
\end{definition}

\begin{node}\label{sml-000D}%
We can now discuss the base types for our fragment of \SML.
\begin{node}[Integers]\label{sml-000E}%
\textbf{Type:} \texttt{int}.

\textbf{Values:} All integer values. If $n\in\ZZ$ is an integer, we will
write $\quasiquote{n}$ for the \SML\ expression which is the integer
literal with value $n$ (so $\quasiquote{7}$ refers to the
\SML\ expression \texttt{7}).

\textbf{Operations:} Let $e_{1}$, $e_{2}$ be \texttt{int}
expressions. Then so are $e_{1}\mbox{\texttt{ + }}e_{2}$,
$e_{1}\mbox{\texttt{ - }}e_{2}$, $e_{1}\mbox{\texttt{ * }}e_{2}$,
$e_{1}\mbox{\texttt{ div }}e_{2}$, $e_{1}\mbox{\texttt{ mod }}e_{2}$,
and possibly a few others.

\textbf{Typing Rules:}
\begin{subequations}
\begin{equation}
\infer[\Rule{T-IntAdd}]{e_{1}\mbox{\texttt{ + }}e_{2}\esti\mathtt{int}}{e_{1}\esti\mathtt{int}
& e_{2}\esti\mathtt{int}}
\end{equation}
\begin{equation}
\infer[\Rule{T-IntSub}]{e_{1}\mbox{\texttt{ - }}e_{2}\esti\mathtt{int}}{e_{1}\esti\mathtt{int}
& e_{2}\esti\mathtt{int}}
\end{equation}
\begin{equation}
\infer[\Rule{T-IntMul}]{e_{1}\mbox{\texttt{ * }}e_{2}\esti\mathtt{int}}{e_{1}\esti\mathtt{int}
& e_{2}\esti\mathtt{int}}
\end{equation}
\begin{equation}
\infer[\Rule{T-IntDiv}]{e_{1}\mbox{\texttt{ div }}e_{2}\esti\mathtt{int}}{e_{1}\esti\mathtt{int}
& e_{2}\esti\mathtt{int}}
\end{equation}
\begin{equation}
\infer[\Rule{T-IntMod}]{e_{1}\mbox{\texttt{ mod }}e_{2}\esti\mathtt{int}}{e_{1}\esti\mathtt{int}
& e_{2}\esti\mathtt{int}}
\end{equation}
\end{subequations}

\textbf{Evaluation:} Like all of \SML, evaluation occurs from left to
right until we obtain values. Formally:
\begin{subequations}
\begin{equation}
\infer[\Rule{E-IntAdd1}]{e_{1}\mbox{\texttt{ + }}e_{2}\Reduces{1}e_{1}'\mbox{\texttt{ + }}e_{2}}{e_{1}\Reduces{1}e_{1}'}
\end{equation}
\begin{equation}
\infer[\Rule{E-IntAdd2}]{n_{1}\mbox{\texttt{ + }}e_{2}\Reduces{1}n_{1}\mbox{\texttt{ + }}e_{2}'}{e_{2}\Reduces{1}e_{2}'
& n_{1}~\Value}
\end{equation}
\begin{equation}
\infer[\Rule{E-IntAddV}]{\quasiquote{n_{1}}\mbox{\texttt{ + }}\quasiquote{n_{2}}\Reduces{1}\quasiquote{n_{1}+n_{2}}}{%
\quasiquote{n_{1}}~\Value &
\quasiquote{n_{2}}~\Value}
\end{equation}
\end{subequations}
\begin{subequations}
\begin{equation}
\infer[\Rule{E-IntSub1}]{e_{1}\mbox{\texttt{ - }}e_{2}\Reduces{1}e_{1}'\mbox{\texttt{ - }}e_{2}}{e_{1}\Reduces{1}e_{1}'}
\end{equation}
\begin{equation}
\infer[\Rule{E-IntSub2}]{n_{1}\mbox{\texttt{ - }}e_{2}\Reduces{1}n_{1}\mbox{\texttt{ - }}e_{2}'}{e_{2}\Reduces{1}e_{2}'
& n_{1}~\Value}
\end{equation}
\begin{equation}
\infer[\Rule{E-IntSubV}]{\quasiquote{n_{1}}\mbox{\texttt{ - }}\quasiquote{n_{2}}\Reduces{1}\quasiquote{n_{1}-n_{2}}}{%
\quasiquote{n_{1}}~\Value &
\quasiquote{n_{2}}~\Value}
\end{equation}
\end{subequations}
\begin{subequations}
\begin{equation}
\infer[\Rule{E-IntMul1}]{e_{1}\mbox{\texttt{ * }}e_{2}\Reduces{1}e_{1}'\mbox{\texttt{ * }}e_{2}}{e_{1}\Reduces{1}e_{1}'}
\end{equation}
\begin{equation}
\infer[\Rule{E-IntMul2}]{n_{1}\mbox{\texttt{ * }}e_{2}\Reduces{1}n_{1}\mbox{\texttt{ * }}e_{2}'}{e_{2}\Reduces{1}e_{2}'
& n_{1}~\Value}
\end{equation}
\begin{equation}
\infer[\Rule{E-IntMulV}]{\quasiquote{n_{1}}\mbox{\texttt{ * }}\quasiquote{n_{2}}\Reduces{1}\quasiquote{n_{1}\cdot n_{2}}}{%
\quasiquote{n_{1}}~\Value &
\quasiquote{n_{2}}~\Value}
\end{equation}
\end{subequations}
\begin{subequations}
\begin{equation}
\infer[\Rule{E-IntDiv1}]{e_{1}\mbox{\texttt{ div }}e_{2}\Reduces{1}e_{1}'\mbox{\texttt{ div }}e_{2}}{e_{1}\Reduces{1}e_{1}'}
\end{equation}
\begin{equation}
\infer[\Rule{E-IntDiv2}]{n_{1}\mbox{\texttt{ div }}e_{2}\Reduces{1}n_{1}\mbox{\texttt{ div }}e_{2}'}{e_{2}\Reduces{1}e_{2}'
& n_{1}~\Value}
\end{equation}
When we take $n$ \texttt{div} $k$ for $k\neq0$, it returns the
greatest integer $m$ such that $m\cdot k\leq n$. For the ``normal''
division situation, we have (with premises stacked vertically for clarity):
\begin{equation}
\infer[\Rule{E-IntDivV}]{\quasiquote{n_{1}}\mbox{\texttt{ div }}\quasiquote{n_{2}}\Reduces{1}\quasiquote{m}}{%
\deduce{\quasiquote{n_{1}}~\Value}{\deduce{%
\quasiquote{n_{2}}~\Value}{%
\deduce{n_{2}\neq0}{%
m\cdot n_{2}\leq n_{1}<(m+1)\cdot n_{2}}}}}
\end{equation}
When we try to divide by zero, an exception is raised\footnote{This is
according to the \SML\ Basis specification \url{https://smlfamily.github.io/Basis/integer.html\#SIG:INTEGER.div:VAL}}.
\begin{equation}
\infer[\Rule{E-IntDivZ}]{\quasiquote{n_{1}}\mbox{\texttt{ div }}\quasiquote{0}\Reduces{1}\Raise{\texttt{Div}}}{%
\quasiquote{n_{1}}~\Value &
\quasiquote{0}~\Value}
\end{equation}
\end{subequations}
\begin{subequations}
\begin{equation}
\infer[\Rule{E-IntMod1}]{e_{1}\mbox{\texttt{ mod }}e_{2}\Reduces{1}e_{1}'\mbox{\texttt{ mod }}e_{2}}{e_{1}\Reduces{1}e_{1}'}
\end{equation}
\begin{equation}
\infer[\Rule{E-IntMod2}]{n_{1}\mbox{\texttt{ mod }}e_{2}\Reduces{1}n_{1}\mbox{\texttt{ mod }}e_{2}'}{e_{2}\Reduces{1}e_{2}'
& n_{1}~\Value}
\end{equation}
We have $(\quasiquote{i}$ \texttt{div} $\quasiquote{j})$ \texttt{*} $\quasiquote{j}$ \texttt{+}
$(\quasiquote{i}$ \texttt{mod} $\quasiquote{j}) = \quasiquote{i}$.
\begin{equation}
\infer[\Rule{E-IntModV}]{\quasiquote{n_{1}}\mbox{\texttt{ mod }}\quasiquote{n_{2}}\Reduces{1}\quasiquote{m}}{%
\deduce{\quasiquote{n_{1}}~\Value}{\deduce{\quasiquote{n_{2}}~\Value}{%
\deduce{n_{2}\neq0}{%
\lfloor n_{1}/n_{2}\rfloor\cdot n_{2} + m = n_{1}}}}}
\end{equation}
Following the specification\footnote{\url{https://smlfamily.github.io/Basis/integer.html\#SIG:INTEGER.mod:VAL}}, when trying to take any value modulo zero
should raise an exception:
\begin{equation}
\infer[\Rule{E-IntModZ}]{\quasiquote{n_{1}}\mbox{\texttt{ mod }}\quasiquote{0}\Reduces{1}\Raise{\texttt{Div}}}{%
\quasiquote{n_{1}}~\Value &
\quasiquote{0}~\Value}
\end{equation}
\end{subequations}
\end{node}

\begin{node}[Booleans]\label{sml-000G}%
\textbf{Type:} \texttt{bool}

\textbf{Values:} \texttt{true} and \texttt{false}

\textbf{Operations:} $e_{1}\mbox{ \texttt{andalso} }e_{2}$,
$e_{1}\mbox{ \texttt{orelse} }e_{2}$,
$\mbox{\texttt{if }}e_{1}\mbox{ \texttt{then} }e_{2}\mbox{ \texttt{else} }e_{3}$,
$e_{1}\mbox{ \texttt{<} }e_{2}$,
$e_{1}\mbox{ \texttt{<=} }e_{2}$,
$e_{1}\mbox{ \texttt{=} }e_{2}$ when $e_{1}$ and $e_{2}$ belong to an ``equality type''.
We will take $e_{1}>e_{2}$ as a synonym for $e_{2}<e_{1}$,
and $e_{1}\geq e_{2}$ as a synonym for $e_{2}\leq e_{1}$.

\textbf{Typing Rules:}
\begin{subequations}
\begin{equation}
\infer[\Rule{T-If}]{\mbox{\texttt{if }}e_{1}\mbox{ \texttt{then} }e_{2}\mbox{ \texttt{else} }e_{3}\esti T}{e_{1}\esti\mathtt{bool} & e_{2}\esti T & e_{3}\esti T}
\end{equation}
\begin{equation}
\infer[\Rule{T-And}]{e_{1}\mbox{ \texttt{andalso} }e_{2}}{e_{1}\esti\mathtt{bool}
& e_{2}\esti\mathtt{bool}}
\end{equation}
\begin{equation}
\infer[\Rule{T-Or}]{e_{1}\mbox{ \texttt{orelse} }e_{2}}{e_{1}\esti\mathtt{bool}
& e_{2}\esti\mathtt{bool}}
\end{equation}
\begin{equation}
\infer[\Rule{T-Lt}]{e_{1}\mbox{ \texttt{<} }e_{2}\esti\mathtt{bool}}{%
  e_{1}\esti T &
  e_{2}\esti T &
  T\in\{\mathtt{string},\mathtt{int},\mathtt{char}\}}
\end{equation}
\begin{equation}
\infer[\Rule{T-Leq}]{e_{1}\mbox{ \texttt{<=} }e_{2}\esti\mathtt{bool}}{%
  e_{1}\esti T &
  e_{2}\esti T &
  T\in\{\mathtt{string},\mathtt{int},\mathtt{char}\}}
\end{equation}
\end{subequations}

\textbf{Evaluation:}
\begin{subequations}
\begin{equation}
\infer[\Rule{E-If1}]{{\mbox{\texttt{if }}e_{1}\mbox{ \texttt{then} }e_{2}\mbox{ \texttt{else} }e_{3}}\Reduces{1}{\mbox{\texttt{if }}e_{1}'\mbox{ \texttt{then} }e_{2}\mbox{ \texttt{else} }e_{3}}}{e_{1}\Reduces{1}e_{1}'}
\end{equation}
\begin{equation}
\infer[\Rule{E-IfTrue}]{\mbox{\texttt{if }}\mathtt{true}\mbox{ \texttt{then} }e_{2}\mbox{ \texttt{else} }e_{3}\Reduces{1}e_{2}}{}
\end{equation}
\begin{equation}
\infer[\Rule{E-IfFalse}]{\mbox{\texttt{if }}\mathtt{false}\mbox{ \texttt{then} }e_{2}\mbox{ \texttt{else} }e_{3}\Reduces{1}e_{3}}{}
\end{equation}
\end{subequations}
\begin{subequations}
\begin{equation}
\infer[\Rule{E-And1}]{e_{1}\mbox{ \texttt{andalso} }e_{2}\Reduces{1}e_{1}'\mbox{ \texttt{andalso} }e_{2}}{e_{1}\Reduces{1}e_{1}'}
\end{equation}
\begin{equation}
\infer[\Rule{E-AndT}]{\mathtt{true}\mbox{ \texttt{andalso} }e_{2}\Reduces{1}e_{2}}{}
\end{equation}
\begin{equation}
\infer[\Rule{E-AndF}]{\mathtt{false}\mbox{ \texttt{andalso} }e_{2}\Reduces{1}\mathtt{false}}{}
\end{equation}
\end{subequations}
\begin{subequations}
\begin{equation}
\infer[\Rule{E-Or1}]{e_{1}\mbox{ \texttt{orelse} }e_{2}\Reduces{1}e_{1}'\mbox{ \texttt{orelse} }e_{2}}{e_{1}\Reduces{1}e_{1}'}
\end{equation}
\begin{equation}
\infer[\Rule{E-OrT}]{\mathtt{true}\mbox{ \texttt{orelse} }e_{2}\Reduces{1}\mathtt{true}}{}
\end{equation}
\begin{equation}
\infer[\Rule{E-OrF}]{\mathtt{false}\mbox{ \texttt{orelse} }e_{2}\Reduces{1}e_{2}}{}
\end{equation}
\end{subequations}
\begin{subequations}
\begin{equation}
\infer[\Rule{E-Leq1}]{e_{1}\mbox{ \texttt{<=} }e_{2}\Reduces{1}e_{1}'\mbox{ \texttt{<=} }e_{2}}{%
e_{1}\Reduces{1}e_{1}'}
\end{equation}
\begin{equation}
\infer[\Rule{E-Leq2}]{n_{1}\mbox{ \texttt{<=} }e_{2}\Reduces{1}n_{1}\mbox{ \texttt{<=} }e_{2}'}{%
e_{2}\Reduces{1}e_{2}' & n_{1}~\Value}
\end{equation}
\begin{equation}
\infer[\Rule{E-LeqT}]{\quasiquote{n_{1}}\mbox{ \texttt{<=} }\quasiquote{n_{2}}\Reduces{1}\mathtt{true}}{n_{1}\leq n_{2}}
\end{equation}
\begin{equation}
\infer[\Rule{E-LeqF}]{\quasiquote{n_{1}}\mbox{ \texttt{<=} }\quasiquote{n_{2}}\Reduces{1}\mathtt{false}}{n_{1}>n_{2}}
\end{equation}
\end{subequations}
\begin{subequations}
\begin{equation}
\infer[\Rule{E-Lt1}]{e_{1}\mbox{ \texttt{<} }e_{2}\Reduces{1}e_{1}'\mbox{ \texttt{<} }e_{2}}{%
e_{1}\Reduces{1}e_{1}'}
\end{equation}
\begin{equation}
\infer[\Rule{E-Lt2}]{n_{1}\mbox{ \texttt{<} }e_{2}\Reduces{1}n_{1}\mbox{ \texttt{<} }e_{2}'}{%
e_{2}\Reduces{1}e_{2}' & n_{1}~\Value}
\end{equation}
\begin{equation}
\infer[\Rule{E-LtT}]{\quasiquote{n_{1}}\mbox{ \texttt{<} }\quasiquote{n_{2}}\Reduces{1}\mathtt{true}}{n_{1}<n_{2}}
\end{equation}
\begin{equation}
\infer[\Rule{E-LtF}]{\quasiquote{n_{1}}\mbox{ \texttt{<} }\quasiquote{n_{2}}\Reduces{1}\mathtt{false}}{n_{1}\geq n_{2}}
\end{equation}
\end{subequations}
\end{node}

\begin{node}[Products]\label{sml-000H}%
\textbf{Type:} $T_{1}\mbox{ \texttt{*} }T_{2}$ for any types $T_{1}$ and $T_{2}$.
More generally, for any $n\geq 2$ and types $T_{1}$, \dots, $T_{n}$, we
form the product type $T_{1}\mbox{ \texttt{*} }\dots\mbox{ \texttt{*} }T_{n}$.

\textbf{Values:} For values $v_{1}\esti T_{1}$, \dots, $v_{n}\esti T_{n}$
we have the tuple $(v_{1},\dots,v_{n})$ be a value of type 
$T_{1}\mbox{ \texttt{*} }\dots\mbox{ \texttt{*} }T_{n}$.

\textbf{Operations:} We can have projections for product types, but
instead we use patterns to destructure tuples.

\textbf{Typing Rules:}
\begin{equation}
\infer[\Rule{T-Tuple}]{(t_{1},\dots,t_{n})\esti T_{1}\mbox{ \texttt{*} }\dots\mbox{ \texttt{*} }T_{n}}{%
t_{1}\esti T_{1},&\dots,&t_{n}\esti T_{n}
}
\end{equation}

\textbf{Evaluation:} We evaluate components of a tuple from left to
right, until all the components are values.
\begin{equation}
\infer[\Rule{E-Tuple}_{k}]{(v_{1},\dots,v_{k-1},t_{k},\dots,t_{n})\Reduces{1}(v_{1},\dots,v_{k-1},t_{k}',\dots,t_{n})}{%
v_{1}~\Value &\dots&v_{k-1}~\Value&t_{k}\Reduces{1}t_{k}'}
\end{equation}

\begin{node}[Remark]\label{sml-000J}%
Note that the record type is ``more primitive'' than the product type,
since $T_{1}\mbox{ \texttt{*} }\dots\mbox{ \texttt{*} }T_{n}$ is then
translated implicitly to
$\{\quasiquote{1}\esti T_{1},\dots,\quasiquote{n}\esti T_{n}\}$.
Tuples $(t_{1},\dots,t_{n})$ is translated to records
$\{\quasiquote{1}=t_{1},\dots,\quasiquote{n}=t_{n}\}$.
This is stated in Appendix A of the definition~\cite{milner1997definition}.
\end{node}
\end{node}

\begin{node}[Functions]\label{sml-000I}%
\textbf{Type:} $T_{1}\mbox{ \texttt{->} }T_{2}$ where $T_{1}$ and
$T_{2}$ are types. Note that $\texttt{->}$ is right associative, so
$T_{1}\texttt{ -> }T_{2}\texttt{ -> }T_{3}$ is parsed as 
$T_{1}\texttt{ -> }(T_{2}\texttt{ -> }T_{3})$.

\textbf{Values:} Function closures (see \S6.6 of the definition~\cite{milner1997definition}),
which is a tuple $(\mathtt{fn}~pat~\texttt{=>}~e_{b}, E)$ where $pat$ is
a pattern, $E$ is an environment.

\textbf{Operations:} The only operation is applying an argument to a
function, written $f~t$ for applying $t$ to the function $f$.

\textbf{Typing Rules:}
\begin{subequations}
\begin{equation}
\infer[\Rule{T-Abs}]{\mathtt{fn}~x~\texttt{=>}~e\esti T_{1}\to T_{2}}{x\esti T_{1}\vdash e\esti T_{2}}
\end{equation}
\begin{equation}
\infer[\Rule{T-App}]{e_{1,2}~e_{1}\esti T_{2}}{e_{1,2}\esti T_{1}\to T_{2}
& e_{1}\esti T_{1}}
\end{equation}
\end{subequations}

\textbf{Evaluation:} We evaluate the function expression before
evaluating the argument.
\begin{subequations}
\begin{equation}
\infer[\Rule{E-App1}]{e_{1}~e_{2}\Reduces{1}e_{1}'~e_{2}}{e_{1}\Reduces{1}e_{1}'}
\end{equation}
\begin{equation}
\infer[\Rule{E-App2}]{v_{1}~e_{2}\Reduces{1}v_{1}~e_{2}'}{v_{1}~\Value &
  e_{2}\Reduces{1}e_{2}'}
\end{equation}
Now, since we are using the conventions that we never redefine values or
functions, the bindings are determined once. Then we can define beta
reduction with the bindings $\Delta$ explicit in the judgement:
\begin{equation*}
\infer%[\Rule{E-Beta}]
{\Delta\vdash (\mathtt{fn}~x~\texttt{=>}~e_{b})~v_{1}\Reduces{1}\Delta,x=v_{1}\vdash e_{b}}{}
\end{equation*}
This is made more precise by rules (109--113) of the Definition~\cite{milner1997definition},
which is mildly complicated due to pattern matching.
\end{subequations}
\begin{node}[Pattern matching]\label{sml-000M}%
\SML\ functions can be given in multiple clauses depending on the value
of the argument given. This is known as pattern matching. If the value
\define{matches} the pattern, then that corresponding clause is executed.

For now, there are several possible patterns since we have only
discussed integers, Booleans, tuples, and had an aside on records. Oh,
and functions, we discussed functions.

The simplest pattern is a wildecard, denoted by an underscore ``\verb|_|''.
This is always matched by any argument.

The next simplest pattern is a variable. If matched, it behaves as we
expected in our simplified $\beta$-reduction rule: we temporarily append
to our definitions the variable bound to the value of the function argument.

We could also have ``special constants'' (literals for strings,
characters, integers, Booleans). The argument matches special constant
patterns if they are equal as values.

For tuples, we can form a tuple of patterns and perform pattern matching
component-wise. If all components of the argument passed into the
function match all components of the tuple pattern, then we extend the
definitions temporarily (if necessary) and execute the associated
clause.

We can also annotate patterns with types to clarify what type is
expected, which is done with the esti operator ``$p\esti T$'' for
pattern $p$ expected to be type $T$.

We can also ``preserve'' the argument passed in as a variable, writing
``$x$ \texttt{as} $p$'' which will then match the argument passed in
against the pattern $p$ and, if there is a match, stores the argument as
$x$ in the dictionary of bindings.

As we introduce more complicated types (like lists), we will mention
their pattern matching (if relevant).
\end{node}

\begin{node}[Remark]\label{sml-000K}%
Case expressions are syntactic sugar for functions. The \SML\ compiler
translates $\mathtt{case}~exp~\mathtt{of}~match$ to $(\mathtt{fn}~match)~exp$
according to Appendix~A of the Definition~\cite{milner1997definition}.
\end{node}
\end{node}

\begin{node}[Lists]\label{sml-000L}%
\textbf{Type:} ``\texttt{T list}'' for any type \texttt{T}

\textbf{Values:} $\texttt{[}v_{1},\dots,v_{n}\texttt{]}$ where
$v_{j}$ is a value of type $T$ and $n\geq0$. We also have \texttt{[]} be
a value called ``nil'' or ``the empty list''.

\textbf{Expressions:} The collection of expressions for this type
includes all values of type \texttt{T list}, and it also includes all
expressions of the form $e\texttt{::}\ell$ with $e\esti T$ and
$\ell\esti\texttt{T list}$. Note that $\texttt{[}e_{1},\dots,e_{n}\texttt{]}$
is syntactic sugar for $e_{1}\texttt{::}\dots\texttt{::}e_{n}\texttt{::}\texttt{[]}$.

The infixed operator ``\texttt{::}'' is right associative. We pronounce
it as ``cons''.

\textbf{Typing Rules:}
\begin{subequations}
\begin{equation}
\infer[\Rule{T-Nil}]{\texttt{[]}\esti T~\texttt{list}}{}
\end{equation}
\begin{equation}
\infer[\Rule{T-Cons}]{t\texttt{::}\ell\esti T~\texttt{list}}{t\esti T & \ell\esti T~\texttt{list}}
\end{equation}
\end{subequations}

\textbf{Evaluation:}
\begin{subequations}
\begin{equation}
\infer[\Rule{E-Nil}]{\texttt{[]}~\Value}{}
\end{equation}
\begin{equation}
\infer[\Rule{E-Cons1}]{e_{1}\texttt{::}e_{2}\Reduces{1}e_{1}'\texttt{::}e_{2}}{e_{1}\Reduces{1}e_{1}'}
\end{equation}
\begin{equation}
\infer[\Rule{E-Cons2}]{v_{1}\texttt{::}e_{2}\Reduces{1}v_{1}\texttt{::}e_{2}}{%
  v_{1}~\Value & e_{2}\Reduces{1}e_{2}'}
\end{equation}
\end{subequations}

\textbf{Patterns:} There are two patterns which will match a list
\begin{enumerate}
\item \texttt{[]} matches the empty list, and
\item $p_{h}\texttt{::}p_{\ell}$ will try to match
  pattern $p_{h}$ against the head of the list and $p_{\ell}$ against
  the tail of the list.
\end{enumerate}

\begin{definition}[{Johnstone~\cite[A2.5.15]{johnstone2002sketches:v1}}]\label{sml-000N}%
Let $\cat{C}$ be a category with finite products and a terminal object
$\mathbf{1}$. Let $A\in\cat{C}$ be an object. Then a \define{List Object} over
$A$ is an object $L_{A}\in\cat{C}$ equipped with
\begin{enumerate}
\item a morphism $o_{A}\colon\mathbf{1}\to L_{A}$ called ``nil'', and
\item a morphism $s_{A}\colon A\times L_{A}\to L_{A}$ called ``cons''
\end{enumerate}
such that
\begin{enumerate}
\item Universal Property of Lists: for any object $B$ of $\cat{C}$ and morphisms
  $b\colon\mathbf{1}\to B$ and $t\colon A\times B\to B$, there exists a
  unique morphism $f\colon L_{A}\to B$ 
  such that the following diagram commutes:
\begin{equation}\def\comma{,}
  \begin{tikzcd}[sep=large]
\mathbf{1}
   \arrow[r, "o_{A}"]
   \arrow[dr, swap, "b"]
   & L_{A} \arrow[d, dashed, "f"]
   & A\times L_{A} \arrow[l, swap, "s_{A}"]\arrow[d, dashed, "\langle \id_{A}\comma f\rangle"] \\
 {}
 & B
 & A\times B \arrow[l, "t"]
\end{tikzcd}
\end{equation}
\end{enumerate}
The universal property says $f(\texttt{[]})=b()$ and $(t\circ\langle\id_{A}, f\rangle)(a\texttt{::}\ell)=t(a,f(\ell))=f(a\texttt{::}\ell)$.
In \SML\ this says that $f$ is just \verb|foldr t b()| and therefore
lists in \SML\ satisfy the universal property of lists.
\end{definition}
\end{node}

\begin{node}[Characters]\label{sml-000O}%
\textbf{Type:} \texttt{char}

\textbf{Values:} The \SML\ Definition~\cite[\S2.2]{milner1997definition}
allows for an ordered alphabet with $N\geq 127$ letters provided the
first 127 letters coincide with ASCII. The literal values look like
\Char{a}, \Char{b}, etc. There are escape sequences
\begin{itemize}
\item \Char{\texttt{\textbackslash a}} (for ``alert'', ASCII 7)
\item \Char{\texttt{\textbackslash b}} (for ``alert'', ASCII 8)
\item \Char{\texttt{\textbackslash t}} (for ``tab'', ASCII 9)
\item \Char{\texttt{\textbackslash n}} (for ``linefeed'' or ``new line'', ASCII 10)
\item \Char{\texttt{\textbackslash v}} (for ``vertical tab'', ASCII 11)
\item \Char{\texttt{\textbackslash f}} (for ``form feed'', ASCII 12)
\item \Char{\texttt{\textbackslash r}} (for ``carriage return'', ASCII 13)
\item \Char{\texttt{\textbackslash"}} for a single quote
\item \Char{\texttt{\textbackslash\textbackslash}} for a backslash (``\textbackslash'')
\item \Char{\textbackslash\texttt{u}$xxxx$} where $x$ is a hexadecimal
  digit refers to the character with codepoint given by $xxxx$
\item \Char{\textbackslash$ddd$} where $d$ is a digit, refers to the
  character whose codepoint is given by $ddd$ provided $0\leq ddd\leq255$.
\end{itemize}
Note that ASCII considers a character ``printable'' if its code point
value lies between numbers 33-126, inclusive.

\textbf{Operations:} This is an equality type, so we may compare two
characters $c_{1}\texttt{ = }c_{2}$ and $c_{1}\texttt{<>} c_{2}$ (for
$c_{1}\neq c_{2}$). We may also obtain the ``codepoint'' for the
character using \texttt{ord}~$c\esti\texttt{int}$. This allows us to
write comparison operations for characters, e.g., $c_{1}\texttt{ < }c_{2}$
iff $\texttt{ord}~c_{1}\texttt{ < }\texttt{ord}~c_{2}$.

We can turn an integer into a character using $\texttt{chr}~n\esti\texttt{char}$,
which raises the \texttt{Chr} exception for negative or overly large arguments.


\textbf{Typing Rules:}
\begin{subequations}
\begin{equation}
\infer[\Rule{T-CharOrd}]{\texttt{ord}~c\esti\texttt{int}}{c\esti\texttt{char}}
\end{equation}
\begin{equation}
\infer[\Rule{T-CharChr}]{\texttt{chr}~n\esti\texttt{char}}{n\esti\texttt{int}}
\end{equation}
\end{subequations}

\textbf{Evaluation:} If we denote by $N$ the maximum possible code point
value for characters for our encoding, then we have the following semantics:
\begin{subequations}
\begin{equation}
\infer[\Rule{E-CharOrd}]{\texttt{ord}~c~\Value}{c~\Value}
\end{equation}
\begin{equation}
\infer[\Rule{E-CharChr}]{\texttt{chr}~n\Reduces{1}c}{\quasiquote{n}~\Value & c~\Value &
  0\leq n\leq N & \texttt{ord}~c=\quasiquote{n}}
\end{equation}
\begin{equation}
\infer[\Rule{E-CharLt1}]{c_{1}\texttt{ < }c_{2}\Reduces{1}c_{1}'\texttt{ < }c_{2}}{c_{1}\Reduces{1}c_{1}'}
\end{equation}
\begin{equation}
\infer[\Rule{E-CharLt2}]{v_{1}\texttt{ < }c_{2}\Reduces{1}v_{1}\texttt{ < }c_{2}'}{c_{2}\Reduces{1}c_{2}' & v_{1}~\Value}
\end{equation}
\begin{equation}
\infer[\Rule{E-CharLtV}]{v_{1}\texttt{ < }v_{2}\Reduces{1}\texttt{ord}~v_{1}\texttt{ < }\texttt{ord}~v_{2}}{v_{2}~\Value & v_{1}~\Value}
\end{equation}
\begin{equation}
\infer[\Rule{E-CharLeq1}]{c_{1}\texttt{ <= }c_{2}\Reduces{1}c_{1}'\texttt{ <= }c_{2}}{c_{1}\Reduces{1}c_{1}'}
\end{equation}
\begin{equation}
\infer[\Rule{E-CharLeq2}]{v_{1}\texttt{ <= }c_{2}\Reduces{1}v_{1}\texttt{ <= }c_{2}'}{c_{2}\Reduces{1}c_{2}' & v_{1}~\Value}
\end{equation}
\begin{equation}
\infer[\Rule{E-CharLeqV}]{v_{1}\texttt{ <= }v_{2}\Reduces{1}\texttt{ord}~v_{1}\texttt{ <= }\texttt{ord}~v_{2}}{v_{2}~\Value & v_{1}~\Value}
\end{equation}
\end{subequations}
\end{node}

\begin{node}[Strings]\label{sml-000P}%
\textbf{Type:} \texttt{string} (which is isomorphic to the type
\texttt{char list})
  
\textbf{Values:} Literal values for strings are given by a sequence,
between quotes (\texttt{"}), of zero or more printable characters (i.e.,
numbered 33--126), spaces, or escape characters.

\textbf{Operations:} A \texttt{string} is basically the same as a list
of characters. We can transform a string instance into a list of its
characters using $\texttt{explode}~s\esti\texttt{char list}$, and
convert it back using $\texttt{implose}~\ell\esti\texttt{string}$.
The primitive operations can be defined using these functions alone.

\textsc{But nice-to-have operations:}
Given two strings $s_{1}$ and $s_{2}$, we can append one after the other
using $s_{1}\appendstr s_{2}\esti\texttt{string}$ (Verbatim: \verb#s1 ^ S2#). If we have a list of
strings $\ell\esti\texttt{string list}$, then we could iteratively
concatenate them all together using $\texttt{concat}~\ell\esti\texttt{string}$.

The length of a string (i.e., the number of characters it contains) is
given by $\texttt{size}~s\esti\texttt{int}$.

We can transform any character $c\esti\texttt{char}$ into a single
letter string using the primitive function $\texttt{str}~c\esti\texttt{string}$.

This is an equality type, so for any two strings $s_{1}$ and $s_{2}$ we
may obtain the Boolean expression $s_{1}\texttt{ = }s_{2}$ testing for
equality, and $s_{1}\texttt{ <> }s_{2}$ for the negation of equality. We
can also use lexicographical ordering to have $s_{1}\texttt{ < }s_{2}$
as well as $s_{1}\texttt{ <= }s_{2}$.

\textbf{Typing Rules:}
\begin{subequations}
\begin{equation}
\infer[\Rule{T-StrExp}]{\texttt{explode}~s\esti\texttt{char list}}{s\esti\texttt{string}}
\end{equation}
\begin{equation}
\infer[\Rule{T-StrImp}]{\texttt{implode}~\ell\esti\texttt{string}}{\ell\esti\texttt{char list}}
\end{equation}
\end{subequations}

\textbf{Evaluation:}
\begin{subequations}
\begin{equation}
\infer[\Rule{E-StrExp1}]{\texttt{explode}~s\Reduces{1}\texttt{explode}~s'}{s\Reduces{1}s'}
\end{equation}
If $\ell$ is the list of character literals appearing in $s$, then:
\begin{equation}
\infer[\Rule{E-StrExp2}]{\texttt{explode}~s\Reduces{1}\ell}{s~\Value & \ell~\Value}
\end{equation}
\begin{equation}
\infer[\Rule{E-StrImp1}]{\texttt{implode}~\ell\Reduces{1}\texttt{implode}~\ell'}{\ell\Reduces{1}\ell'}
\end{equation}
If $s$ is the string consisting of the character literals as they appear
in the value $\ell$, then:
\begin{equation}
\infer[\Rule{E-StrImp2}]{\texttt{implode}~\ell\Reduces{1}s}{\ell~\Value & s~\Value}
\end{equation}
\end{subequations}
As I said, a string is isomorphic to a list of characters, and we see in
fact for all string values $s$ we have:
\begin{equation}
\texttt{implode}(\texttt{explode}~s)=s,
\end{equation}
and for all list of characters values $\ell$, we have
\begin{equation}
\texttt{explode}(\texttt{implode}~\ell)=\ell.
\end{equation}
This gives us our isomorphism claim.
\end{node}
\end{node}

\begin{node}\label{sml-000Q}%
The evaluation of a \SML\ program amounts to checking a declaration is
well-typed, then evaluating it, and adding it to the list of definitions
\begin{subequations}
\begin{equation}
\infer[\Rule{E-Val}]{\Delta\vdash (\mathtt{val}~id=e);prog\Reduces{k}\Delta,id=v\vdash prog}{\Delta\vdash e\Reduces{k}v
& \Delta\vdash e\esti T & \Delta\vdash v~\Value}
\end{equation}
Functions require a bit more care because we allow recursion. In this
case, we temporarily add a binding for the identifier of the function
when evaluating it:
\begin{equation}
\infer[\Rule{E-Fun}]{\Delta\vdash (\mathtt{fun}~id~p);prog\Reduces{k}\Delta,id=v\vdash prog}{\Delta,id=\mathtt{fn}~p\vdash (\mathtt{fn}~p)\Reduces{k}v
& \Delta\vdash (\mathtt{fn}~p)\esti T\to T' & \Delta\vdash v~\Value}
\end{equation}
\end{subequations}
These correspond to the rules found in the \SML\ definition for
evaluating top-level programs.

Care must be taken because, despite all this, each implementation of
\SML\ has its own way of determining the ``main function'' to be
executed as the program. MLton will allow you to run code wherever you
wish, but Poly/ML requires an entry point --- specifically a function
named \verb|main()|. Chris Cannam observed this difficulty in his blogpost\footnote{\url{https://thebreakfastpost.com/2015/06/10/standard-ml-and-how-im-compiling-it/}}.
\end{node}

