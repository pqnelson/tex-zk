% 24
\begin{node}[Grothendieck invented internalization]\label{bourbaki-000A}%
Note that internalization seems to be traced back to Grothendieck's
comments in \textit{FGA} \S4.2\footnote{English translation \url{https://thosgood.com/fga/FGA-3-II.html\#fga-3-ii-section-A.1}}
and groups with operators given in \S5.3\footnote{English translation \url{https://thosgood.com/fga/FGA-3-III.html\#fga-3-iii-section-3}}.
\end{node}

\begin{node}[Group object]\label{internal-0007}%
\begin{definition}
Let $\cat{C}$ be a category with finite products and a terminal object
$\mathbf{1}$. We define a \define{Monoid object} in $\cat{C}$ to consist
of an object $M$ equipped with morphisms $m\colon M\times M\to M$
and $e\colon\mathbf{1}\to M$ such that
\begin{enumerate}
\item $m$ is associative, i.e., the following diagram commutes
\begin{equation*}
\begin{tikzcd}[sep=large]
M\times M\times M\ar[d,swap,"m\times\id_{M}"]\ar[r,"\id_{M}\times m"] & M\times M\ar[d,"m"]\\
M\times M\ar[r,swap,"m"] & M
\end{tikzcd}
\end{equation*}
\item the unit laws hold (i.e., the identity element acts as expected),
  i.e., the following diagram commutes
\begin{equation*}
\begin{tikzcd}[sep=large]
\mathbf{1}\times M\ar[dr,swap,"\pi_{2}"]\ar[r,"e\times\id_{M}"] &M\times M\ar[d,"m"] & \ar[l,swap,"\id_{M}\times e"]\ar[dl,"\pi_{1}"]M\\
& M &
\end{tikzcd}
\end{equation*}
\end{enumerate}
\end{definition}

\begin{node}[Examples]\label{internal-000J}%
\begin{node}\label{internal-000K}%
Let $\FF$ be a field. A monoid object in $\Vect{\FF}$ (considered as a
monoidal category with tensor product of vector spaces) is precisely an
associative unital algebra over $\FF$.
\end{node}

\begin{node}\label{internal-000L}%
A monoid object in $\Set$ is just the usual notion of a monoid we learn
in abstract algebra.
\end{node}

\begin{node}[Joke]\label{internal-000M}%
A monoid in the category of endofunctors is a monad.
\end{node}
\end{node}% examples

\begin{definition}\label{internal-0009}%
Let $\cat{C}$ be a category with finite products and a terminal object
$\mathbf{1}$. A \define{Group object} in $\cat{C}$ consists of a monoid
object $(G,m,e)$ equipped with an ``inverse'' morphism $i\colon G\to G$
such that the following diagram commutes:
\begin{equation*}
\begin{tikzcd}[sep=large]
G\ar[d,swap,"!"]\ar[r,"\Delta"] & G\times G\ar[r,"i\times\id_{G}"] & G\times G\ar[d,"m"]\\
\mathbf{1}\ar[rr,swap,"e"] & & G
\end{tikzcd}
\end{equation*}
where $\Delta$ is the diagonal morphism defined by
$\pi_{1}\circ\Delta=\pi_{2}\circ\Delta$.
\end{definition}

\begin{node}[Examples]\label{internal-000A}%

\begin{node}\label{internal-000B}%
In $\Set$, a group object is simply the usual notion of a group we all
know and love.
\end{node}

\begin{node}\label{internal-000C}%
In $\Top$, a group object is precisely a topological group.
\end{node}

\begin{node}\label{internal-000D}%
In $\Mfld$, a group object is a Lie group.
\end{node}

\begin{node}\label{internal-000E}%
In $\Grp$, a group object looks like\dots well, what? We would have $G$
be a group, its law of composition $m\colon G\times G\to G$ be a group
morphism. Could it be different than the usual law of composition? 

We would have $e\colon\mathbf{1}\to G$ pick out the usual identity
element since $\mathbf{1}$ is the trivial group, and we have no choice
in the matter. This means that the identity element of $G$ is precisely
$e(1)$ the identity for the law of composition $m$.

Now, for any $a,b\in G$, we would have $m(a,b)=ab=m(b,a)=ba$, since
\begin{subequations}
\begin{align}
m(a,b) &= m(1\cdot a,b\cdot 1)\\
&=m(1,b)\cdot m(a,1)\quad\mbox{by group morphism}\\
&=m(b,1)\cdot m(1,a)\quad\mbox{by unit laws}\\
&=b\cdot a\\
&=m(b,1)\cdot m(1,a)\\
&=m(b\cdot 1,1\cdot a)\\
&=m(b,a),
\end{align}
\end{subequations}
and similar reasoning gives us the rest of the claim.

Therefore, a group object in $\Grp$ is an Abelian group.
\end{node}

\begin{node}\label{internal-000F}%
A group object in the category of supermanifolds $\sMfld$ is precisely a
super Lie group.
\end{node}

\begin{node}\label{internal-000G}%
If we consider the category of algebraic varieties, a group object is
simply an algebraic group. If we consider the category of schemes, then
a group object is a group scheme.
\end{node}
\end{node} % examples

\begin{node}[References]\label{internal-0008}%
See Mac Lane~\cite[III\S6]{mac1998categories} and
the note of Forrester-Barker~\cite{forresterbarker2002group}.
\end{node}
\end{node}

\begin{node}[Internal ring object]\label{internal-000H}%
\begin{definition}
Let $\cat{C}$ be a Cartesian monoidal category. A \define{Ring object}
in $\cat{C}$ consists of an object $R$ equipped with
\begin{enumerate}
\item addition $a\colon R\times R\to R$ and multiplication $m\colon R\times R\to R$
\item a zero element $z\colon\mathbf{1}\to R$
\item a unit element $e\colon\mathbf{1}\to R$
\item additive inversion $n\colon R\to R$
\end{enumerate}
such that the usual axioms apply as encoded by commutative diagrams ---
so $(R,a,z,n)$ is a group object and $(R,m,e)$ is a monoid object, and
distributivity holds.
\end{definition}

\begin{definition}
Let $(\cat{C},\otimes,\mathbf{1})$ be a monoidal category. Let
$(A,\mu,e)$ be a monoid in $(\cat{C},\otimes,\mathbf{1})$.
We define a \define{Left module object} in $\cat{C}$ over $A$ is an
object $M\in\cat{C}$ equipped with a morphism $\rho\colon A\otimes M\to M$
(called the action, or ``scalar multiplication'') such that
\begin{enumerate}
\item unitality: the following diagram commutes (where $\ell$ is the
  left unitor isomorphism of $\cat{C}$)
\begin{equation*}
\begin{tikzcd}
\mathbf{1}\otimes M\arrow[dr,swap,"\ell"]\ar[r,"e\otimes\id"] & A\otimes M\ar[d,"\rho"]\\
& M
\end{tikzcd}
\end{equation*}
\item Action property: the following diagram commutes (where $a_{A,A,M}$
  is the natural isomorphism called the ``actor'')
\begin{equation*}
\begin{tikzcd}[sep=large]
(A\otimes A)\otimes M \ar[r,"a_{A,A,M}","\sim"']\ar[d,"\mu\otimes M"] &A\otimes(A\otimes M)\ar[r,"A\otimes\rho"] & A\otimes M\ar[d,"\rho"]\\
A\otimes M \ar[rr,"\rho"] & & M
\end{tikzcd}
\end{equation*}
\end{enumerate}
\end{definition}

\begin{node}[References]\label{internal-000I}%
See Mac Lane~\cite[VII\S4]{mac1998categories} for left-module objects
(usually called ``left action objects'' in the category theory literature).
\end{node}
\end{node}

\begin{node}[Internal Category Theory]\label{internal-0004}%
\begin{definition}\label{internal-0000}%
Let $\cat{C}$ be a category. We define a \define{Category Object}
$X$ internal to $\cat{C}$ to consist of
\begin{enumerate}
\item an \textsc{object of objects} $X_{0}\in\cat{C}$
\item an \textsc{object of morphisms} $X_{1}\in\cat{C}$
\end{enumerate}
equipped with
\begin{enumerate}
\item source and target morphisms $s,t\colon X_{1}\to X_{0}$;
\item an identity-assignment $e\colon X_{0}\to X_{1}$;
\item a composition operator $c\colon X_{1}\times_{X_{0}}X_{1}\to X_{1}$;
\end{enumerate}
such that the usual laws of categories hold:
\begin{enumerate}
\item laws specifying the target and source of the identity morphisms
\begin{equation*}
\begin{tikzcd}[sep=large]
X_{0}\arrow[dr,swap,"\id"]\arrow[r,"e"] & X_{1}\arrow[d,"s"]\\
                                       & X_{0}
\end{tikzcd}\qquad
\begin{tikzcd}[sep=large]
X_{0}\arrow[dr,swap,"\id"]\arrow[r,"e"] & X_{1}\arrow[d,"t"]\\
                                       & X_{0}
\end{tikzcd}
\end{equation*}
\item laws specifying the source and target of composite morphisms
\begin{equation*}
\begin{tikzcd}[sep=large]
X_{1}\times_{X_{0}}X_{1}\ar[d,"p_{1}"] \ar[r,"c"] & X_{1}\ar[d,"s"]\\
    X_{1} \ar[r,"s"] & X_{0}
\end{tikzcd}\qquad
\begin{tikzcd}[sep=large]
X_{1}\times_{X_{0}}X_{1}\ar[d,"p_{2}"] \ar[r,"c"] & X_{1}\ar[d,"t"]\\
    X_{1} \ar[r,"t"] & X_{0}
\end{tikzcd}
\end{equation*}
\item associativity for composition
\begin{equation*}
\begin{tikzcd}[sep=large]
X_{1}\times_{X_{0}}X_{1}\times_{X_{0}}X_{1}\ar[r,"c\times_{X_{0}}\id"]\ar[d,"\id\times_{X_{0}}c"]&X_{1}\times_{X_{0}}X_{1}\ar[d,"c"]\\
  X_{1}\times_{X_{0}}X_{1}\ar[r,"c"] & X_{1}
\end{tikzcd}
\end{equation*}
\item the left and right unit laws for composition of morphisms:
\begin{equation*}
\begin{tikzcd}[sep=large]
X_{0}\times_{X_{0}} X_{1}\ar[r,"e\times\id"]\ar[dr,swap,"p_{2}"] & X_{1}\times_{X_{0}} X_{1}\ar[d,"c"] & \ar[l,swap,"\id\times e"] X_{1}\times_{X_{0}}X_{0}\ar[dl,"p_{1}"]\\
& X_{1} & 
\end{tikzcd}
\end{equation*}
\end{enumerate}
This is precisely corresponding to Definition~\zref{cat-0000}.

\begin{node}[Examples]\label{internal-0001}%
A category object internal to $\Set$ is just the usual notion of a
category \zref{cat-0001}.
\end{node}

\begin{node}\label{cat:internal-000N}%
The internal category in $\Grp$ is a crossed module~\cite{pirashvili2024internal}.
\end{node}
\end{definition}

\begin{definition}\label{internal-0002}%
Let $\cat{C}$ be a category. A \define{Groupoid object} internal to
$\cat{C}$ consists of a category object $X$ internal to $\cat{C}$
equipped with
\begin{enumerate}
\item a morphism $\iota\colon X_{1}\to X_{1}$
\end{enumerate}
such that
\begin{enumerate}
\item $t=\iota\circ s$ and $s=\iota\circ t$;
\item left invertibility for internal morphisms: if $\Delta\colon X_{1}\to X_{1}\times X_{1}$ is the diagonal
  morphism, then
\begin{equation*}
\begin{tikzcd}[sep=large]
  X_{1} \ar[d,"s"] \ar[r,"\Delta"] & X_{1}\mathbin{{}_{t}\times_{X_{0},t}}X_{1} \ar[r,"\id\times\iota"]&\ar[d,"c"]X_{1}\mathbin{{}_{t}\times_{X_{0},s}}X_{1}\\
X_{0} \ar[rr,"e"] & & X_{1}
\end{tikzcd}
\end{equation*}
\item right invertibility for internal morphisms:
\begin{equation*}
\begin{tikzcd}[sep=large]
  X_{1} \ar[d,"t"] \ar[r,"\Delta"] & X_{1}\mathbin{{}_{s}\times_{X_{0},s}}X_{1} \ar[r,"\iota\times\id"]&\ar[d,"c"]X_{1}\mathbin{{}_{t}\times_{X_{0},s}}X_{1}\\
X_{0} \ar[rr,"e"] & & X_{1}
\end{tikzcd}
\end{equation*}
\end{enumerate}

\begin{node}[Example]\label{internal-0005}%
A groupoid object internal to $\Set$ is just a groupoid as we know and
love it~\zref{cat-000B}.
\end{node}
\begin{node}[Example]\label{internal-0006}%
Lie groupoids~\zref{hda-0009} are groupoids internal to the category of
smooth manifolds.
\end{node}
\end{definition}

\begin{node}[References]\label{internal-0003}%
I haven't found a suitable reference for this definition. Internal
categories may be found in Mac~Lane~\cite[XII\S1]{mac1998categories}, Baez and Crans~\cite[\S2]{baez2010higherdimensional},
the note of Forrester-Barker~\cite{forresterbarker2002group}, and elsewhere.
P\^{e}gas~\cite{pegas2013groupoids} and
Martins-Ferreira~\cite{martinsferreira2022internal} provide definitions
for internal groupoids. The commutative diagrams for the groupoid's left
and right invertibility laws should be compared to the diagram for the
group object~\zref{internal-0009}.
\end{node}
\end{node}
