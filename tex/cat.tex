% 26
\section{Basic Definitions}

\begin{node}[Size issues]\label{cat-0002}%
Depending on our choice of set theory, we have different notions of
``small'' and ``large'' collections. In \NBG\ set theory~\zref{sec:set:nbg}
``large'' collections are proper classes, and ``small'' collections are
sets. In Tarski--Grothendieck set theory~\zref{sec:set:tarski-grothendieck},
we usually pick some ``smallest'' Grothendieck universe $\mathcal{U}$
containing $\NN$, which would also contain $\RR$, $\CC$, and all other
sets we use in classical mathematics, then consider elements of
$\mathcal{U}$ as small collections. Since there's possibly more than one
choice of universe, we speak of ``$\mathcal{U}$-small'' and
``$\mathcal{U}$-large'' sets depending on if they are elements of
$\mathcal{U}$ or not. Further, subsets of $\mathcal{U}$ are sometimes
called ``$\mathcal{U}$-moderate''. For a review of the different notions
of ``small'' and ``large'' collections germane to category theory, see
Shulman~\cite{shulman2008set}.

For various constructions in Category theory, \NBG\ will prove not
enough. We could try \MK\ axioms, but it will still not be enough. Some
authors invent a notion of ``conglomerates'' for forming the collection
of all proper classes. But this will turn out to still not be enough. This
is why we turn to Tarski--Grothendieck set theory.
\end{node}

\begin{node}[Category]\label{cat-0001}%
\begin{definition}[Version 1]
A \define{Category} $\cat{C}$ consists of
\begin{enumerate}
\item a collection of objects $\ob(\cat{C})$;
\item for any objects $X,Y\in\ob(\cat{C})$, there is a (possibly empty)
  collection of morphisms $\hom(X,Y)$;
\item for each object $X\in\ob(\cat{C})$, an identity morphism $\id_{X}\in\hom(X,X)$;
\item for objects $X,Y,Z\in\ob(\cat{C})$ and morphisms $f\in\hom(X,Y)$,
  $g\in\hom(Y,Z)$, there exists a morphism $g\circ f\in\hom(X,Z)$
\end{enumerate}
such that
\begin{enumerate}
\item composition is associative: for any $W,X,Y,Z\in\ob(\cat{C})$, if
  $f\in\hom(W,X)$, $g\in\hom(X,Y)$, $h\in\hom(Y,Z)$, then $(h\circ g)\circ f=h\circ(g\circ f)$;
\item unit laws for composition: for any objects $X,Y\in\ob(\cat{C})$,
  if $f\in\hom(X,Y)$, then $\id_{Y}\circ f=f$ and $f\circ\id_{X}=f$.
\end{enumerate}
\end{definition}

\begin{definition}[Version 2]\label{cat-0000}%
A \define{Category} $\cat{C}$ consists of
\begin{enumerate}
\item a collection of objects $\ob(\cat{C})$;
\item a collection of morphisms $\hom(\cat{C})$
\end{enumerate}
equipped with
\begin{enumerate}
\item a mapping $\id\colon\ob(\cat{C})\to\hom(\cat{C})$ assigning to
  each object its identity morphism;
\item a mapping $\circ\colon\hom(\cat{C})\times\hom(\cat{C})\to\hom(\cat{C})$;
\item a source and target mapping $s,t\colon\hom(\cat{C})\to\ob(\cat{C})$;
\end{enumerate}
such that
\begin{enumerate}
\item source and target are respected by composition: $s(g\circ f)=s(f)$
  and $t(g\circ f)=t(g)$;
\item source and target are respected by identities: $s(\id_{X})=X$ and $t(\id_{X})=X$;
\item composition is associative: $(h\circ g)\circ f=h\circ(g\circ f)$
  whenever $t(f)=s(g)$ and $t(g)=s(h)$;
\item composition satisfies the unit laws: if $s(f)=X$ and $t(f)=Y$,
  then $\id_{Y}\circ f=f$ and $f\circ\id_{X}=f$.
\end{enumerate}
\end{definition}

\begin{node}[Examples]\label{cat-0007}%
\begin{node}\label{cat-0004}%
The large category $\Set$ of sets\dots depends greatly on your choice of
foundations. ``Morally'', $\ob(\Set)$ consists of all [``small''] sets, and its
morphisms are [``small''] functions between those sets. But the exact
details varies depending on whether our foundations is predicative or
impredicative, classical or intuitionistic, admit or negate the axiom of
choice, and so on.
\end{node}

\begin{node}\label{cat-000A}%
If $\mathcal{U}$ is a Grothendieck universe~\zref{tg-0001}, then we can
form $\mathcal{U}\mbox{-}\Set$ whose collection of objects is precisely
$\mathcal{U}$ and whose collection of morphisms is precisely the
functions which are elements of $\mathcal{U}$.
\end{node}

\begin{node}\label{cat-0005}%
The large category $\cat{Ens}$ or $\cat{SET}$ is the category whose object is
the proper class of all sets $\ob(\cat{SET})=\ob(\cat{Ens})=V$ and whose
morphisms is all functions. This is a large category.
\end{node}

\begin{node}\label{cat-0003}%
The large category $\Grp$ of groups consists of $\ob(\Grp)$ the collection of all
``small'' groups [i.e., whose underlying set belongs to the universe of
sets], and the morphisms are all group morphisms.
\end{node}

\begin{node}\label{cat-0006}%
The large category $\Top$ of topological spaces consists of the collection of
all ``small'' topological spaces and morphisms are continuous maps
between them.
\end{node}

\begin{node}\label{cat-0008}%
Let $\FF$ be a field. The large category $\Vect{\FF}$ has its objects be
[``small''] vector spaces over $\FF$, and morphisms are $\FF$-linear
mappings between them.
\end{node}

\begin{node}\label{cat-000F}%
The category $\Mfld$ of $C^{\infty}$ manifolds is a category whose
objects are smooth manifolds, and morphisms are smooth mappings between
them. 
\end{node}

\begin{node}\label{cat-000M}%
Let $\cat{C}$ be a category consisting of a single object. This encodes
a monoid in the collection of morphisms.
\end{node}
\end{node} % examples

\begin{definition}\label{cat-000O}%
Let $\cat{C}$ be a category. We may form the \define{Opposite} category
of $\cat{C}$, denoted $\cat{C}^{op}$, which consists of the same objects
but a morphism $f\colon X\to Y$ in $\cat{C}^{op}$ is precisely the same
as a morphism $f\colon Y\to X$ in $\cat{C}$, and composition of
morphisms in the opposite category $g\circ f$ is precisely the reverse
of composition in the original category $(f\circ g)_{orig}$.

\begin{node}[Duality principle]\label{cat-000P}%
Every construction we can do in a category $\cat{C}$ has a dual
construction in the opposite category $\cat{C}^{op}$ which is usually
prefixed by ``co-''.
\end{node}
\end{definition}

\begin{definition}\label{cat-000G}%
Let $\cat{C}$ be a category. We define a \define{Subcategory} of $\cat{C}$
to consist of a category $\cat{D}$ such that $\ob(\cat{D})\subset\ob(\cat{C})$
and for all objects $X,Y\in\ob(\cat{D})$,
$\hom_{\cat{D}}(X,Y)\subset\hom_{\cat{C}}(X,Y)$. The composition of
morphisms in $\cat{D}$ is given by restricting the composition operator
from $\cat{C}$. The identity morphisms in $\cat{D}$ are precisely the
identity morphisms of $\cat{C}$ restricted to the objects of $\cat{D}$.

\begin{definition}\label{cat-000H}%
Let $\cat{D}$ be a subcategory of $\cat{C}$. We say $\cat{D}$ is a
\define{Full} subcategory if for all objects $X,Y\in\ob(\cat{D})$, we
have $\hom_{\cat{D}}(X,Y)=\hom_{\cat{C}}(X,Y)$.
\end{definition}
\end{definition}
\end{node} % category

\begin{node}[Groupoids]\label{cat-000B}%
\begin{definition}\label{cat-000C}%
A \define{Groupoid} is a category whose morphisms are all isomorphisms.
\end{definition}
\begin{node}[Examples]\label{cat-000D}%
\begin{node}\label{cat-000I}%
Let $G$ be a group. Then we can form a groupoid whose set of objects is
just a singleton set, and collection of morphisms is $G$, and
composition operator is the law of composition from the group structure
on $G$. We can reverse this intuition, and assert \emph{all} groups are
``just'' one-object groupoids.
\end{node}

\begin{node}[Fundamental groupoid]\label{cat-000E}%
For any topological space $X$, we may form the fundamental groupoid
$\Pi_{1}(X)$ whose objects are the set of underlying points of $X$ and
morphisms $f\colon A\to B$ are homotopy equivalence classes of paths
from $A$ to $B$.  Composition then concatenates paths.
\end{node}
\end{node}
\end{node}

\begin{node}[Functors]\label{cat-0009}%
When we define a new mathematical gadget, it will be useful to turn our
attention to defining morphisms for our new gadget. We call the
morphisms of categories ``functors''.

\begin{definition}
Let $\cat{C}$, $\cat{D}$ be categories. We define a \define{Functor}
from $\cat{C}$ to $\cat{D}$ to be $F\colon\cat{C}\to\cat{D}$ consisting
of
\begin{enumerate}
\item a mapping of objects $F_{ob}\colon\ob(\cat{C})\to\ob(\cat{D})$;
\item a mapping of morphisms $F_{mor}\colon\hom_{\cat{C}}(X,Y)\to\hom_{\cat{D}}(F_{ob}(X),F_{ob}(Y))$;
\end{enumerate}
such that
\begin{enumerate}
\item $F$ preserves composition: $F_{mor}(g\circ f)=F_{mor}(g)\circ F_{mor}(f)$ whenever
  $g\circ f$ is defined in $\cat{C}$;
\item $F$ preserves identity morphisms: for each object $X\in\cat{C}$, $F_{mor}(\id_{X})=\id_{F_{ob}(X)}$.
\end{enumerate}
We often suppress the subscripts and just write things like
\[F\left(X\xrightarrow{f}Y\right)\quad=\quad F(X)\xrightarrow{F(f)}F(Y).\]
\end{definition}

\begin{node}[Examples]\label{cat-000J}%
\begin{node}\label{cat-000L}%
Let $\cat{C}$ be a category. We define the \define{Identity functor}
$\Id_{\cat{C}}$ which is just the identity function on objects, and the
identity function on morphisms.
\end{node}
  
\begin{node}[Linear representations of groups]\label{cat-000K}%
Let $\FF$ be a field, let $G$ be a group. We know we can form the
groupoid $\cat{G}$ from $G$ \zref{cat-000I}. Then $\FF$-linear
representations of $G$ are precisely the functions $\cat{G}\to\Vect{\FF}$.
\end{node}

\begin{node}\label{cat-000N}%
Let $\cat{C}$ be a locally-small category, let $X,Y\in\ob(\cat{C})$ be
objects. Then $\hom(X,-)\colon\cat{C}\to\Set$ maps each object
$B\in\cat{C}$ to the set of morphisms $\hom(X,A)$ and each function
$f\colon A\to B$ to $\hom(X,f)\colon\hom(X,A)\to\hom(X,B)$ sending
$g\mapsto f\circ g$ for each $g\in\hom(X,A)$.

We also have $\hom(-,Y)\colon\cat{C}^{op}\to\Set$ be a functor sending each
object $A\in\ob(\cat{C})$ to the set of morphisms $\hom(A,Y)$, and each
morphism $h\colon A\to B$ in $\cat{C}$ to $\hom(h,Y)\colon\hom(B,Y)\to\hom(A,Y)$
sending $g\in\hom(B,Y)$ to $g\circ h\in\hom(A,Y)$.
\end{node}
\end{node} % examples
\end{node}

\section{Universal Properties}
% 10

\begin{node}[Product objects]\label{universal-prop-0003}%
\begin{definition}\label{universal-prop-0004}%
Let $\cat{C}$ be a category, let $A$ and $B$ be objects in
$\cat{C}$. The \define{Product object} of $A$ and $B$ is the object
denoted $A\times B$ with morphisms $\pi_{1}\colon A\times B\to A$ and
$\pi_{2}\colon A\times B\to B$ such that for all objects $X\in\ob(\cat{C})$
and morphisms $f_{1}\colon X\to A$ and $f_{2}\colon X\to B$, there
exists a unique morphism $f\colon X\to A\times B$ such that the diagram
commutes
\begin{equation*}
\begin{tikzcd}[sep=large]
  & \arrow[dl,swap,"f_{1}"] X \arrow[d,dashed,"f"] \arrow[dr,"f_{2}"] & \\
A & \ar[l,"\pi_{1}"] A\times B \ar[r,swap,"\pi_{2}"] & B
\end{tikzcd}
\end{equation*}
\end{definition}

\begin{definition}\label{universal-prop-0009}%
More generally, in a category $\cat{C}$,
if we have a family of objects $\{A_{i}\}_{i\in I}\subset\ob(\cat{C})$
equipped with morphisms $\pi_{j}\colon\prod_{i\in I}A_{i}\to A_{j}$ for
each $j\in I$ such
that for any object $X$ and family of morphisms 
$f_{i}\colon X\to A_{i}$ there is a unique morphism $f\colon X\to\prod_{i\in I}A_{i}$
such that the following diagram commutes
\begin{equation*}
\begin{tikzcd}[sep=large]
  X\arrow[dr,"f_{j}"]\arrow[d,swap,dashed,"f"]  & \\
\prod_{i\in I}A_{i} \arrow[r,swap,"\pi_{j}"] & A_{j}
\end{tikzcd}
\end{equation*}
\end{definition}

\begin{node}[Examples]\label{universal-prop-0007}%
\begin{node}\label{universal-prop-0005}%
In $\Set$, the product object is just the usual direct product.
\end{node}

\begin{node}\label{universal-prop-0006}%
In $\Grp$, the product object is the usual product group.
\end{node}

\begin{node}\label{universal-prop-0008}%
In $\Top$, the product object is the usual product space. The topology
is the product topology.
\end{node}
\end{node} % examples
\end{node}

\begin{node}[Pullbacks]\label{universal-prop-0001}%
\begin{definition}\label{universal-prop-0000}%
In a category $\cat{C}$ with products, let $f\colon B\to A$ and $g\colon D\to A$
be morphisms. A \define{Pullback} over $A$ given by these morphisms is
the object $P$ such that for any other object $X$ with morphisms
$h_{1}\colon X\to B$ and $h_{2}\colon X\to D$, there exists a unique
$h\colon X\to P$ making the diagram commute:
\begin{equation*}
\begin{tikzcd}%[sep=huge]
  X \arrow[rrd,bend left,"h_{2}"]
    \arrow[ddr,bend right,swap,"h_{1}"]
    \arrow[dr,dashed,"h"] \\
  & P \arrow[d,"\pi_{1}"'] \arrow[r,"\pi_{2}"] & D \arrow[d,"g"]  \\
  & B \arrow[r,swap,"f"]  & A
\end{tikzcd}
\end{equation*}
\end{definition}

\begin{node}[Example]
In the category of sets, $X\times_{Z}Y=\{(x,y)\in X\times Y\mid f(x)=g(y)\}$.
\end{node}
\end{node}

\begin{node}\label{universal-prop-0002}%

\end{node}

\section{Internalization}
% 23

\begin{node}[Group object]\label{internal-0007}%
\begin{definition}
Let $\cat{C}$ be a category with finite products and a terminal object
$\mathbf{1}$. We define a \define{Monoid object} in $\cat{C}$ to consist
of an object $M$ equipped with morphisms $m\colon M\times M\to M$
and $e\colon\mathbf{1}\to M$ such that
\begin{enumerate}
\item $m$ is associative, i.e., the following diagram commutes
\begin{equation*}
\begin{tikzcd}[sep=large]
M\times M\times M\ar[d,swap,"m\times\id_{M}"]\ar[r,"\id_{M}\times m"] & M\times M\ar[d,"m"]\\
M\times M\ar[r,swap,"m"] & M
\end{tikzcd}
\end{equation*}
\item the unit laws hold (i.e., the identity element acts as expected),
  i.e., the following diagram commutes
\begin{equation*}
\begin{tikzcd}[sep=large]
\mathbf{1}\times M\ar[dr,swap,"\pi_{2}"]\ar[r,"e\times\id_{M}"] &M\times M\ar[d,"m"] & \ar[l,swap,"\id_{M}\times e"]\ar[dl,"\pi_{1}"]M\\
& M &
\end{tikzcd}
\end{equation*}
\end{enumerate}
\end{definition}

\begin{node}[Examples]\label{internal-000J}%
\begin{node}\label{internal-000K}%
Let $\FF$ be a field. A monoid object in $\Vect{\FF}$ (considered as a
monoidal category with tensor product of vector spaces) is precisely an
associative unital algebra over $\FF$.
\end{node}

\begin{node}\label{internal-000L}%
A monoid object in $\Set$ is just the usual notion of a monoid we learn
in abstract algebra.
\end{node}

\begin{node}[Joke]\label{internal-000M}%
A monoid in the category of endofunctors is a monad.
\end{node}
\end{node}% examples

\begin{definition}\label{internal-0009}%
Let $\cat{C}$ be a category with finite products and a terminal object
$\mathbf{1}$. A \define{Group object} in $\cat{C}$ consists of a monoid
object $(G,m,e)$ equipped with an ``inverse'' morphism $i\colon G\to G$
such that the following diagram commutes:
\begin{equation*}
\begin{tikzcd}[sep=large]
G\ar[d,swap,"!"]\ar[r,"\Delta"] & G\times G\ar[r,"i\times\id_{G}"] & G\times G\ar[d,"m"]\\
\mathbf{1}\ar[rr,swap,"e"] & & G
\end{tikzcd}
\end{equation*}
where $\Delta$ is the diagonal morphism defined by
$\pi_{1}\circ\Delta=\pi_{2}\circ\Delta$.
\end{definition}

\begin{node}[Examples]\label{internal-000A}%

\begin{node}\label{internal-000B}%
In $\Set$, a group object is simply the usual notion of a group we all
know and love.
\end{node}

\begin{node}\label{internal-000C}%
In $\Top$, a group object is precisely a topological group.
\end{node}

\begin{node}\label{internal-000D}%
In $\Mfld$, a group object is a Lie group.
\end{node}

\begin{node}\label{internal-000E}%
In $\Grp$, a group object looks like\dots well, what? We would have $G$
be a group, its law of composition $m\colon G\times G\to G$ be a group
morphism. Could it be different than the usual law of composition? 

We would have $e\colon\mathbf{1}\to G$ pick out the usual identity
element since $\mathbf{1}$ is the trivial group, and we have no choice
in the matter. This means that the identity element of $G$ is precisely
$e(1)$ the identity for the law of composition $m$.

Now, for any $a,b\in G$, we would have $m(a,b)=ab=m(b,a)=ba$, since
\begin{subequations}
\begin{align}
m(a,b) &= m(1\cdot a,b\cdot 1)\\
&=m(1,b)\cdot m(a,1)\quad\mbox{by group morphism}\\
&=m(b,1)\cdot m(1,a)\quad\mbox{by unit laws}\\
&=b\cdot a\\
&=m(b,1)\cdot m(1,a)\\
&=m(b\cdot 1,1\cdot a)\\
&=m(b,a),
\end{align}
\end{subequations}
and similar reasoning gives us the rest of the claim.

Therefore, a group object in $\Grp$ is an Abelian group.
\end{node}

\begin{node}\label{internal-000F}%
A group object in the category of supermanifolds $\sMfld$ is precisely a
super Lie group.
\end{node}

\begin{node}\label{internal-000G}%
If we consider the category of algebraic varieties, a group object is
simply an algebraic group. If we consider the category of schemes, then
a group object is a group scheme.
\end{node}
\end{node} % examples

\begin{node}[References]\label{internal-0008}%
See Mac Lane~\cite[III\S6]{mac1998categories} and
the note of Forrester-Barker~\cite{forresterbarker2002group}.
\end{node}
\end{node}

\begin{node}[Internal ring object]\label{internal-000H}%
\begin{definition}
Let $\cat{C}$ be a Cartesian monoidal category. A \define{Ring object}
in $\cat{C}$ consists of an object $R$ equipped with
\begin{enumerate}
\item addition $a\colon R\times R\to R$ and multiplication $m\colon R\times R\to R$
\item a zero element $z\colon\mathbf{1}\to R$
\item a unit element $e\colon\mathbf{1}\to R$
\item additive inversion $n\colon R\to R$
\end{enumerate}
such that the usual axioms apply as encoded by commutative diagrams ---
so $(R,a,z,n)$ is a group object and $(R,m,e)$ is a monoid object, and
distributivity holds.
\end{definition}

\begin{definition}
Let $(\cat{C},\otimes,\mathbf{1})$ be a monoidal category. Let
$(A,\mu,e)$ be a monoid in $(\cat{C},\otimes,\mathbf{1})$.
We define a \define{Left module object} in $\cat{C}$ over $A$ is an
object $M\in\cat{C}$ equipped with a morphism $\rho\colon A\otimes M\to M$
(called the action, or ``scalar multiplication'') such that
\begin{enumerate}
\item unitality: the following diagram commutes (where $\ell$ is the
  left unitor isomorphism of $\cat{C}$)
\begin{equation*}
\begin{tikzcd}
\mathbf{1}\otimes M\arrow[dr,swap,"\ell"]\ar[r,"e\otimes\id"] & A\otimes M\ar[d,"\rho"]\\
& M
\end{tikzcd}
\end{equation*}
\item Action property: the following diagram commutes (where $a_{A,A,M}$
  is the natural isomorphism called the ``actor'')
\begin{equation*}
\begin{tikzcd}[sep=large]
(A\otimes A)\otimes M \ar[r,"a_{A,A,M}","\sim"']\ar[d,"\mu\otimes M"] &A\otimes(A\otimes M)\ar[r,"A\otimes\rho"] & A\otimes M\ar[d,"\rho"]\\
A\otimes M \ar[rr,"\rho"] & & M
\end{tikzcd}
\end{equation*}
\end{enumerate}
\end{definition}

\begin{node}[References]\label{internal-000I}%
See Mac Lane~\cite[VII\S4]{mac1998categories} for left-module objects
(usually called ``left action objects'' in the category theory literature).
\end{node}
\end{node}

\begin{node}[Internal Category Theory]\label{internal-0004}%
\begin{definition}\label{internal-0000}%
Let $\cat{C}$ be a category. We define a \define{Category Object}
$X$ internal to $\cat{C}$ to consist of
\begin{enumerate}
\item an \textsc{object of objects} $X_{0}\in\cat{C}$
\item an \textsc{object of morphisms} $X_{1}\in\cat{C}$
\end{enumerate}
equipped with
\begin{enumerate}
\item source and target morphisms $s,t\colon X_{1}\to X_{0}$;
\item an identity-assignment $e\colon X_{0}\to X_{1}$;
\item a composition operator $c\colon X_{1}\times_{X_{0}}X_{1}\to X_{1}$;
\end{enumerate}
such that the usual laws of categories hold:
\begin{enumerate}
\item laws specifying the target and source of the identity morphisms
\begin{equation*}
\begin{tikzcd}[sep=large]
X_{0}\arrow[dr,swap,"\id"]\arrow[r,"e"] & X_{1}\arrow[d,"s"]\\
                                       & X_{0}
\end{tikzcd}\qquad
\begin{tikzcd}[sep=large]
X_{0}\arrow[dr,swap,"\id"]\arrow[r,"e"] & X_{1}\arrow[d,"t"]\\
                                       & X_{0}
\end{tikzcd}
\end{equation*}
\item laws specifying the source and target of composite morphisms
\begin{equation*}
\begin{tikzcd}[sep=large]
X_{1}\times_{X_{0}}X_{1}\ar[d,"p_{1}"] \ar[r,"c"] & X_{1}\ar[d,"s"]\\
    X_{1} \ar[r,"s"] & X_{0}
\end{tikzcd}\qquad
\begin{tikzcd}[sep=large]
X_{1}\times_{X_{0}}X_{1}\ar[d,"p_{2}"] \ar[r,"c"] & X_{1}\ar[d,"t"]\\
    X_{1} \ar[r,"t"] & X_{0}
\end{tikzcd}
\end{equation*}
\item associativity for composition
\begin{equation*}
\begin{tikzcd}[sep=large]
X_{1}\times_{X_{0}}X_{1}\times_{X_{0}}X_{1}\ar[r,"c\times_{X_{0}}\id"]\ar[d,"\id\times_{X_{0}}c"]&X_{1}\times_{X_{0}}X_{1}\ar[d,"c"]\\
  X_{1}\times_{X_{0}}X_{1}\ar[r,"c"] & X_{1}
\end{tikzcd}
\end{equation*}
\item the left and right unit laws for composition of morphisms:
\begin{equation*}
\begin{tikzcd}[sep=large]
X_{0}\times_{X_{0}} X_{1}\ar[r,"e\times\id"]\ar[dr,swap,"p_{2}"] & X_{1}\times_{X_{0}} X_{1}\ar[d,"c"] & \ar[l,swap,"\id\times e"] X_{1}\times_{X_{0}}X_{0}\ar[dl,"p_{1}"]\\
& X_{1} & 
\end{tikzcd}
\end{equation*}
\end{enumerate}
This is precisely corresponding to Definition~\zref{cat-0000}.

\begin{node}[Examples]\label{internal-0001}%
A category object internal to $\Set$ is just the usual notion of a
category \zref{cat-0001}.
\end{node}
\end{definition}

\begin{definition}\label{internal-0002}%
Let $\cat{C}$ be a category. A \define{Groupoid object} internal to
$\cat{C}$ consists of a category object $X$ internal to $\cat{C}$
equipped with
\begin{enumerate}
\item a morphism $\iota\colon X_{1}\to X_{1}$
\end{enumerate}
such that
\begin{enumerate}
\item $t=\iota\circ s$ and $s=\iota\circ t$;
\item left invertibility for internal morphisms: if $\Delta\colon X_{1}\to X_{1}\times X_{1}$ is the diagonal
  morphism, then
\begin{equation*}
\begin{tikzcd}[sep=large]
  X_{1} \ar[d,"s"] \ar[r,"\Delta"] & X_{1}\mathbin{{}_{t}\times_{X_{0},t}}X_{1} \ar[r,"\id\times\iota"]&\ar[d,"c"]X_{1}\mathbin{{}_{t}\times_{X_{0},s}}X_{1}\\
X_{0} \ar[rr,"e"] & & X_{1}
\end{tikzcd}
\end{equation*}
\item right invertibility for internal morphisms:
\begin{equation*}
\begin{tikzcd}[sep=large]
  X_{1} \ar[d,"t"] \ar[r,"\Delta"] & X_{1}\mathbin{{}_{s}\times_{X_{0},s}}X_{1} \ar[r,"\iota\times\id"]&\ar[d,"c"]X_{1}\mathbin{{}_{t}\times_{X_{0},s}}X_{1}\\
X_{0} \ar[rr,"e"] & & X_{1}
\end{tikzcd}
\end{equation*}
\end{enumerate}

\begin{node}[Example]\label{internal-0005}%
A groupoid object internal to $\Set$ is just a groupoid as we know and
love it~\zref{cat-000B}.
\end{node}
\begin{node}[Example]\label{internal-0006}%
Lie groupoids~\zref{hda-0009} are groupoids internal to the category of
smooth manifolds.
\end{node}
\end{definition}

\begin{node}[References]\label{internal-0003}%
I haven't found a suitable reference for this definition. Internal
categories may be found in Mac~Lane~\cite[XII\S1]{mac1998categories}, Baez and Crans~\cite[\S2]{baez2010higherdimensional},
the note of Forrester-Barker~\cite{forresterbarker2002group}, and elsewhere.
P\^{e}gas~\cite{pegas2013groupoids} and
Martins-Ferreira~\cite{martinsferreira2022internal} provide definitions
for internal groupoids. The commutative diagrams for the groupoid's left
and right invertibility laws should be compared to the diagram for the
group object~\zref{internal-0009}.
\end{node}
\end{node}

