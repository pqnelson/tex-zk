

\begin{node}[Semantics]\label{fol-0005}%
\begin{definition}\label{fol-0006}%
An \define{Interpretation} (or ``\textit{Model}\,'') $M$ specifies the
interpretation of function and predicate symbols, consisting of:
\begin{enumerate}
\item A nonempty set $D$ called the \define{Domain} of the interpretation;
\item An assignment sending each $n$-ary function symbol $f$ to a mapping
  $f_{M}\colon D^{n}\to D$;
\item An assignment sending each $n$-ary relation symbol $P$ to a
  relation $P_{M}\subset D^{n}$.
\end{enumerate}
\begin{node}[Remark: Nonempty domain]\label{fol-000A}\index{Logic!Free}\index{Free Logic}%
Note this definition precludes empty domains $D\neq\emptyset$. Classical
logic \emph{requires} nonempty domains of discourse. If we allow empty
domains, we end up with something called ``free logic''. Within free
logic, the tautology $(\forall x,P[x])\implies P[y]$ fails to hold.
More serious of a problem, for us, is that the existence of prenex
normal forms fail when empty domains are allowed.
\end{node}

\begin{node}[Remark: finite mechanization]\label{fol-000B}%
We will implement this in \SML\ using finite domains of discourse.
Pulgar\'{\i}n and Uribe-Zapata~\cite{pulgarin2023came} wrote a paper
trying to formalize a finitistic intuitionistic metatheory, which is
what Kunen~\cite{kunen2009foundations} advocates.

When $D$ is infinite (logicians allow $D$ to be uncountable), our
strategy we take in \zref{fol-0009} will fail and we cannot realize
\texttt{holds} as a \SML\ function.
\end{node}

\begin{node}[Remark: On semantics]\label{fol-000C}%
Other foundations do not concern themselves too much with
[model theoretic] semantics. In fact, the only other family of
foundations which has a notion of models appears to be higher-order
logic (HOL). This is because Type Theory focuses on more proof theoretic
concerns.

\begin{node}[Cynical perspective]\label{fol-000H}%
After thinking about this further, I think what is happening ``really''
(from a cynic's perspective) is that we have a theory of sets $S$
(presumably one which generalizes the \ZF\ axioms suitably), and then we
take our theory $\mathfrak{T}=(D,\mathfrak{R},\mathfrak{F})$ and form an ``interpretation''
$\iota$ which consists of an assignment of $D$ to a ``set'' $\iota(D)$,
and each relation $R\in\mathfrak{R}$ of arity $n$ to a ``set''
$\iota(R)\subset\iota(D)^{n}$, and each function $f\in\mathfrak{F}$ of
arity $k$ to the ``function'' $\iota(f)\colon\iota(D)^{k}\to\iota(D)$.
As a cynic, these are all working at the level of syntax.

If we stipulate this cynical perspective is ``true enough'', then
semantics of a theory (first-order or otherwise) in a type theory is a
similar assignment at the level of syntax. There are no ``real models'':
it's symbol manipulation all the way down. We just pick some foundations
for being ``real'', then assert ``mathematical objects exist if they can
be encoded in our foundations''.
\end{node}
\begin{node}\label{fol-000I}%
After thinking about this cynical perspective, I think it more or less
works provided the choice of foundations forms a topos (or could encode
a topos). This is why set theory works as a choice of foundations,
because $\Set$ forms a topos.

The overall cynical perspective (``Pick one theory as a foundations,
then interpret other theories in terms of our foundational theory'') is
not unique. It appears Farmer, Guttman, and
Javier~\cite{farmer1992little} discusses a similar perspective, calling
such a perspective ``small
theories''. Enderton~\cite[\S2.7]{enderton2001mathematical} is the only book I
have found discussing ``interpretation of theories in other theories''. John
Harrison~\cite[\S1.4]{harrison1996formalized} argues Bourbaki worked
\emph{contra} this perspective, using a ``big theory'' rather than a
constellation of ``small theories''.
\end{node}
\end{node}
\end{definition}

\begin{definition}\label{fol-0007}%
Let $M$ be an interpretation with domain $D$. A \define{Valuation} in
first-order logic is a mapping from the set of variables to $D$.
This is consistent with our usage of the term in propositional logic
\zref{prop-0004} which maps our set of propositional variables to the
set of truth values.
\end{definition}

\begin{definition}\label{fol-0008}%
We define the value of a term in a particular interpretation $M$ and
valuation $v$ by the mapping $\mathtt{termval}~M~v~t$ by induction on
the structure of $t$ as:
\begin{center}
\begin{tabular}{rcl}
$\mathtt{termval}~M~v~x$ & $=$ & $v(x)$\\
$\mathtt{termval}~M~v~(f(t_{1},\dots,t_{n}))$ & $=$ & $f_{M}(\mathtt{termval}~M~v~t_{1},\dots,\mathtt{termval}~M~v~t_{n})$\\
\end{tabular}
\end{center}
\end{definition}

\begin{definition}\label{fol-0009}%
The main question for semantics is whether a formula holds (i.e., has
its value be ``true'' $\verum$) in a particular interpretation $M$ and
valuation $v$ is defined inductively as the function $\mathtt{holds}~M~v~p$.
We use the notation
\begin{equation}
((x\mapsto c)v)(y) = \begin{cases}c & \mbox{if }x=y\\
v(y) & \mbox{otherwise}.
  \end{cases}
\end{equation}
Now $\mathtt{holds}~M~v~p$ may be given by the structure of the formula $p$ as:
\begin{center}
\begin{tabular}{rcp{0.5\linewidth}}
$\mathtt{holds}~M~v~\falsum$ & $=$ & false\\
$\mathtt{holds}~M~v~\verum$ & $=$ & true\\
$\mathtt{holds}~M~v~(R(t_{1},\dots,t_{n}))$ & $=$ & $R_{M}(\mathtt{termval}~M~v~t_{1},\dots,\mathtt{termval}~M~v~t_{n})$\\
$\mathtt{holds}~M~v~(\neg p)$ & $=$ & not $(\mathtt{holds}~M~v~p)$\\
$\mathtt{holds}~M~v~(p\land q)$ & $=$ & $(\mathtt{holds}~M~v~p)$ and $(\mathtt{holds}~M~v~q)$\\
$\mathtt{holds}~M~v~(p\lor q)$ & $=$ & $(\mathtt{holds}~M~v~p)$ or $(\mathtt{holds}~M~v~q)$\\
$\mathtt{holds}~M~v~(p\implies q)$ & $=$ & either not $(\mathtt{holds}~M~v~p)$ or $(\mathtt{holds}~M~v~q)$\\
$\mathtt{holds}~M~v~(p\iff q)$ & $=$ & $(\mathtt{holds}~M~v~p)=(\mathtt{holds}~M~v~q)$\\
$\mathtt{holds}~M~v~(\forall x.p)$ & $=$ & for all $c\in D$,
  $\mathtt{holds}~M~((x\mapsto c)v)~p$\\
$\mathtt{holds}~M~v~(\exists x.p)$ & $=$ & there is at least one $c\in D$ such that
  $\mathtt{holds}~M~((x\mapsto c)v)~p$\\
\end{tabular}
\end{center}
\end{definition}

\begin{definition}\label{fol-000D}%
We call a first-order formula \define{(Logically) Valid} if it holds in all
interpretations and all valuations. If $\varphi\iff\psi$ is logically
valid, then we say $\varphi$ and $\psi$ are \define{Logically Equivalent}.
Valid formulas are the first-order counterparts to
tautologies~\zref{prop-000L} in propositional logic.
\end{definition}

\begin{definition}\label{fol-000E}%
We say that an interpretation $M$ \define{Satisfies} a first-order
formula $\varphi$ (or simply that $\varphi$ \textit{holds in} $M$) if
\textbf{for all valuations} $v$ we have $\mathtt{holds}~M~v~\varphi$
be true.

We can extend this notion for a set of first-order formulas $S$, saying
an interpretation $M$ satisfies $S$ if for each $\varphi\in S$ we have
$M$ satisfy $\varphi$.
\end{definition}

\begin{definition}\label{fol-000F}%
We say a first-order formula $\varphi$ (or set of formulas) is
\define{Satisfiable} if there exists some interpretation which satisfies it.
That is, there exists an interpretation $M$ such that for all valuations
$v$ we have $\mathtt{holds}~M~v~\varphi$ evaluate to true.

The valuation plays a role only if there are free variables in $\varphi$.
\end{definition}
\end{node}
