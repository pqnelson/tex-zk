

\begin{node}[Syntax]\label{fol-0001}%
\begin{definition}\label{fol-0002}%
Let $V$ be a (nonempty, countably infinite) set of variables. We define
a \define{Term} in first-order logic inductively as
\begin{enumerate}
\item variables are terms;
\item constants are terms;
\item a ``function'' is a term, i.e., a term parametrized by a fixed
  number (called its \define{arity}) of terms.
\end{enumerate}
We can also specify terms by the following grammar (where
constants are functions with zero arguments):
\begin{center}
\begin{tabular}{rcl}
$\langle$\textit{term}$\rangle$ & ::= & $\langle$\textit{variable}$\rangle$\\
& $|$ & $\langle$\textit{function symbol}$\rangle$ \verb#(# $\langle$\textit{arguments}$\rangle$ \verb#)#\\
  & & \\
$\langle$\textit{arguments}$\rangle$ & ::= & $\langle$\textit{empty}$\rangle$\\
  & $|$ & $\langle$\textit{argument list}$\rangle$\\
  & & \\
$\langle$\textit{argument lists}$\rangle$ & ::= & $\langle$\textit{term}$\rangle$\\
  & $|$ & $\langle$\textit{term}$\rangle$ \verb#,# $\langle$\textit{argument list}$\rangle$\\
\end{tabular}
\end{center}

\begin{node}\label{fol-000K}%
We can implement this with varying degrees of flexibility. The quickest
way is to ignore arity, and just use a list of terms. This is fine.

But a bigger consideration worth taking time to seriously think through
is how should we handle \emph{bound variables}? This seems like a silly
consideration, but it's an easy source of bugs. If we're just doing
first-order logic (i.e., we're not doing $\varepsilon$-calculus), then
de Bruijn indices should be seriously considered as a contender for
bound variables. In this case, we should add \texttt{BVar of Int}
for bound variables represented by de Bruijn indices. For now, we will
be cavalier and work with ``variables''.
\begin{sml}
datatype Term = Var of Var
              | Fn of string * (Term list);
\end{sml}
\end{node}
\end{definition}

\begin{definition}\label{fol-0003}%
Let $n\in\NN_{0}$ be a non-negative integer. A \define{Predicate} (or
\emph{Relation}) in first-order logic is a relation symbol followed by a
list of $n$ terms. This forms an \textit{Atomic Formula} in first-order logic.
Its BNF grammar is:
\begin{center}
\begin{tabular}{rcl}
$\langle$\textit{atomic formula}$\rangle$ & ::= & $\langle$\textit{predicate symbol}$\rangle$ \verb#(# $\langle$\textit{arguments}$\rangle$ \verb#)#\\
\end{tabular}
\end{center}

\begin{node}\label{fol-000L}%
We can implement this in \SML\ using similar considerations as functions
in terms. We will call this datatype \lstinline{Fol}, so we can work
with \texttt{Fol Formula} types to reinforce our intuition that we're
working with first-order logic formulas.

\begin{sml}
datatype Fol = R of string * (Term list);
\end{sml}
\end{node}
\end{definition}

\begin{definition}\label{fol-0004}%
In first-order logic, a \define{Formula} may be formed from atomic
formulas, the usual connectives from propositional logic, and quantified
formulas. When we replace atomic formulas by propositions, we see we can
recover propositional logic's grammar \zref{prop-000I}. The BNF grammar
for formulas in first-order logic is then: 
\begin{center}
\begin{tabular}{rcl}
$\langle$\textit{formula}$\rangle$ & ::= & $\langle$\textit{atomic formula}$\rangle$\\
& $|$ & $\neg\,\langle$\textit{formula}$\rangle$\\
& $|$ & $\langle$\textit{formula}$\rangle\land\langle$\textit{formula}$\rangle$\\
& $|$ & $\langle$\textit{formula}$\rangle\lor\langle$\textit{formula}$\rangle$\\
& $|$ & $\langle$\textit{formula}$\rangle\implies\langle$\textit{formula}$\rangle$\\
& $|$ & $\langle$\textit{formula}$\rangle\iff\langle$\textit{formula}$\rangle$\\
& $|$ & $\forall$ $\langle$\textit{variable}$\rangle\langle$\textit{formula}$\rangle$\\
& $|$ & $\exists$ $\langle$\textit{variable}$\rangle\langle$\textit{formula}$\rangle$\\
\end{tabular}
\end{center}
\end{definition}
\end{node}
