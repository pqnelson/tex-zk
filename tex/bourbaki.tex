% 12
\begin{node}\label{bourbaki-0000}%
The Bourbaki group has long since been intriguing to me. Everything about it: the organization, writing style, content of their books, etc.
\end{node}

\begin{node}[Writing style]\label{bourbaki-0001}%
Bourbaki's early work appeared to organize results into ``sections'' and
``subsections''. The exact words vary, sometimes calling ``subsections"
variously ``numbers'' or ``paragraphs''. For the sake of consistency, I
will use \LaTeX\ terminology (sections contain subsections).

It is unclear to me how Bourbaki's books were sold. At times, it sounds
like each individual chapter was bound and published as a separate
book. But other times, it sounds like \emph{eventually} the chapters
were collected and bound as we find them in the English translation.

\begin{node}[Subsections]\label{bourbaki-0006}%
Let me see, for the first three chapters of their \textit{Lie Groups and Lie Algebras} (English translation):


% 
\begin{center}
\begin{tabular}{|c|c|c|c|c|c|}
\hline\multicolumn{2}{|c|}{Ch.1}
& \multicolumn{2}{c|}{Ch.2} & \multicolumn{2}{c|}{Ch.3} \\\hline
Subsec. & Pages & Subsec. & Pages & Subsec. & Pages\\
      9 &   11  &      6  &    8  &     11  &    24\\
      9 &   13  &     11  &   14  &      3  &     5\\
      8 &   13  &      3  &    4  &     18  &    41\\
      5 &    5  &      6  &    7  &      7  &    18\\
      6 &    7  &      5  &    6  &      4  &     7\\
     10 &   18  &      5  &    9  &     10  &    22\\
      3 &    5  &      3  &    6  &      6  &    11\\
        &       &      4  &    6  &      2  &     5\\
        &       &         &       &      8  &    17\\
        &       &         &       &      3  &     9\\
      \hline
     50 &   72  &     43  &   60  &     72  &   159\\
      \hline
\end{tabular}
\end{center}

This is an average of 97 pages per chapter, $8.3333$ sections per chapter;
$11.64$ pages per section, $6.6$ subsections per section; 55 subsections per
chapter, and $1.76364$ pages per subsection.

(If memory serves, in the first volume of \textit{General Topology},
there was something on the order of magnitude of $1.5$ pages per
subsection. So this informal survey suggests an approximate size to the
subsections.)

This seems to be consistent with their early drafts\footnote{E.g., on
point-set topology
(\url{http://sites.mathdoc.fr/archives-bourbaki/PDF/144_nbr_046.pdf})
and the rest of the
archives (\url{http://archives-bourbaki.ahp-numerique.fr/items}).}.
\end{node}

\begin{node}[Theorem numbering]\label{bourbaki-0002}%
In \textit{Lie Groups and Lie Algebras}, lemmas, propositions, theorems,
and definitions are numbered independently and by section.
Examples are numbered independently, and by subsection. Corollaries are
sequentially numbered \emph{if there are more than one corollary}, and are
numbered after each lemma, proposition, theorem.

This is probably the worst ``design decision'' made, both for the reader
(how to locate a reference?) and the writer (keeping track of a half
dozen ``counters'' when writing a chapter).
\end{node}

\begin{node}[No ``proofs'']\label{bourbaki-0007}%
There are no explicit ``proof environments'' in Bourbaki's books.
\end{node}
\end{node}

\begin{node}[Choice of foundations]\label{bourbaki-0003}%
Bourbaki's foundations is Hilbert's epsilon
calculus~\zref{sec:epsilon-calc}, i.e., first-order calculus without
quantifiers but with a (global) ``choice operator''. The quantifiers are
defined using this choice operator, thus extendeding beyond first-order
logic. Hilbert denoted this operator by $\varepsilon$, Bourbaki used a
convoluted $\tau$ and system of linkages to avoid introducing bound
variables. Then, for any predicate $P[x]$, we would have
$\varepsilon_{x}P[x]$ represent some object which satisfies the
predicate (if one exists, otherwise it's an arbitrary object).

If memory serves, Bourbaki used a Hilbert system for its proof calculus,
and then never touches the subject again. Instead (after much irrelevant
work), Bourbaki relies on familiar natural deduction proofs.

Bourbaki's axioms for their set theory amounted to Zermelo set
theory. It wasn't until Grothendieck joined the group and added the
axiom that, to any set $X$ there exists a Grothendieck Universe~\zref{set:tg-0000} $U$
containing $X$. This made it far stronger than \ZFC\ set theory.

The difficulty is that the various constructs in set theory --- von
Neumann's stages $V_{\alpha}$, Godel's constructible $L_{\alpha}$,
hereditarily finite sets $H_{\alpha}$, and so on --- are not closed
under the choice operator $\tau$ in general.

A good reference on Bourbaki's \textit{Theory of Sets} is \textsc{Wayne Aitkens}'s commentary on the first chapter of Bourbaki's book.\footnote{\url{https://public.csusm.edu/aitken_html/Essays/Bourbaki/BourbakiSetTheory1.pdf}}
\end{node}

\begin{node}[Model Theory]\label{bourbaki-0004}%
Bourbaki does not consider model theory at all. Instead, they are purely
formalists: a formula is \emph{true} if it is provable. The idea of a
``true but unprovable'' proposition is meaningless to them.

However, they \textit{do} discuss notions of ``models'' and
``interpretations'' in the history section of chapter 4 of their
\textit{Theory of Sets} book. So they apparently are aware of the notion
of model theory.

I \emph{think} that Bourbaki did this deliberately, as a way of saying
we ``live'' in the world of sets, and ultimately work there, too. But I
have no proof to support this other than it appears to be in line with
their underlying philosophical views. If memory serves, \textsc{Rodin}'s
``One Mathematic(s) or Many?''~\cite{Rodin_2021} argues a similar perspective.
\end{node}

\begin{node}[Puzzle]\label{bourbaki-0005}%
If you were to form a ``modern Bourbaki group'', what exactly would be
the premise or gimmick to it?

Presumably the ``mission statement'' is the same (i.e., to provide a
coherent, uniform presentation of modern mathematics). But then what's
unique about it? I could hazard a few speculative guesses.

\begin{node}[Formal proofs]\label{bourbaki-0009}%
If I had to hazard a guess, I suppose it might be something along the
lines of something analogous to Java emerging in the 1990s as a way to
handle the myriad CPU architectures and difficulties porting software
from one CPU architecture to another. Now, in math, we have many
different possible foundations and different proof assistants, which are
incompatible with each other. A ``Bourbaki 2.0'' might try to formulate
a controlled subset of the English language [think: Mizar or ForTheL]
which ``compiles'' to different proof assistants, which is independent
of foundations. This would make the work readable on its own (as it is
written in a controlled subset of English) and verifiable (since it can
``compile'' to any proof assistant, and be checked for correctness).
\end{node}

\begin{node}[Template for definitions]\label{bourbaki-0008}%
We could also consider following John Baez's ``stuff, structure,
properties'' template\footnote{See \url{https://ncatlab.org/nlab/show/stuff,+structure,+property}} for definition. This can be made ``canonical'' by
referring to internal definitions\footnote{See \url{https://ncatlab.org/nlab/show/internalization}},
e.g., ``A group object in a category $\cat{C}$ consists of an object
$G\in\cat{C}$ equipped with morphisms\dots such that the following
diagrams commute: \dots''. This has the advantage that when we consider
a ``gadget'' internal to $\cat{Cat}$, we get one version of a strictly
categorified ``2-gadget'' (hence giving us one possible categorified
version of concepts).

\begin{node}[Grothendieck invented internalization]\label{bourbaki-000A}%
Note that internalization seems to be traced back to Grothendieck's
comments in \textit{FGA} \S4.2\footnote{English translation \url{https://thosgood.com/fga/FGA-3-II.html\#fga-3-ii-section-A.1}}
and groups with operators given in \S5.3\footnote{English translation \url{https://thosgood.com/fga/FGA-3-III.html\#fga-3-iii-section-3}}.
\end{node}

\begin{node}\label{bourbaki-000B}%
For a review of internal group objects and categories, see Forrester-Barker~\cite{forresterbarker2002group}.
\end{node}
\end{node}
\end{node}
