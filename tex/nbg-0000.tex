\begin{puzzle}
Can we formalize NBG set theory using the Mizar language? Arguably, the
unique primitive notion would be a ``class'' instead of a ``set'', and
we could define ``\texttt{mode set} \verb|->| \texttt{class means ex C
  being class st it in C}''. The primitive notion ``\texttt{in}'' is a
predicate taking a mathematical object as its left argument and a class
as its right argument. We also have a primitive equality predicate.

\begin{node}[References]
There are few good references on NBG set theory. Banakh
(\arXiv{2006.01613}) has some lecture notes. Someone has done some
heroic work on the English wikipedia page for NBG set theory.
\end{node}
\begin{node}[NBG is about sets]
Remember, the whole point of NBG set theory is to reason about sets. We
just get the ability to explicitly make statements concerning proper
classes as a side effect.
\end{node}
\end{puzzle}
