% 12
\begin{node}\label{epsilon-calc-0000}%
Hilbert invented this logical calculus, which basically is first-order
logic without quantifiers, and has instead an operator $\varepsilon$
with a signature
\[\varepsilon\colon\langle\textit{variable}\rangle\to\langle\textit{predicate}\rangle\to\langle\textit{formula}\rangle\]
We informally interpret $\varepsilon x.P[x]$ as ``The term which
satisfies predicate $P[x]$ if one exists, otherwise an arbitrary object''.

We probably would not remember it, but for its appearance in Bourbaki's
\textit{Theory of Sets}~\cite{bourbaki1968theory} as $\tau$.

\begin{node}[References]\label{epsilon-calc-0001}%
Zach~\cite{zach2001practice,zach2016semantics} has written about proof theory related to
it, Wirth~\cite{Wirth2008epsilon,wirth2017simplified} reviews attempts to relate
$\varepsilon$ to indefinite committed choice operators. Claus-Peter Wirth
and Richard Zach have both written a bit about this subject, and I defer
to them on this subject. In particular, Wirth~\cite{wirth2017simplified}
has reviewed the different formalizations of the $\varepsilon$ (and
related) operators by Hilbert and friends.
I am relying on material from Wirth's 2015 talk ``The Descriptive
Operators iota, tau, and epsilon'' for the history~\zref{epsilon-calc-0002},
which is reviewed in the companion paper~\cite[\S4]{wirth2017simplified}.
Wirth offers a formalization using ``variables'' (as we normally think
of them), ``free atoms'' (also called ``eigenvariables'' or
``parameters'' in the literature), and ``bound atoms''.
\end{node}
\end{node}

\begin{node}[History]\label{epsilon-calc-0002}%

\begin{node}[Peano's iota]\label{epsilon-calc-0003}%
Frege (1893) used a bold backslash in front of a [unary] predicate to
refer to \emph{the} term satisfying the predicate, if one exists, and an
arbitrary-but-fixed object otherwise. Peano writes either
``$\overline{\iota}$'' (1896) or an inverted ``$\iota$'' (1899). Russell
and Whitehead's \textit{Principia} uses an inverted ``$\iota$''. Hilbert
and Bernays (1934) requires a proof of $\exists!x.P[x]$ before we can
form $\iota x.P[x]$.
\end{node}

\begin{node}[Hilbert's tau]\label{epsilon-calc-0004}%
Hilbert's \textit{Die Logischen Grundlagen der Mathematik} (talk dated
September 1922) introduces a \emph{transfinite  function} denoted $\tau$.
Its ``type signature'' $\tau\colon(i\to o)\to o$ takes a predicate and
returns an ``individual'' [i.e., a term]. If we let $P\colon i\to o$
be a predicate, Hilbert's transfinite axiom 11 gives us
$A[\tau A]\implies A[a]$.
In a footnote, Hilbert acknowledges Bernays for showing Hilbert that
this axiom suffices for everything he needs.

\begin{node}[Caution]\label{epsilon-calc-0005}%
\begin{enumerate}
\item This is a \emph{different} $\tau$ than the one appearing in
  Kneser's private notes to Hilbert's 1921--1922 lectures
  \textit{Grundlagen der Mathematik}.
\item An ``$\varepsilon$'' is written for the $\tau$ in Kneser's
  private notes to Hilbert's 1922--1923 lectures
  \textit{Grundlagen der Mathematik}.
\item Nicholas Bourbaki~\cite{bourbaki1968theory} uses a ``$\tau$'', but
  it really refers to Hilbert's ``$\varepsilon$''.
\end{enumerate}
\end{node}
\end{node}

\begin{node}[Hilbert' epsilon]\label{epsilon-calc-0006}%
Ackermann's 1924 PhD thesis seems to have introduced the $\varepsilon$
notation. The name for the axiom $A[a]\implies A[\varepsilon A]$, or in
Wirth's reading $\exists x.A[x]\implies A[\varepsilon A]$, has
changed variously (Ackermann in his thesis calls it ``transfinite axiom 1'' and
treated $\varepsilon$ as a binder, Hilbert calls it the axiom of choice
in a \textit{\"{U}ber das Unendliche} lecture in June 1925,
the logical $\varepsilon$-axiom in a
\textit{Die Grundlagen der Mathematik} lecture dated July 1927, and Hilbert and
Bernays's \textit{Grundlagen der Mathematik} [vol 2] calls it the ``$\varepsilon$-formula'' in 1939 and treats it as
a binder). In the case when $\exists!x.P[x]$, the Peano's iota and
Hilbert's epsilon picks out the same object.
\end{node}
\end{node}

\begin{node}\label{epsilon-calc-0007}%
Ackermann's (II,4), Bourbaki's S7 axiom~\cite[I~\S5.1]{bourbaki1968theory}, Leisenring's (E2)
offers the axiom
\[\forall x(P[x]\iff Q[x])\implies \varepsilon x.P[x] = \varepsilon x.Q[x]\]
There are two difficulties Wirth identifies with this axiom:
\begin{enumerate}
\item All classical theorems become intuitionistic ones (which doesn't
  seem too terrible to me).
\item Committed choice should be an option, not a must. What does this mean?

  For example, ``The Holy Ghost is a father of Jesus'' and ``Joseph is a
  father of Jesus'' should not \emph{commit} us to conclude that Joseph is the
  Holy Ghost.

  For a more mathematical example, there is no solution to
  $x\neq x$, so $\varepsilon x.(x\neq x)$ can be replaced by anything,
  so in the equation
\begin{equation*}
\varepsilon x.(x\neq x)\neq\varepsilon x.(x\neq x)
\end{equation*}
we can independently replace the left-hand side by anything, independent
of what we choose for the right-hand side, thus enabling the following
chain of reasoning:
\begin{align*}
\varepsilon x.(x\neq x)&\neq\varepsilon x.(x\neq x)\\
0&\neq\varepsilon x.(x\neq x)\\
0&\neq1
\end{align*}
which gives us a contradiction. What went wrong? When we chose 0 for the
left-hand side, we should have committed to this choice, and it should
have forced our hand to pick 0 for the right-hand side as well.
\end{enumerate}
\end{node}

\begin{node}[Quantifiers]\label{epsilon-calc-0008}%
From the axiom
\begin{equation*}\tag{$\varepsilon_{0}$}
\exists x.P[x] \implies P\bigl[\varepsilon x.P[x]\bigr]
\end{equation*}
we can define quantifiers in terms of $\varepsilon$ using the tautologies
\begin{equation*}\tag{$\varepsilon_{1}$}
\exists x.P[x]\iff P\bigl[\varepsilon x.P[x]\bigr],
\end{equation*}
and then using this and $\forall x.P[x]\iff\neg(\exists x.\neg P[x])$,
\begin{equation*}\tag{$\varepsilon_{2}$}
\forall x.P[x]\iff P\bigl[\varepsilon x.\neg P[x]\bigr]
\end{equation*}
Really? Well, let $Q[x]=\neg P[x]$ for simplicity,
then $\neg(\exists x.\neg P[x])=\neg(\exists x.Q[x])$ and using $(\varepsilon_{1})$
gives us
$\neg(\exists x.Q[x])=\neg(Q\bigl[\varepsilon x.Q[x]\bigr])$. When we
substitute in $Q[x]=\neg P[x]$ back into this, we obtain
$\neg(Q\bigl[\varepsilon x.Q[x]\bigr])=\neg(\neg P\bigl[\varepsilon x.\neg P[x]\bigr])$.
The law of double negation gives us the result $(\varepsilon_{2})$.
\end{node}

\begin{node}[Epsilon Theorems]\label{epsilon-calc-0009}%
Hilbert and Bernays's \textit{Grundlagen der Mathematik}, vol 2 (1939)
relates deductions made by the $\varepsilon$ calculus to those made in
the ``usual'' logic.

\begin{theorem}[First epsilon]\label{epsilon-calc-000A}%
Suppose $\Gamma\cup\{A\}$ is a set of quantifier-free formulas not
involving the $\varepsilon$ symbol. If $A$ is derivable from $\Gamma$ in
the $\varepsilon$ calculus, then $A$ is derivable from $\Gamma$ in
quantifier-free predicate logic.
\end{theorem}

\begin{theorem}[Second epsilon]\label{epsilon-calc-000B}%
Suppose $\Gamma\cup\{A\}$ is a set of formulas not involving the
$\varepsilon$ symbol. If $A$ is derivable from $\Gamma$ in the
$\varepsilon$ calculus, then $A$ is derivable from $\Gamma$ in predicate
logic.
\end{theorem}
\end{node}
