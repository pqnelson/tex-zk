% 24
\begin{node}\label{fol-0000}%
I'm nonplussed with Mendelson's treatment~\cite{mendelson2015mathematical}
of first-order logic using a Hilbert system, but his treatment of
semantics (\S2.2) is adequate. I'm also borrowing heavily from the
notation used by John Harrison~\cite{harrison2009handbook} for explicit
discussion of the semantics of first-order logic. This is where nonlogical
objects enter the game. Unlike Mendelson, I'm not interested in
philosophical word games (``Translate into first-order logic: `Every
mathematician loves music'\,'' type questions are just inane).
\end{node}

\transclude{fol/syntax}
\transclude{fol/semantics}

\begin{node}[Testing for validity]\label{fol-000M}%
The trick for determining if $\varphi$ is valid (i.e., it's a theorem),
is to disprove $\neg\varphi$ is unsatisfiable.

\begin{node}[Resolution methods]\label{fol-000N}%
There is an entire family of methods based on transforming $\neg\varphi$ to
some suitable equivalid (or equisatisfiable) formula $\psi$, then apply
the resolution rule until we arrive at a contradiction. The resolution
rule may be viewed as \textit{modus ponens} in a slightly more
generalized form, since \textit{modus ponens} is (rewriting
$\varphi\implies\alpha$ as $\neg\varphi\lor\alpha$),
\begin{equation*}
\infer{\vdash \alpha}{\vdash\alpha\lor\neg\varphi,&\vdash\varphi}.
\end{equation*}
The hypothetical syllogism takes the form
\begin{equation*}
\infer{\vdash\alpha\implies\beta}{\vdash\alpha\implies\varphi,&\vdash\varphi\implies\beta}.
\end{equation*}
Transforming the implications into disjunctions,
\begin{equation}
\infer{\vdash \alpha\lor\beta}{\vdash\alpha\lor\varphi,&\vdash\neg\varphi\lor\beta}.
\end{equation}
\end{node}
\end{node}

\begin{node}[References]\label{fol-000J}%
Emil Post~\cite{post1921introduction} is credited for inventing this
distinction between ``syntax'' and ``semantics'' in logic.
\end{node}
