% 22
\begin{node}\label{fol-0000}%
I'm nonplussed with Mendelson's treatment~\cite{mendelson2015mathematical}
of first-order logic using a Hilbert system, but his treatment of
semantics (\S2.2) is adequate. I'm also borrowing heavily from the
notation used by John Harrison~\cite{harrison2009handbook} for explicit
discussion of the semantics of first-order logic. This is where nonlogical
objects enter the game. Unlike Mendelson, I'm not interested in
philosophical word games (``Translate into first-order logic: `Every
mathematician loves music'\,'' type questions are just inane).
\end{node}

\begin{node}[Syntax]\label{fol-0001}%
\begin{definition}\label{fol-0002}%
Let $V$ be a (nonempty, countably infinite) set of variables. We define
a \define{Term} in first-order logic inductively as
\begin{enumerate}
\item variables are terms;
\item constants are terms;
\item a ``function'' is a term, i.e., a term parametrized by a fixed
  number (called its \define{arity}) of terms.
\end{enumerate}
We can also specify terms by the following grammar (where
constants are functions with zero arguments):
\begin{center}
\begin{tabular}{rcl}
$\langle$\textit{term}$\rangle$ & ::= & $\langle$\textit{variable}$\rangle$\\
& $|$ & $\langle$\textit{function symbol}$\rangle$ \verb#(# $\langle$\textit{arguments}$\rangle$ \verb#)#\\
  & & \\
$\langle$\textit{arguments}$\rangle$ & ::= & $\langle$\textit{empty}$\rangle$\\
  & $|$ & $\langle$\textit{argument list}$\rangle$\\
  & & \\
$\langle$\textit{argument lists}$\rangle$ & ::= & $\langle$\textit{term}$\rangle$\\
  & $|$ & $\langle$\textit{term}$\rangle$ \verb#,# $\langle$\textit{argument list}$\rangle$\\
\end{tabular}
\end{center}

\begin{node}\label{fol-000K}%
We can implement this with varying degrees of flexibility. The quickest
way is to ignore arity, and just use a list of terms. This is fine.

But a bigger consideration worth taking time to seriously think through
is how should we handle \emph{bound variables}? This seems like a silly
consideration, but it's an easy source of bugs. If we're just doing
first-order logic (i.e., we're not doing $\varepsilon$-calculus), then
de Bruijn indices should be seriously considered as a contender for
bound variables. In this case, we should add \texttt{BVar of Int}
for bound variables represented by de Bruijn indices. For now, we will
be cavalier and work with ``variables''.
\begin{sml}
datatype Term = Var of Var
              | Fn of string * (Term list);
\end{sml}
\end{node}
\end{definition}

\begin{definition}\label{fol-0003}%
Let $n\in\NN_{0}$ be a non-negative integer. A \define{Predicate} (or
\emph{Relation}) in first-order logic is a relation symbol followed by a
list of $n$ terms. This forms an \textit{Atomic Formula} in first-order logic.
Its BNF grammar is:
\begin{center}
\begin{tabular}{rcl}
$\langle$\textit{atomic formula}$\rangle$ & ::= & $\langle$\textit{predicate symbol}$\rangle$ \verb#(# $\langle$\textit{arguments}$\rangle$ \verb#)#\\
\end{tabular}
\end{center}

\begin{node}\label{fol-000L}%
We can implement this in \SML\ using similar considerations as functions
in terms. We will call this datatype \lstinline{Fol}, so we can work
with \texttt{Fol Formula} types to reinforce our intuition that we're
working with first-order logic formulas.

\begin{sml}
datatype Fol = R of string * (Term list);
\end{sml}
\end{node}
\end{definition}

\begin{definition}\label{fol-0004}%
In first-order logic, a \define{Formula} may be formed from atomic
formulas, the usual connectives from propositional logic, and quantified
formulas. When we replace atomic formulas by propositions, we see we can
recover propositional logic's grammar \zref{prop-000I}. The BNF grammar
for formulas in first-order logic is then: 
\begin{center}
\begin{tabular}{rcl}
$\langle$\textit{formula}$\rangle$ & ::= & $\langle$\textit{atomic formula}$\rangle$\\
& $|$ & $\neg\,\langle$\textit{formula}$\rangle$\\
& $|$ & $\langle$\textit{formula}$\rangle\land\langle$\textit{formula}$\rangle$\\
& $|$ & $\langle$\textit{formula}$\rangle\lor\langle$\textit{formula}$\rangle$\\
& $|$ & $\langle$\textit{formula}$\rangle\implies\langle$\textit{formula}$\rangle$\\
& $|$ & $\langle$\textit{formula}$\rangle\iff\langle$\textit{formula}$\rangle$\\
& $|$ & $\forall$ $\langle$\textit{variable}$\rangle\langle$\textit{formula}$\rangle$\\
& $|$ & $\exists$ $\langle$\textit{variable}$\rangle\langle$\textit{formula}$\rangle$\\
\end{tabular}
\end{center}
\end{definition}
\end{node}

\begin{node}[Semantics]\label{fol-0005}%
\begin{definition}\label{fol-0006}%
An \define{Interpretation} (or ``\textit{Model}\,'') $M$ specifies the
interpretation of function and predicate symbols, consisting of:
\begin{enumerate}
\item A nonempty set $D$ called the \define{Domain} of the interpretation;
\item An assignment taking each $n$-ary function symbol $f$ to a mapping
  $f_{M}\colon D^{n}\to D$;
\item An assignment taking each $n$-ary relation symbol $P$ to a
  relation $P_{M}\subset D^{n}$.
\end{enumerate}
\begin{node}[Remark: Nonempty domain]\label{fol-000A}\index{Logic!Free}\index{Free Logic}%
Note this definition precludes empty domains $D\neq\emptyset$. Classical
logic \emph{requires} nonempty domains of discourse. If we allow empty
domains, we end up with something called ``free logic''. Within free
logic, the tautology $(\forall x,P[x])\implies P[y]$ fails to hold.
More serious of a problem, for us, is that the existence of prenex
normal forms fail when empty domains are allowed.
\end{node}

\begin{node}[Remark: finite mechanization]\label{fol-000B}%
We will implement this in \SML\ using finite domains of discourse.
Pulgar\'{\i}n and Uribe-Zapata~\cite{pulgarin2023came} wrote a paper
trying to formalize a finitistic intuitionistic metatheory, which is
what Kunen~\cite{kunen2009foundations} advocates.

When $D$ is infinite (logicians allow $D$ to be uncountable), our
strategy we take in \zref{fol-0009} will fail and we cannot realize
\texttt{holds} as a \SML\ function.
\end{node}

\begin{node}[Remark: On semantics]\label{fol-000C}%
Other foundations do not concern themselves too much with
[model theoretic] semantics. In fact, the only other family of
foundations which has a notion of models appears to be higher-order
logic (HOL). This is because Type Theory focuses on more proof theoretic
concerns.

\begin{node}[Cynical perspective]\label{fol-000H}%
After thinking about this further, I think what is happening ``really''
(from a cynic's perspective) is that we have a theory of sets $S$
(presumably one which generalizes the \ZF\ axioms suitably), and then we
take our theory $\mathfrak{T}=(D,\mathfrak{R},\mathfrak{F})$ and form an ``interpretation''
$\iota$ which consists of an assignment of $D$ to a ``set'' $\iota(D)$,
and each relation $R\in\mathfrak{R}$ of arity $n$ to a ``set''
$\iota(R)\subset\iota(D)^{n}$, and each function $f\in\mathfrak{F}$ of
arity $k$ to the ``function'' $\iota(f)\colon\iota(D)^{k}\to\iota(D)$.
As a cynic, these are all working at the level of syntax.

If we stipulate this cynical perspective is ``true enough'', then
semantics of a theory (first-order or otherwise) in a type theory is a
similar assignment at the level of syntax. There are no ``real models'':
it's symbol manipulation all the way down. We just pick some foundations
for being ``real'', then assert ``mathematical objects exist if they can
be encoded in our foundations''.
\end{node}
\begin{node}\label{fol-000I}%
After thinking about this cynical perspective, I think it more or less
works provided the choice of foundations forms a topos (or could encode
a topos). This is why set theory works as a choice of foundations,
because $\Set$ forms a topos.

The overall cynical perspective (``Pick one theory as a foundations,
then interpret other theories in terms of our foundational theory'') is
not unique. It appears Farmer, Guttman, and
Javier~\cite{farmer1992little} discusses a similar perspective, calling
such a perspective ``small
theories''. Enderton~\cite[\S2.7]{enderton2001mathematical} is the only book I
have found discussing ``interpretation of theories in other theories''. John
Harrison~\cite[\S1.4]{harrison1996formalized} argues Bourbaki worked
\emph{contra} this perspective, using a ``big theory'' rather than a
constellation of ``small theories''.
\end{node}
\end{node}
\end{definition}

\begin{definition}\label{fol-0007}%
Let $M$ be an interpretation with domain $D$. A \define{Valuation} in
first-order logic is a mapping from the set of variables to $D$.
This is consistent with our usage of the term in propositional logic
\zref{prop-0004} which maps our set of propositional variables to the
set of truth values.
\end{definition}

\begin{definition}\label{fol-0008}%
We define the value of a term in a particular interpretation $M$ and
valuation $v$ by the mapping $\mathtt{termval}~M~v~t$ by induction on
the structure of $t$ as:
\begin{center}
\begin{tabular}{rcl}
$\mathtt{termval}~M~v~x$ & $=$ & $v(x)$\\
$\mathtt{termval}~M~v~(f(t_{1},\dots,t_{n}))$ & $=$ & $f_{M}(\mathtt{termval}~M~v~t_{1},\dots,\mathtt{termval}~M~v~t_{n})$\\
\end{tabular}
\end{center}
\end{definition}

\begin{definition}\label{fol-0009}%
The main question for semantics is whether a formula holds (i.e., has
its value be ``true'' $\verum$) in a particular interpretation $M$ and
valuation $v$ is defined inductively as the function $\mathtt{holds}~M~v~p$.
We use the notation
\begin{equation}
((x\mapsto c)v)(y) = \begin{cases}c & \mbox{if }x=y\\
v(y) & \mbox{otherwise}.
  \end{cases}
\end{equation}
Now $\mathtt{holds}~M~v~p$ may be given by the structure of the formula $p$ as:
\begin{center}
\begin{tabular}{rcp{0.45\linewidth}}
$\mathtt{holds}~M~v~\falsum$ & $=$ & false\\
$\mathtt{holds}~M~v~\verum$ & $=$ & true\\
$\mathtt{holds}~M~v~(R(t_{1},\dots,t_{n}))$ & $=$ & $R_{M}(\mathtt{termval}~M~v~t_{1},\dots,\mathtt{termval}~M~v~t_{n})$\\
$\mathtt{holds}~M~v~(\neg p)$ & $=$ & not $(\mathtt{holds}~M~v~p)$\\
$\mathtt{holds}~M~v~(p\land q)$ & $=$ & $(\mathtt{holds}~M~v~p)$ and $(\mathtt{holds}~M~v~q)$\\
$\mathtt{holds}~M~v~(p\lor q)$ & $=$ & $(\mathtt{holds}~M~v~p)$ or $(\mathtt{holds}~M~v~q)$\\
$\mathtt{holds}~M~v~(p\implies q)$ & $=$ & either not $(\mathtt{holds}~M~v~p)$ or $(\mathtt{holds}~M~v~q)$\\
$\mathtt{holds}~M~v~(p\iff q)$ & $=$ & $(\mathtt{holds}~M~v~p)=(\mathtt{holds}~M~v~q)$\\
$\mathtt{holds}~M~v~(\forall x.p)$ & $=$ & for all $c\in D$,
  $\mathtt{holds}~M~((x\mapsto c)v)~p$\\
$\mathtt{holds}~M~v~(\exists x.p)$ & $=$ & there is at least one $c\in D$ such that
  $\mathtt{holds}~M~((x\mapsto c)v)~p$\\
\end{tabular}
\end{center}
\end{definition}

\begin{definition}\label{fol-000D}%
We call a first-order formula \define{(Logically) Valid} if it holds in all
interpretations and all valuations. If $\varphi\iff\psi$ is logically
valid, then we say $\varphi$ and $\psi$ are \define{Logically Equivalent}.
Valid formulas are the first-order counterparts to
tautologies~\zref{prop-000L} in propositional logic.
\end{definition}

\begin{definition}\label{fol-000E}%
We say that an interpretation $M$ \define{Satisfies} a first-order
formula $\varphi$ (or simply that $\varphi$ \textit{holds in} $M$) if
\textbf{for all valuations} $v$ we have $\mathtt{holds}~M~v~\varphi$
be true.

We can extend this notion for a set of first-order formulas $S$, saying
an interpretation $M$ satisfies $S$ if for each $\varphi\in S$ we have
$M$ satisfy $\varphi$.
\end{definition}

\begin{definition}\label{fol-000F}%
We say a first-order formula $\varphi$ (or set of formulas) is
\define{Satisfiable} if there exists some interpretation which satisfies it.
That is, there exists an interpretation $M$ such that for all valuations
$v$ we have $\mathtt{holds}~M~v~\varphi$ evaluate to true.

The valuation plays a role only if there are free variables in $\varphi$.
\end{definition}
\end{node}

\begin{node}[References]\label{fol-000J}%
Emil Post~\cite{post1921introduction} is credited for inventing this
distinction between ``syntax'' and ``semantics'' in logic.
\end{node}
