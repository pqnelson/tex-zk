% 72
\chapter{Mizar}\label{chapter:mizar}

\begin{node}\label{mizar-0000}%
\Mizar\ is both a formal language and the name of a proof assistant
which uses the formal language as its input language. Unlike other proof
assistants, \Mizar\ focuses on readability and reflecting how working
mathematicians write mathematical proofs ``in the wild''.

Broadly speaking, the input language for any proof assistant (and,
indeed, any formal language offered to formalize mathematics) can be
partitioned into two interconnected sub-languages: the first for definitions,
the second for theorems and proofs. These share the same language for
encoding formulas (``propositions'') and terms.
\end{node}

\begin{node}[Foundations]\label{mizar-0001}%
The foundations for \Mizar\ is purely pragmatic, reflecting working
mathematics. Specifically, it uses a slightly modified version of
first-order logic (jokingly referred to as $(1+\varepsilon)$-order logic)
using free second-order variables for schemes, using a soft-type system
atop untyped logic, and Tarski--Grothendieck set theory~\zref{sec:set:tarski-grothendieck}.

I am told that there were early experiments using \MK\ set theory, but
Andrzej Trybulec and friends ran into difficulties formalizing category
theory. This led the \Mizar\ group to strengthen their foundations to
Tarski--Grothendieck set theory.
\end{node}

\begin{node}[Formulas]\label{mizar-0003}%
Logical formulas can be translated in a fairly straightforward manner,
using the following recursive translation process:
\begin{center}
\begin{tabular}{rcl}
$\neg\alpha$ & $\mapsto$ & \texttt{not} $\alpha$\\
$\alpha\land\beta$ & $\mapsto$ & $\alpha$ \texttt{\&} $\beta$\\
$\alpha\lor\beta$ & $\mapsto$ & $\alpha$ \texttt{or} $\beta$\\
$\alpha\implies\beta$ & $\mapsto$ & $\alpha$ \texttt{implies} $\beta$\\
$\alpha\iff\beta$ & $\mapsto$ & $\alpha$ \texttt{iff} $\beta$\\
$(\exists x)\alpha$ & $\mapsto$ & \texttt{ex x st} $\alpha$\\
$(\forall x)\alpha$ & $\mapsto$ & \texttt{forall x holds} $\alpha$\\
$(\forall x)(\alpha\implies\beta)$ & $\mapsto$ & \texttt{forall x st} $\alpha$ \texttt{holds} $\beta$\\
\end{tabular}
\end{center}
Atomic formulas are built either applying predicates to terms, or
something of the form ``$\langle$\textit{term}$\rangle$ \texttt{is}
$\langle$\textit{type or adjective}$\rangle$''.
\end{node}

\section{Proof Steps}

\begin{node}\label{mizar-0002}%
The proof language, when using informal notation for terms and formulas,
may stand alone as its own formal language. Wiedijk~\cite{wiedijk2000vernacular}
presents this as \emph{a} mathematical vernacular.
\end{node}

\begin{node}[Let or $\forall$-introduction]\label{mizar-0004}%
When trying to prove the claim $(\forall x)P[x]$, we begin with the step
``\texttt{let} $y$;'' the proof obligation transforms to
$P[y]$. Schematically:
\begin{verse}
$\langle$\emph{proof of \/}$(\forall x)P[x]\rangle \equiv$\\*
\qquad $\langle$\emph{preliminary steps\/}$\rangle$\\*
\qquad \texttt{let }$y$\texttt{;}\\*
\qquad $\langle$\emph{proof of \/}$P[y]\rangle$\\*
\end{verse}
Usually the name of the variable introduced in the \texttt{let} step is
the same as the name of the variable bound to the $\forall$ quantifier.
This step corresponds to the natural deduction rule of
$\forall$-introduction:
\begin{equation*}
\frac{\Gamma,y\vdash P[y]}{\Gamma\vdash(\forall x)P[x]}\mathtt{let\ }y
\end{equation*}
As usual with natural deduction, we read this rule from bottom to top.
\end{node}

\begin{node}[Assume]\label{mizar-0005}%
The \texttt{assume} step can be used in three ways.

\begin{node}[Implication-introduction]\label{mizar-0006}%
When we are trying to prove the claim $A\implies B$, then we can begin
by ``\texttt{assume} $A$;'' which transforms the goal into just
$B$. This means:
\begin{verse}
$\langle$\emph{proof of \/}$A\to B\rangle \equiv$\\*
\qquad $\langle$\emph{preliminary steps\/}$\rangle$\\*
\qquad \texttt{assume }$A$\texttt{;}\\*
\qquad $\langle$\emph{proof of \/}$B\rangle$\\*
\end{verse}
This corresponds to the natural deduction $\Longrightarrow$-introduction
\[\frac{\Gamma,A\vdash B}{\Gamma\vdash A\implies B}\mathtt{assume}~A.\]
\end{node}

\begin{node}[Negation-introduction]\label{mizar-0007}%
Since $\neg A$ is the same as $A\implies\falsum$, we can use this proof
step to prove $\neg A$:
\begin{verse}
$\langle$\emph{proof of \/}$\neg A\rangle \equiv$\\*
\qquad $\langle$\emph{preliminary steps\/}$\rangle$\\*
\qquad \texttt{assume }$A$\texttt{;}\\*
\qquad $\langle$\emph{proof of \/}$\bot\rangle$\\*
\end{verse}
This is precisely $\neg$-introduction
\[\frac{\Gamma,A\vdash\falsum}{\Gamma\vdash\neg A}\mathtt{assume}~A.\]
\end{node}

\begin{node}[Reductio ad absurdum]\label{mizar-0008}%
If stipulating $\neg A$ yields a contradiction, then we can infer $A$ to
be true. This is precisely the \emph{reductio ad absurdum} gambit:
\begin{verse}
$\langle$\emph{proof of \/}$A\rangle \equiv$\\*
\qquad $\langle$\emph{preliminary steps\/}$\rangle$\\*
\qquad \texttt{assume }$\neg A$\texttt{;}\\*
\qquad $\langle$\emph{proof of \/}$\falsum\rangle$\\*
\end{verse}
The natural deduction rule for this is
\[\frac{\Gamma,\neg A\vdash\falsum}{\Gamma\vdash A}\mathtt{assume}~\neg A.\]
\end{node}
\end{node}

\begin{node}[Thus]\label{mizar-0009}%
Arguably the ``converse'' to the \texttt{assume} step (which eliminates
a hypothesis from the statement to be proven) is the \texttt{thus} step
(which eliminates a conclusion to the statement to be proven). When the
proof obligation is $A\land B$, then ``\texttt{thus} $A$'' will
transform the goal into $B$:

\begin{verse}
$\langle$\emph{proof of \/}$A\wedge B\rangle \equiv$\\*
\qquad $\langle$\emph{preliminary steps\/}$\rangle$\\*
\qquad \texttt{thus }$A$\texttt{ by }$\ldots$\texttt{;}\\*
\qquad $\langle$\emph{proof of \/}$B\rangle$\\*
\end{verse}
In natural deduction, this is the $\land$-introduction rule:
\[\frac{\Gamma\vdash A,\qquad\Gamma\vdash B}{\Gamma\vdash A\land B}\mathtt{thus}~A\]
where the ``top left'' corner is the justification for the step.

\textbf{Every proof must end with a ``thus'' statement.}

\begin{node}[Hence = then + thus]\label{mizar-000E}%
When we prefix ``\texttt{thus}'' with ``\texttt{then}'' (using the
previous step as justification), \Mizar\ provides the syntactic sugared
proof step ``\texttt{hence}'' with optional references.
\end{node}
\end{node}

\begin{node}[Per cases $\lor$-elimination]\label{mizar-000A}%
Suppose we have $A_{1}\lor\dots\lor A_{n}$ proven already, and we wish
to use this to prove $B$. We just need to prove $A_{1}\implies B$,
\dots, $A_{n}\implies B$. Each of these cases are then a subproof, where
we begin by ``\texttt{suppose} $A_{j}$;'' and work a proof of $B$ from
this hypothesis $A_{j}$. Schematically, this looks like
\begin{verse}
$\langle$\emph{proof of \/}$B\rangle \equiv$\\*
\qquad $\langle$\emph{preliminary steps\/}$\rangle$\\*
\qquad \texttt{per cases by }$\ldots$\texttt{;}\\*
\qquad \texttt{suppose }$A_{1}$\texttt{;}\\*
\qquad\quad $\langle$\emph{proof of \/}$B$ \emph{from \/}$A_{1}\rangle$\\*
\qquad $\ldots$\\*
\qquad \texttt{suppose }$A_{n}$\texttt{;}\\*
\qquad\quad $\langle$\emph{proof of \/}$B$ \emph{from \/}$A_{n}\rangle$\\*
\end{verse}
The justification ``\texttt{by}\dots'' refers to the place where we have
proven $A_{1}\lor\dots\lor A_{n}$. When $A_{1}\lor\dots\lor A_{n}$ is a
tautology (like $A_{1}\lor\neg A_{1}$), no justification is necessary.

This corresponds to the natural deduction rule of $\lor$-elimination:
\[\frac{\Gamma\vdash A_{1}\lor\dots\lor A_{n},\quad\Gamma,A_{1}\vdash B,\quad\dots\quad\Gamma,A_{n}\vdash B}{\Gamma\vdash B}\mathtt{per\ cases}\]
As usual, the left-most of the top subgoals justifies the proof step.
\end{node}

\begin{node}[Consider or $\exists$-elimination]\label{mizar-000B}%
We can apply an existentially quantified theorem $(\exists x) P[x]$ to
reason about such an $x$:
\begin{verse}
$\langle$\emph{proof of \/}$A\rangle \equiv$\\*
\qquad $\langle$\emph{preliminary steps\/}$\rangle$\\*
\qquad \texttt{consider }$x$\texttt{ such that }$P[x]$\texttt{ by }$\ldots$\texttt{;}\\*
\qquad $\langle$\emph{proof of \/}$A\rangle$\\*
\end{verse}
In natural deduction, this corresponds to $\exists$-elimination:
\[\frac{\Gamma\vdash(\exists x)P[x],\qquad\Gamma,x,P[x]\vdash A}{\Gamma\vdash A} \mathtt{consider}~x~\mathtt{such\ that}~P[x]\]
The top-left subgoal justifies this proof step.

\begin{node}[Given = assume + consider]\label{mizar-000D}%
There is syntactic sugar in \Mizar\ to combine the steps
\begin{verse}
\texttt{assume }$(\exists x)P[x]$\texttt{;}\\*
\texttt{then consider }$x$\texttt{ such that }$P[x]$\texttt{;}\\*
\end{verse}
into one equivalent single step
\begin{verse}
\texttt{given }$x$\texttt{ such that }$P[x]$\texttt{;}\\*
\end{verse}
This is pure syntactic sugar.
\end{node}
\end{node}

\begin{node}[Take, or $\exists$-introduction]\label{mizar-000C}%
If we wish to prove an existential statement, we need to
``\texttt{take}'' a term and prove it satisfies the predicate:
\begin{verse}
$\langle$\emph{proof of \/}$(\exists x)P[x]\rangle \equiv$\\*
\qquad $\langle$\emph{preliminary steps\/}$\rangle$\\*
\qquad \texttt{take }$a$\texttt{;}\\*
\qquad $\langle$\emph{proof of \/}$P[a]\rangle$\\*
\end{verse}
This is precisely natural deduction $\exists$-introduction rule:
\[\frac{\Gamma\vdash P[a]}{\Gamma\vdash(\exists x)P[x]}\mathtt{take}~a\]
No justification is needed for this step.
\end{node}

\begin{node}[Set]\label{mizar-000H}%
We can introduce new variables which abbreviate some complex terms by
writing ``\texttt{set} $A$ \texttt{=} $\langle$\textit{term\/}$\rangle$''.
\end{node}

\begin{node}[Reconsider as coercion]\label{mizar-000I}%
We can force \Mizar\ to treat a given term as if it was one stated,
i.e., we can coerce a term to be a specific type. This requires
justification. Schematically, this step looks like:
\begin{verse}
  \texttt{reconsider} $t$ \texttt{=} $\langle$\textit{term\/}$\rangle$ \texttt{as} $\langle$\textit{type\/}$\rangle$ \texttt{by}\dots\texttt{;}\\*
\end{verse}
Here $t$ is usually a fresh identifier.
\end{node}

\begin{node}[Bi-conditional proof skeleton]\label{mizar-000F}%
If we wish to prove the formula $A\iff B$, the Mizar proof skeleton
looks like
\begin{verse}
$\langle$\emph{proof of \/}$A\iff B\rangle \equiv$\\*
\qquad $\langle$\emph{preliminary steps\/}$\rangle$\\*
\qquad \texttt{thus }$A\implies B$\texttt{;}\\*
\qquad $\langle$\emph{preliminary steps\/}$\rangle$\\*
\qquad \texttt{thus }$B\implies A$\texttt{;}\\*
\end{verse}
This is quite straightforward after some reflection, since the proof
obligation is $(A\implies B)\land(B\implies A)$.
\end{node}

\begin{node}[Disjunction proof skeleton]\label{mizar-000G}%
When proving the claim $A\lor B$, we begin by stipulating $\neg A$,
then proving $B$:
\begin{verse}
$\langle$\emph{proof of \/}$A\lor B\rangle \equiv$\\*
\qquad $\langle$\emph{preliminary steps\/}$\rangle$\\*
\qquad \texttt{assume }$\neg A$\texttt{;}\\*
\qquad $\langle$\emph{proof of \/}$B\rangle$\\*
\end{verse}
An important caveat for \Mizar\ is that although logically $A\lor B$
is equivalent to $B\lor A$, this is not true for \Mizar\ since the
normal form for $A\lor B$ is different than that for $B\lor A$. So you'd
have to tell \Mizar\ that $B\lor A\implies A\lor B$, then in a subproof
demonstrate $B\lor A$, and combine them together to obtain $A\lor B$.
\end{node}

\section{Definitions}

\begin{node}\label{mizar-000J}%
All new notions in \Mizar\ have the following possible components, as
they are found within a ``\texttt{definition}\dots\texttt{end}'' block:
\begin{enumerate}
\item First we write down the arguments (if any) along with their
  corresponding types, called the \textit{loci}. These are written down
  using the ``\texttt{let}'' keyword.
\item Next we writen down any assumptions for the definition using the
  ``\texttt{assume}'' or ``\texttt{given}'' keywords.
\item Then we have the main part of the definition, which is discussed
  in this section.
\item Then we have correctness conditions (if needed) along with their proofs.
\item Then we have [optionally] properties for the notion (like
  commutativity for an operator, reflexivity for a predicate, etc.).
\end{enumerate}
These are the basic components for a definition.

\begin{node}[Loci accumulate in a block]\label{mizar-000K}%
We can define multiple notions within the same block. In this case, the
loci ``accumulate'' from the start of the definition block all the way
to the current point.
\end{node}

\begin{node}[Notions available after block closes]\label{mizar-000L}%
Newly defined notions become ``available'' only \emph{after} the block
is closed the the ``\texttt{end;}'' keyword. If we try to use a defined
notion within the same block defining it, then \Mizar\ reports an error
marked with a \texttt{*102} flag (for unknown predicates) or similar flags.
\end{node}

\begin{node}[Loci must be used]\label{mizar-000M}%
None of the arguments in a definition may be omitted in the declaration
part of a definitional block. Arguments may be ``hidden'', for example:
\begin{mizar}
definition
  let E be set, A be Subset of E;
  func A' -> Subset of E equals
  E \ A;
  :: ...
end;
\end{mizar}
Here ``\texttt{E}'' is a hidden argument, and ``\texttt{A}'' is a
visible argument. We could have two visible arguments in our example,
writing something like
\begin{mizar}
definition
  let E be set, A be Subset of E;
  func '(E,A) -> Subset of E equals
  E \ A;
  :: ...
end;
\end{mizar}
This is also fine, but less frequently done in \Mizar.

But this is the key property of loci: \textbf{they must either be
  a visible argument of the defined notion or be used as a parameter in another locus}.
\end{node}

\begin{node}[No repeated arguments]\label{mizar-000N}%
We cannot repeat arguments in definitions, for example
\begin{mizar}
definition
  let E be set, A be Subset of E;
  func '(E,A,A) -> Subset of E equals
::>          *141
  E \ A;
  :: ...
end;
::> 141: Locus repeated
\end{mizar}
Mizar flags a repeated locus appearing in the definition.
\end{node}

\begin{definition}
The \define{format} of a definition is an ordered triple consisting of
the identifier of the newly defined notion, the number of arguments to
its left, and the number of arguments to its right.
\end{definition}

\begin{definition}
The \define{pattern} of a definition is an ordered quadruple consisting of
\begin{enumerate}
\item the identifier of the notion,
\item the tuple of types for the arguments to its left,
\item the tuple of types for the arguments to its right, and
\item the type of the defined notion (if applicable).
\end{enumerate}
\end{definition}
\end{node}

\begin{node}[Predicates]\label{mizar-000O}%
We have a list of loci variables, followed by the the ``\texttt{pred}''
keyword followed by the predicate's pattern. The arity is specified with
a certain number of loci symbols on both sides of a predicate
symbol. For example,
\begin{mizar}
definition
  let f,g be Function;
  let A be set;
  pred f,g equal_outside A means
:Def:
  f|(dom f \ A) = g|(dom g \ A);
end;
\end{mizar}
This formalizes the formula $f|_{\dom(f)\setminus A}=g|_{\dom(g)\setminus A}$
in a predicate \verb|equal_outside|.

\begin{node}[Definitional theorem]\label{mizar-000P}%
When referring to a definition, \Mizar\ is really making a reference to
its corresponding ``definitional theorem'' implicitly generated by the
system. This looks like for the given example:
\begin{mizar}
for f,g being Function, A being set holds
f,g equal_outside A iff f|(dom f \ A) = g|(dom g \ A);
\end{mizar}
\end{node}

\begin{node}[Piecewise definitions]\label{mizar-000Q}%
We can define a notion by cases depending on the values of the loci
arguments. For example,
\begin{mizar}
definition
  let x,y be ext-real number;
  pred x <= y means
  ex p,q being Element of REAL st p = x & q = y & p <= q
    if x in REAL & y in REAL
  otherwise x = -infty or y = +infty;
end;
\end{mizar}
In this case, the definitional theorem has the form
\begin{mizar}
for x,y being ext-real number holds
  (x in REAL & y in REAL implies
    (x <= y iff ex p,q being Element of REAL
                st p = x & q = y & p <= q)) &
  (not x in REAL or not y in REAL implies
    (x <= y iff x = -infty or y = +infty));
\end{mizar}
\end{node}
\end{node}

\begin{node}[Attributes]\label{mizar-000R}%
Attributes are defined with an ``\texttt{attr}'' keyword, followed by
its \emph{subject}, then the ``\texttt{is}'' keyword, the attribute's
symbol, the ``\texttt{means}'' keyword, and then its
definiens. \Mizar\ requires the last locus declared by its subject. For
example,
\begin{mizar}
definition
  let E be set, A be Subset of E;
  attr A is proper means :Def:
  A <> E;
end;
\end{mizar}
This corresponds to the definitional theorem:
\begin{mizar}
for E being set, A being Subset of E holds
A is proper iff A <> E;
\end{mizar}
Attributes can have visible arguments.

\begin{node}[Idiom]\label{mizar-000V}%
I've seen it be idiomatic \Mizar\ to name the subject ``\texttt{IT}'',
in analogy to the ``\texttt{it}'' keyword used in mode definitions.
\end{node}
\end{node}

\begin{node}[Modes]\label{mizar-000S}%
Types are called ``modes'' in \Mizar. There are two kinds of mode
definitions in \Mizar:
\begin{enumerate}
\item expandable modes: this is an abbreviation of the form
  ``$\langle$\textit{new type name}$\rangle$ \texttt{is} $\langle$\textit{adjectives}$\rangle$
  $\langle$\textit{type}$\rangle$''; this is an ``abbreviation''.
\item non-expandable modes: this introduces a new type in terms of an
  existing type satisfying some condition.
\end{enumerate}
Any loci which appear must be in the suffix of the mode definition
following ``\texttt{of}'' keyword.

\begin{node}[Correctness conditions]\label{mizar-000X}%
We need to prove the existence of at least one object with the specific type.
This is for non-expandable modes only, because expandable modes
leverages the existence as registered by registration blocks (for the
adjectives) and the existence of pre-existing types. The correctness
condition has a proof skeleton of the form:
\begin{mizar}
definition
  let x1 be T1, ..., xN be TN;
  mode T -> MotherMode means P[x1, ..., xN, it];
  existence
  proof
    thus ex t being MotherMode st P[x1, ..., xN, t];
  end;
end;
\end{mizar}
This is standard classical logic which is glossed over in undergraduate education.
\end{node}

\begin{node}[Widening]\label{mizar-000Y}%
The ``mother type'' in a mode definition is used to specify the direct
predecessor of the defined mode in the tree of \Mizar\ types. We say the
type constructed with the new mode \define{widens} to its mother
type. This also takes into account the adjectives that come with the
ancestor types.

The built-in \texttt{object} is the widest type, the \texttt{set} type
is the next widest, and the zeroeth axiom identifies these two types as
the same.
\end{node}

\begin{node}[Example modes]\label{mizar-000W}%
\begin{node}\label{mizar-000T}%
An example of an expandable mode is a function:
\begin{mizar}
definition
  mode Function is Function-like Relation;
end;
\end{mizar}
\end{node}

\begin{node}[Non-expandable mode]\label{mizar-000U}%
Another example of a mode, one which is non-expandable, is:
\begin{mizar}
definition
  let X be set;
  mode a_partition of X -> Subset-Family of X means :Def:
  union it = X
  & for A being Subset of X st A in it holds A <> {}
  & for B being Subset of X st B in it holds A = B or A misses B;
end;
\end{mizar}
Observe that ``\texttt{it}'' is a keyword which refers to an object of
the newly defined type.
\end{node}
\end{node}

\begin{node}[Best practices]\label{mizar-0019}%
Following the advice of Korni\l{}owicz~\cite{Korniowicz2009-KORHTD},
the type to the right of \texttt{->} in non-expandable mode definitions
should not include adjectives, because we can register adjectives later
in a conditional registration (something along the lines of ``\texttt{-> adjective Type}'').
In general, we should try to make the types ``as weak as possible'', and
use registration as much as possible.
\end{node}
\end{node}

\begin{node}[Functors]\label{mizar-000Z}%
Terms are constructed using the ``\texttt{func}'' keyword. After the
loci declarations and optional assumptions, we have the \texttt{func}
keyword, arguments to its left, the identifier being defined, arguments
to its right, then the ``\texttt{->}'' keyword, then the type of the
term, followed by either ``\texttt{equals} $\langle$\textit{term\/}$\rangle$''
or ``\texttt{means} $\langle$\textit{formula\/}$\rangle$''.

\begin{node}[Correctness]\label{mizar-0010}%
The correctness condition varies depending on how we're defining the
term:
\begin{enumerate}
\item for ``\texttt{equals} $\langle$\textit{term\/}$\rangle$'', we need
  to prove the right-hand side of this equality has the correct type;
\item ``\texttt{means} $\langle$\textit{formula\/}$\rangle$'' requires
  proving the existence of the term and the uniqueness of the term.
\end{enumerate}
The proof skeleton for these look like:
\begin{mizar}
definition
  let x1 be T1, ..., xN be TN;
  func t(x1, ..., xN) -> T means P[x1, ..., xN, it];
  existence
  proof
    thus ex tm being T st P[x1, ..., xN, tm];
  end;
  uniqueness
  proof
    thus for t1 being T st P[x1, ..., xN, t1]
         for t2 being T st P[x1, ..., xN, t2]
         holds t1 = t2;
  end;
end;
definition
  let x1 be T1, ..., xN be TN;
  func t(x1, ..., xN) -> T equals f(x1, ..., xN);
  coherence
  proof
    thus f(x1, ..., xN) is T;
  end;
end;
\end{mizar}
\end{node}
\begin{node}[Definitions environ]\label{mizar-0011}%
When we include an article in the \texttt{definitions} environment
directive, \Mizar\ will automatically perform $\delta$-expansion on
definitions using the ``\texttt{equals}'' keyword.
\end{node}
\end{node}

\begin{node}[Structures]\label{mizar-0012}%
A structure in \Mizar\ allows us to formalize a logical signature (the
primitive operators for a gadget). A structure definition contains a
list of \emph{selectors} to denote its fields, characterized by their
name and type. For example,
\begin{mizar}
definition
  struct multMagma
  (# carrier -> set,
     multF -> BinOp of the carrier #);
end;
\end{mizar}
Here \texttt{multMagma} is the name of the structure, it has a field
called its \texttt{carrier} and a binary operator on it called its
\texttt{multF}. This structure can be used to define a group, upper
semilattices, and lower semilattices\dots and any notion based on a set
and a binary operator on it, really.

But this is just the ``stuff and structure'' underlying a group. We need
to impose the group axioms on it to obtain a group. This is done using
attributes to formalize the axioms, then a mode to formalize the type of
a group.

\begin{node}[Inheritance]\label{mizar-0014}%
We can form a family of structures as subtypes of other structures using
an inheritance mechanism. For many structures, the \texttt{1-sorted}
structure is the common ancestor (when there is an obvious underlying
set, \texttt{the carrier of it}). For an example, here's how we would
define a topological group structure:
\begin{mizar}
definition
  struct (1-sorted) TopStruct
  (# carrier -> set,
     topology -> Subset-Family of the carrier #);
end;
definition
  struct (multMagma, TopStruct) TopGrStr
  (# carrier -> set,
     multF -> BinOp of the carrier,
     topology -> Subset-Family of the carrier #);
end;
\end{mizar}
The advantage of this approach is that all results about topological
spaces hold for topological groups (as do all results about groups).
\end{node}

\begin{node}[Idiom]\label{mizar-0013}%
It's idiomatic \Mizar\ to define a new \texttt{Gadget} by defining a
\texttt{struct GadgetStr} with appropriate fields, then an attribute
\texttt{Gadget-like}, and finally a new type \texttt{mode Gadget is Gadget-like GadgetStr}.
\end{node}

\begin{node}[Parametrized by terms]\label{mizar-0015}%
We can parametrize a structure using other sets or types. For example,
we have a vector space \emph{over a field}, which can be formalized
using the structure definition:
\begin{mizar}
definition
  let F be 1-sorted;
  struct (addLoopStr) VectSpStr over F
  (# carrier -> set,
     addF -> BinOp of the carrier,
     ZeroF -> Element of the carrier,
     lmult -> Function of [:the carrier of F,the carrier:],
     the carrier #);
end;
\end{mizar}
Here we leave the field $F$ ``much more generally'', as any \texttt{1-sorted}
object. \emph{It's usually a good idea to be more accepting in these parameters}
than a strict ``literalist interpretation'' of the mathematical text.
\end{node}

\begin{node}[Aggregates]\label{mizar-0016}%
We can construct an ``instance'' of a structure using what \Mizar\ calls
\emph{aggregates}. For example, \verb|multMagma(#INT,addint#)|
constructs an instance of a multiplicative magma where \texttt{INT} is
the set of integers, and \texttt{addint} is the binary operator of
adding two integers together to obtain a third integer. The only thing
we need to check is that the terms's types ``match'' the types of the
respective fields (for our example, \texttt{INT} must be a set, and
\texttt{addint} must be of type \texttt{BinOp of INT}).
\end{node}

\begin{node}[Strict adjective]\label{mizar-0017}%
There is a secret adjective for all structure types, ``\texttt{strict}''.
It means an object of a structure type contains exactly the
fields enumerated in its definition, nothing more.
\end{node}

\begin{node}[Forgetful functors]\label{mizar-0018}%
This is a horrible pun. When a structure type has ancestors, we can
access the underlying type of one of its ancestors. For example, if $G$
is an object whose type is \texttt{TopGroupStr}, then we can access
\texttt{the multMagma of G} to obtain
\texttt{multMagma}\verb|(#|\texttt{the carrier of G, the multF of G}\verb|#)|.
\end{node}

\begin{node}[Remarks]\label{mizar-001C}%

\begin{node}[Alternatives using finite functions]\label{mizar-001A}%
Note that Lee and Rudnicki~\cite{Lee2007alternative} discuss using a
finite map in the object language (i.e., literally a \texttt{Function}
whose domain is a subset of \texttt{Nat} and fields are just numbers) as
an alternative to the structure types we've discussed here.
\end{node}

\begin{node}[Types are almost always proper classes]\label{mizar-001B}%
By the axiom of the limitation of size~\zref{set-0004} for a set, we see
that types in \Mizar\ are almost always proper classes. For structure
types, this is clearly obvious for descendents of \texttt{1-sorted}
structures, since we can form at least one instance of a structure for
any given set.
\end{node}
\end{node} % end of Remarks

\end{node}

\section{Notation and Redefinitions}

\begin{node}[Synonyms and Antonyms]\label{mizar-001D}%
We can introduce new terms, predicates, etc., as synonyms or antonyms of
existing notions. These appear in \texttt{notation} blocks.

\begin{node}[Examples]\label{mizar-001E}%

\begin{node}\label{mizar-001F}%
  The predicate $x\notin y$ is the antonym for $x\in y$, which appears in
  \verb|XBOOLE_0|:
\begin{mizar}
notation
  let x be object,y be set;
  antonym x nin y for x in y;
end;
\end{mizar}
\end{node}
\begin{node}\label{mizar-001G}%
The predicate $x\nsubset y$ is the antonym for $x\subset y$, which
appears also in \verb|XBOOLE_0|:
\begin{mizar}
notation
  let x be object,y be set;
  antonym x nin y for x in y;
end;
\end{mizar}
\end{node}
\begin{node}\label{mizar-001H}%
The predicates $b\geq a$, $a > b$, $b < a$ may be written as synonyms
and antonyms of $a\leq b$, as done in \verb|XXREAL_0|:
\begin{mizar}
notation
  let a,b be ExtReal;
  synonym b >= a for a <= b;
  antonym b < a for a <= b;
  antonym a > b for a <= b;
end;
\end{mizar}
\end{node}
\end{node} % end examples

\begin{node}[Loci types can be more specific]\label{mizar-001I}%
The loci for the synonym (or antonym) \emph{can} have more specific
types than the original definition.
\end{node}

\begin{node}[Loci cannot be fixed constant]\label{mizar-001J}%
We cannot fix a constant value for any of the loci variables. We can't
swap a loci (say, $A$) with a constant (say, \texttt{NAT}). This is not
allowed. If you want to do this, you should consider defining a new term
(or predicate, or whatever).
\end{node}

\begin{node}[Synonyms can hide arguments]\label{mizar-001K}%
We can ``hide'' the number of visible arguments (or, if for some reason
you'd like, \emph{increase} the number of visible arguments) using
synonyms. For example:
\begin{mizar}
definition
  let A be set, B be Subset of A;
  pred pred_symbol_1 A,B means not contradiction;
end;

notation
  let A be set, B be Subset of A;
  synonym pred_symbol_1 B for pred_symbol_1 A,B;
end;
\end{mizar}
\end{node}

\begin{node}[Loci cannot be repeated]\label{mizar-001L}%
Like in definitions, we cannot repeat loci, or have unused loci.
\end{node}

\begin{node}[Only non-expandable modes can have synonyms]\label{mizar-001M}%
Only the non-expandable modes (types defined using ``\texttt{means}'')
can have synonyms. So if we tried to do something like:
\begin{mizar}
definition
  let X;
  mode Subset of X is Element of bool X;
end;

notation
  let X be set;
  synonym mode_symbol of X for Subset of X;
::>                                 *134
end;
::> 134: Cannot redefine expandable mode
\end{mizar}
\Mizar\ is unhappy because ``\texttt{Subset of X}'' is defined as an
expandable mode (as we have reproduced its definition above).
\end{node}

\begin{node}\label{mizar-001N}%
An important feature for \Mizar\ synonyms is that they do not inherit
the type of a redefined original constructor. (Say what?) Well,
\Mizar\ allows redefinitions~\zref{mizar-001O}\dots for a variety of reasons. This is
``flat'', in the sense that when considering which redefinition to apply
(and if none apply, just return to the original) we use the best fit
using types. But it's not a hierarchy of attempts, it's just ``one deep''.

So if we try to have a synonym for a type that is redefined, then the
synonym will match \emph{the initial definition} and not the redefinition.
\end{node}
\end{node}

\begin{node}[Redefinitions]\label{mizar-001O}%
Naumorwicz and Byli{\'n}ski~\cite{naumowicz2004improving}, Korni\l{}owicz~\cite{Korniowicz2016registrations} note there are three
reasons to consider redefinitions in \Mizar:
\begin{enumerate}
\item to specialize result types of defined functors and modes;
\item to change the ``right-hand side'' [the ``definientia''] of a
  defined notion (predicates, attributes, functors, and modes);
\item to declare properties of particular constructors.
\end{enumerate}
Let's look at examples where this happens.
\begin{node}[Specializing the result type]\label{mizar-001P}%
For example, if we have defined $f^{-1}$ the inverse function of $f$ as
\verb|f"| in Mizar by
\begin{mizar}
definition
  let f be Function;
  assume f is one-to-one;
  func f" -> Function equals
  f~;
end;
\end{mizar}
Then for permutations, we'd want to instead make sure that $f^{-1}$ is a
permutation, not just ``some function''. We'd do this by changing the
``return type'' of \verb|f"|:
\begin{mizar}
definition
  let X be set;
  let f be Permutation of X;
  redefine func f" -> Permutation of X;
end;
\end{mizar}
This is probably one of the most common uses of redefinitions I've seen.

\begin{node}\label{mizar-001S}%
Note that we cannot ``broaden'' the return type. If we tried to redefine
\verb|f"| to be, say, a \texttt{set} instead, then \Mizar\ would flag
this as an invalid specification error (\texttt{*117}).
\end{node}
\begin{node}[Best practices]\label{mizar-001T}%
Also note that we should try to prefer using attributes and
registrations to change the type of a definition, rather than
redefinitions. Registering an adjective is elegant. Redefinitions are
brute-force kludges. See Korni\l{}owicz~\cite{Korniowicz2016registrations}
and Naumowicz and Byli{\'n}ski~\cite{naumowicz2004improving} for more on
redefinitions and why we should prefer registrations.
\end{node}
\end{node}

\begin{node}[Changing the meaning]\label{mizar-001Q}%
For example, the builtin definition of equality is axiomatized in the
extensionality axiom:
\begin{mizar}
theorem :: TARSKI:2 :: Extensionality
  (for x being object holds x in X iff x in Y) implies X = Y;
\end{mizar}
That's all well and fine, but when proving two relations are equal, it's
more useful to use a different definition (which is equivalent ``under
the hood''). For example:
\begin{mizar}
definition
  let P,R;
  redefine pred P = R means
:: RELAT_1:def 2
  for a,b being set holds [a,b] in P iff [a,b] in R;
end;
\end{mizar}
Similarly, for functions, equality is more easily reasoned about if the
domains are equal and their values are equal point-wise:
\begin{mizar}
definition
  let f,g;
  redefine pred f = g means
:: FUNCT_1:def 11
  dom f = dom g
  & for x being set st x in dom f holds f.x = g.x;
end;
\end{mizar}
And so on.
\end{node}

\begin{node}[Declaring properties]\label{mizar-001R}%
This is the ``preferred usage'' for redefinitions. In \verb|GROUP_6|, we
can add the property that \verb|are_isomorphic| is symmetric in its
arguments:
\begin{mizar}
definition
  let G,H be Group;
  redefine pred G,H are_isomorphic;
  symmetry; :: this is the property
end;
\end{mizar}
It's unexciting to witness, because it looks like nothing is being
redefined. It should better be thought of as an ``addendum'' rather than
a ``redefinition''.
\end{node}
\end{node}

\section{Registrations}

\begin{node}\label{mizar-001U}%
Mizar's registration system is really a spectacular bit of ingenuity. 
For example, it allows us to \emph{dynamically} add more adjectives to a
term without requiring a redefinition.
Schwarzweller~\cite{schwarzweller2007mizar} is the best reference on the
subject, specifically how \Mizar\ uses registrations to make some
deductions automatic.
\end{node}

\begin{node}[Existential registration]\label{mizar-001V}%
When we want to use an attribute as an adjective, we must tell
\Mizar\ that there exists at least one object with the specific type and
attribute. For example,
\begin{mizar}
registration
  cluster finite non empty set;
  existence
  proof
    thus ex x being set st x is finite & x is non empty;
  end;
end;
\end{mizar}
My intuition reflexively thinks of existential registrations when
discussing the registration system.
\end{node}

\begin{node}[Conditional registration]\label{mizar-001W}%
When we want one attribute to automatically imply another, this is the
conditional registration system's job. For example, every empty set is
finite, this is registered as:
\begin{mizar}
registration
  cluster empty -> finite set;
  coherence
  proof
    thus for x being set st x is empty holds x is finite;
  end;
end;
\end{mizar}

\begin{node}[Blank antecedents]\label{mizar-001X}%
We can permit the ``antecedent'' to this registration to be blank, in
which case this records adjectives that will always be associated to the
given type. For example:
\begin{mizar}
registration
  let X be finite set;
  cluster -> finite Subset of X;
  coherence
  proof
    thus for A being Subset of X holds A is finite;
  end;
end;
\end{mizar}
This enables \Mizar\ to automatically make note of the fact that all
subsets of finite sets are finite.
\end{node}

\begin{node}[Cooperating with existential registrations]\label{mizar-001Y}%
When we have conditional registrations occuring \emph{before} an
existential registration, then the existential registration will
automatically ``round up'' (as in ``wrangle up'' --- see
Naumowicz~\cite{naumowicz2009enhanced} for details about this process)
all the adjectives for us. For example:
\begin{mizar}
registration
  cluster empty -> Relation-like set;
  correctness; :: omitted
  cluster empty -> Function-like set;
  correctness; :: omitted
end;

registration
  cluster empty set;
  existence; :: omitted
end;
\end{mizar}
What's really going on ``under the hood'' is that when \Mizar\ 
encounters ``\texttt{cluster empty set}'', it is remembering all the
conditional registrations on \texttt{empty}, and this is secretly
registering the existential cluster:
\begin{mizar}
registration
  cluster Relation-like Function-like empty set;
  existence; :: omitted
end;
\end{mizar}
Why is this important? Because importing such a registration is only
possible if the constructors of all its adjectives are imported in the
article's environment. You will get perplexing errors otherwise, but the
\textsc{Constr} utility may be able to help detect these errors.
\end{node}
\end{node}

\begin{node}[Functorial registration]\label{mizar-001Z}%
Similar to conditional registration, but for terms. That is to say, when
we want to register a term satisfies some attribute or adjective
``automatically'', then we use a functorial registration. For example:
\begin{mizar}
registration
  let X be non empty set;
  let Y be set;
  cluster X \/ Y -> non empty;
  coherence;
end;
\end{mizar}
This tells \Mizar\ that $X\cup Y$ is automatically nonempty when $X$ is
nonempty. 
\end{node}
