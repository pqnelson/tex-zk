% 4
\chapter{Mizar}

\begin{node}\label{mizar-0000}%
\Mizar\ is both a formal language and the name of a proof assistant
which uses the formal language as its input language. Unlike other proof
assistants, \Mizar\ focuses on readability and reflecting how working
mathematicians write mathematical proofs ``in the wild''.

Broadly speaking, the input language for any proof assistant (and,
indeed, any formal language offered to formalize mathematics) can be
partitioned into two interconnected sub-languages: the first for definitions,
the second for theorems and proofs. These share the same language for
encoding formulas (``propositions'') and terms.
\end{node}

\begin{node}[Foundations]\label{mizar-0001}%
The foundations for \Mizar\ is purely pragmatic, reflecting working
mathematics. Specifically, it uses a slightly modified version of
first-order logic (jokingly referred to as $(1+\varepsilon)$-order logic)
using free second-order variables for schemes, using a soft-type system
atop untyped logic, and Tarski--Grothendieck set theory~\zref{sec:set:tarski-grothendieck}.

I am told that there were early experiments using \MK\ set theory, but
Andrzej Trybulec and friends ran into difficulties formalizing category
theory. This led the \Mizar\ group to strengthen their foundations to
Tarski--Grothendieck set theory.
\end{node}

\begin{node}[Formulas]\label{mizar-0003}%
Logical formulas can be translated in a fairly straightforward manner,
using the following recursive translation process:
\begin{center}
\begin{tabular}{rcl}
$\neg\alpha$ & $\mapsto$ & \texttt{not} $\alpha$\\
$\alpha\land\beta$ & $\mapsto$ & $\alpha$ \texttt{\&} $\beta$\\
$\alpha\lor\beta$ & $\mapsto$ & $\alpha$ \texttt{or} $\beta$\\
$\alpha\implies\beta$ & $\mapsto$ & $\alpha$ \texttt{implies} $\beta$\\
$\alpha\iff\beta$ & $\mapsto$ & $\alpha$ \texttt{iff} $\beta$\\
$(\exists x)\alpha$ & $\mapsto$ & \texttt{ex x st} $\alpha$\\
$(\forall x)\alpha$ & $\mapsto$ & \texttt{forall x holds} $\alpha$\\
$(\forall x)(\alpha\implies\beta)$ & $\mapsto$ & \texttt{forall x st} $\alpha$ \texttt{holds} $\beta$\\
\end{tabular}
\end{center}
Atomic formulas are built either applying predicates to terms, or
something of the form ``$\langle$\textit{term}$\rangle$ \texttt{is}
$\langle$\textit{type or adjective}$\rangle$''.
\end{node}

\section{Proof Steps}

\begin{node}\label{mizar-0002}%
The proof language, when using informal notation for terms and formulas,
may stand alone as its own formal language. Wiedijk~\cite{wiedijk2000vernacular}
presents this as \emph{a} mathematical vernacular.
\end{node}