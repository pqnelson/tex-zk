\begin{node}[Records]\label{sml:core-0001}%
\textbf{Type:} \verb|{|$\ell_{1}\esti T_{1},\dots,\ell_{n}\esti T_{n}$\verb|}|
  where $\ell_{1}$, \dots, $\ell_{n}$ are ``labels'' (which are valid
  identifiers) and $T_{1}$, \dots, $T_{n}$ are types, and $n\geq0$.
  
\textbf{Values:} \verb|{|$\ell_{1}=v_{1},\dots,\ell_{n}=v_{n}$\verb|}|
  where $v_{1}\esti T_{1}$, \dots, $v_{n}\esti T_{n}$ are values

\textbf{Expressions:} The collection of expressions for this type
includes all values of type \verb|{|$\ell_{1}\esti T_{1},\dots,\ell_{n}\esti T_{n}$\verb|}|,
as well as all expressions of the form
\verb|{|$\ell_{1}=e_{1},\dots,\ell_{n}=e_{n}$\verb|}|
  where $e_{1}\esti T_{1}$, \dots, $e_{n}\esti T_{n}$ are expressions of
  appropriate type.

\textbf{Operations:} If all the types of the entries are equality types,
then we have equality defined on the record type (defined by
componentwise comparison of key-value pairs with the same labels).

\textbf{Typing rules:}
\begin{subequations}
\begin{equation}
\infer[\Rule{T-EmptyRec}]{\mbox{\texttt{\string{\string}}}\esti\mbox{\texttt{\string{\string}}}}{}
\end{equation}
\begin{equation}
\infer[\Rule{T-Rec}]{\mbox{\texttt{\string{$\ell_{1}=e_{1},\dots,\ell_{n}=e_{n}$\string}}}\esti\mbox{\texttt{\string{$\ell_{1}\esti T_{1},\dots,\ell_{n}\esti T_{n}$\string}}}}{e_{1}\esti T_{1}, & \dots & e_{n}\esti T_{n}}
\end{equation}
\end{subequations}

\textbf{Evaluation:} Informally, we evaluate a record from left to right
until all entries are values.
\begin{subequations}
\begin{equation}
\infer[\Rule{V-EmptyRec}]{\mbox{\texttt{\string{\string}}}~\Value}{}
\end{equation}
\begin{equation}
\infer[\Rule{E-Rec1}]{\mbox{\texttt{\string{$\ell_{1}=v_{1},\dots,\ell_{k-1}=v_{k-1},\ell_{k}=e_{k}\dots$\string}}}\Reduces{1}\mbox{\texttt{\string{$\ell_{1}=v_{1},\dots,\ell_{k-1}=v_{k-1},\ell_{k}=e_{k}'\dots$\string}}}}{v_{1}~\Value, & \dots & v_{k-1}~\Value, & e_{k}\Reduces{1}e_{k}'}
\end{equation}
\end{subequations}

\textbf{Patterns:} We can match a record value against a pattern which
looks like a record, in that it has bracket delimiters, and is a finite
sequence of
``$\langle$\textit{label}$\rangle$=$\langle$\textit{pattern}$\rangle$''
items (separated by commas), optionally ending with ellipses ``\verb|...|''
as a wildcard pattern to match the rest of the record;
e.g., \verb|{|$\ell_{1}=p_{1}$, $\ell_{2}=p_{2}$, \verb|...}| is a valid
pattern. 

\begin{node}[Subtlety around extensional equivalence]\label{sml:core-0002}%
There is some subtlety here with how \SML\ defines semantics of record
types and how we defined extensional equivalence~\zref{sml-0008}, because if we have \verb|{|$\ell_{1} = $ \verb|raise| $en_{1}$, $\ell_{2} = $ \verb|raise| $en_{2}$ \verb|}|
and  \verb|{|$\ell_{2} = $ \verb|raise| $en_{2}$, $\ell_{1} = $ \verb|raise| $en_{1}$ \verb|}|
where $en_{1}\not\exteq en_{2}$, then these two record instances are
not extensionally equivalent: they raise different exceptions. Therefore
the two records with permuted entries are not extensionally equivalent
\emph{when two or more entries raise extensionally inequivalent exceptions}.
But for values, or records with entries which loop forever, they will be
extensionally equivalent.

I suppose one way around this problem is to have the parser sort the
entries in a record alphabetically by its label. But this inelegantly
adds a lot of overhead to the rules as we have introduced them,
requiring $\ell_{1}<\dots<\ell_{n}$.
\end{node}
\begin{node}\label{sml:core-0003}%
Note, in Appendix A of the Definition~\cite{milner1997definition},
the product type~\zref{sml-000H} is syntactic sugar around the record
type $T_{1}\mbox{ \texttt{*} }T_{2}$ is really treated as synonymous with
\texttt{\string{$\quasiquote{1}\esti T_{1},\quasiquote{2}\esti T_{2}$\string}}.
Then the empty product is precisely \texttt{\string{\string}}, and this
is isolated as a separate special type called the \define{Unit Type}.
\end{node}
\end{node}
