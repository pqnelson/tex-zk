

\begin{node}[Lists]\label{sml-000L}%
\textbf{Type:} ``\texttt{T list}'' for any type \texttt{T}

\textbf{Values:} $\texttt{[}v_{1},\dots,v_{n}\texttt{]}$ where
$v_{j}$ is a value of type $T$ and $n\geq0$. We also have \texttt{[]} be
a value called ``nil'' or ``the empty list''.

\textbf{Expressions:} The collection of expressions for this type
includes all values of type \texttt{T list}, and it also includes all
expressions of the form $e\texttt{::}\ell$ with $e\esti T$ and
$\ell\esti\texttt{T list}$. Note that $\texttt{[}e_{1},\dots,e_{n}\texttt{]}$
is syntactic sugar for $e_{1}\texttt{::}\dots\texttt{::}e_{n}\texttt{::}\texttt{[]}$.

The infixed operator ``\texttt{::}'' is right associative. We pronounce
it as ``cons''.

\textbf{Typing Rules:}
\begin{subequations}
\begin{equation}
\infer[\Rule{T-Nil}]{\texttt{[]}\esti T~\texttt{list}}{}
\end{equation}
\begin{equation}
\infer[\Rule{T-Cons}]{t\texttt{::}\ell\esti T~\texttt{list}}{t\esti T & \ell\esti T~\texttt{list}}
\end{equation}
\end{subequations}

\textbf{Evaluation:}
\begin{subequations}
\begin{equation}
\infer[\Rule{E-Nil}]{\texttt{[]}~\Value}{}
\end{equation}
\begin{equation}
\infer[\Rule{E-Cons1}]{e_{1}\texttt{::}e_{2}\Reduces{1}e_{1}'\texttt{::}e_{2}}{e_{1}\Reduces{1}e_{1}'}
\end{equation}
\begin{equation}
\infer[\Rule{E-Cons2}]{v_{1}\texttt{::}e_{2}\Reduces{1}v_{1}\texttt{::}e_{2}}{%
  v_{1}~\Value & e_{2}\Reduces{1}e_{2}'}
\end{equation}
\end{subequations}

\textbf{Patterns:} There are two patterns which will match a list
\begin{enumerate}
\item \texttt{[]} matches the empty list, and
\item $p_{h}\texttt{::}p_{\ell}$ will try to match
  pattern $p_{h}$ against the head of the list and $p_{\ell}$ against
  the tail of the list.
\end{enumerate}

\begin{definition}[{Johnstone~\cite[A2.5.15]{johnstone2002sketches:v1}}]\label{sml-000N}%
Let $\cat{C}$ be a category with finite products and a terminal object
$\mathbf{1}$. Let $A\in\cat{C}$ be an object. Then a \define{List Object} over
$A$ is an object $L_{A}\in\cat{C}$ equipped with
\begin{enumerate}
\item a morphism $o_{A}\colon\mathbf{1}\to L_{A}$ called ``nil'', and
\item a morphism $s_{A}\colon A\times L_{A}\to L_{A}$ called ``cons''
\end{enumerate}
such that
\begin{enumerate}
\item Universal Property of Lists: for any object $B$ of $\cat{C}$ and morphisms
  $b\colon\mathbf{1}\to B$ and $t\colon A\times B\to B$, there exists a
  unique morphism $f\colon L_{A}\to B$ 
  such that the following diagram commutes:
\begin{equation}\def\comma{,}
  \begin{tikzcd}[sep=large]
\mathbf{1}
   \arrow[r, "o_{A}"]
   \arrow[dr, swap, "b"]
   & L_{A} \arrow[d, dashed, "f"]
   & A\times L_{A} \arrow[l, swap, "s_{A}"]\arrow[d, dashed, "\langle \id_{A}\comma f\rangle"] \\
 {}
 & B
 & A\times B \arrow[l, "t"]
\end{tikzcd}
\end{equation}
\end{enumerate}
The universal property says $f(\texttt{[]})=b()$ and $(t\circ\langle\id_{A}, f\rangle)(a\texttt{::}\ell)=t(a,f(\ell))=f(a\texttt{::}\ell)$.
In \SML\ this says that $f$ is just \verb|foldr t b()| and therefore
lists in \SML\ satisfy the universal property of lists.
\end{definition}
\end{node}
