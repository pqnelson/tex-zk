\begin{node}[Characters]\label{sml-000O}%
\textbf{Type:} \texttt{char}

\textbf{Values:} The \SML\ Definition~\cite[\S2.2]{milner1997definition}
allows for an ordered alphabet with $N\geq 127$ letters provided the
first 127 letters coincide with ASCII. The literal values look like
\Char{a}, \Char{b}, etc. There are escape sequences
\begin{itemize}
\item \Char{\texttt{\textbackslash a}} (for ``alert'', ASCII 7)
\item \Char{\texttt{\textbackslash b}} (for ``alert'', ASCII 8)
\item \Char{\texttt{\textbackslash t}} (for ``tab'', ASCII 9)
\item \Char{\texttt{\textbackslash n}} (for ``linefeed'' or ``new line'', ASCII 10)
\item \Char{\texttt{\textbackslash v}} (for ``vertical tab'', ASCII 11)
\item \Char{\texttt{\textbackslash f}} (for ``form feed'', ASCII 12)
\item \Char{\texttt{\textbackslash r}} (for ``carriage return'', ASCII 13)
\item \Char{\texttt{\textbackslash"}} for a single quote
\item \Char{\texttt{\textbackslash\textbackslash}} for a backslash (``\textbackslash'')
\item \Char{\textbackslash\texttt{u}$xxxx$} where $x$ is a hexadecimal
  digit refers to the character with codepoint given by $xxxx$
\item \Char{\textbackslash$ddd$} where $d$ is a digit, refers to the
  character whose codepoint is given by $ddd$ provided $0\leq ddd\leq255$.
\end{itemize}
Note that ASCII considers a character ``printable'' if its code point
value lies between numbers 33-126, inclusive.

\textbf{Operations:} This is an equality type, so we may compare two
characters $c_{1}\texttt{ = }c_{2}$ and $c_{1}\texttt{<>} c_{2}$ (for
$c_{1}\neq c_{2}$). We may also obtain the ``codepoint'' for the
character using \texttt{ord}~$c\esti\texttt{int}$. This allows us to
write comparison operations for characters, e.g., $c_{1}\texttt{ < }c_{2}$
iff $\texttt{ord}~c_{1}\texttt{ < }\texttt{ord}~c_{2}$.

We can turn an integer into a character using $\texttt{chr}~n\esti\texttt{char}$,
which raises the \texttt{Chr} exception for negative or overly large arguments.


\textbf{Typing Rules:}
\begin{subequations}
\begin{equation}
\infer[\Rule{T-CharOrd}]{\texttt{ord}~c\esti\texttt{int}}{c\esti\texttt{char}}
\end{equation}
\begin{equation}
\infer[\Rule{T-CharChr}]{\texttt{chr}~n\esti\texttt{char}}{n\esti\texttt{int}}
\end{equation}
\end{subequations}

\textbf{Evaluation:} If we denote by $N$ the maximum possible code point
value for characters for our encoding, then we have the following semantics:
\begin{subequations}
\begin{equation}
\infer[\Rule{E-CharOrd}]{\texttt{ord}~c~\Value}{c~\Value}
\end{equation}
\begin{equation}
\infer[\Rule{E-CharChr}]{\texttt{chr}~n\Reduces{1}c}{\quasiquote{n}~\Value & c~\Value &
  0\leq n\leq N & \texttt{ord}~c=\quasiquote{n}}
\end{equation}
\begin{equation}
\infer[\Rule{E-CharLt1}]{c_{1}\texttt{ < }c_{2}\Reduces{1}c_{1}'\texttt{ < }c_{2}}{c_{1}\Reduces{1}c_{1}'}
\end{equation}
\begin{equation}
\infer[\Rule{E-CharLt2}]{v_{1}\texttt{ < }c_{2}\Reduces{1}v_{1}\texttt{ < }c_{2}'}{c_{2}\Reduces{1}c_{2}' & v_{1}~\Value}
\end{equation}
\begin{equation}
\infer[\Rule{E-CharLtV}]{v_{1}\texttt{ < }v_{2}\Reduces{1}\texttt{ord}~v_{1}\texttt{ < }\texttt{ord}~v_{2}}{v_{2}~\Value & v_{1}~\Value}
\end{equation}
\begin{equation}
\infer[\Rule{E-CharLeq1}]{c_{1}\texttt{ <= }c_{2}\Reduces{1}c_{1}'\texttt{ <= }c_{2}}{c_{1}\Reduces{1}c_{1}'}
\end{equation}
\begin{equation}
\infer[\Rule{E-CharLeq2}]{v_{1}\texttt{ <= }c_{2}\Reduces{1}v_{1}\texttt{ <= }c_{2}'}{c_{2}\Reduces{1}c_{2}' & v_{1}~\Value}
\end{equation}
\begin{equation}
\infer[\Rule{E-CharLeqV}]{v_{1}\texttt{ <= }v_{2}\Reduces{1}\texttt{ord}~v_{1}\texttt{ <= }\texttt{ord}~v_{2}}{v_{2}~\Value & v_{1}~\Value}
\end{equation}
\end{subequations}
\end{node}
