\begin{node}[Integers]\label{sml-000E}%
\textbf{Type:} \texttt{int}.

\textbf{Values:} All integer values. If $n\in\ZZ$ is an integer, we will
write $\quasiquote{n}$ for the \SML\ expression which is the integer
literal with value $n$ (so $\quasiquote{7}$ refers to the
\SML\ expression \texttt{7}).

\textbf{Operations:} Let $e_{1}$, $e_{2}$ be \texttt{int}
expressions. Then so are $e_{1}\mbox{\texttt{ + }}e_{2}$,
$e_{1}\mbox{\texttt{ - }}e_{2}$, $e_{1}\mbox{\texttt{ * }}e_{2}$,
$e_{1}\mbox{\texttt{ div }}e_{2}$, $e_{1}\mbox{\texttt{ mod }}e_{2}$,
and possibly a few others.

\textbf{Typing Rules:}
\begin{subequations}
\begin{equation}
\infer[\Rule{T-IntAdd}]{e_{1}\mbox{\texttt{ + }}e_{2}\esti\mathtt{int}}{e_{1}\esti\mathtt{int}
& e_{2}\esti\mathtt{int}}
\end{equation}
\begin{equation}
\infer[\Rule{T-IntSub}]{e_{1}\mbox{\texttt{ - }}e_{2}\esti\mathtt{int}}{e_{1}\esti\mathtt{int}
& e_{2}\esti\mathtt{int}}
\end{equation}
\begin{equation}
\infer[\Rule{T-IntMul}]{e_{1}\mbox{\texttt{ * }}e_{2}\esti\mathtt{int}}{e_{1}\esti\mathtt{int}
& e_{2}\esti\mathtt{int}}
\end{equation}
\begin{equation}
\infer[\Rule{T-IntDiv}]{e_{1}\mbox{\texttt{ div }}e_{2}\esti\mathtt{int}}{e_{1}\esti\mathtt{int}
& e_{2}\esti\mathtt{int}}
\end{equation}
\begin{equation}
\infer[\Rule{T-IntMod}]{e_{1}\mbox{\texttt{ mod }}e_{2}\esti\mathtt{int}}{e_{1}\esti\mathtt{int}
& e_{2}\esti\mathtt{int}}
\end{equation}
\end{subequations}

\textbf{Evaluation:} Like all of \SML, evaluation occurs from left to
right until we obtain values. Formally:
\begin{subequations}
\begin{equation}
\infer[\Rule{E-IntAdd1}]{e_{1}\mbox{\texttt{ + }}e_{2}\Reduces{1}e_{1}'\mbox{\texttt{ + }}e_{2}}{e_{1}\Reduces{1}e_{1}'}
\end{equation}
\begin{equation}
\infer[\Rule{E-IntAdd2}]{n_{1}\mbox{\texttt{ + }}e_{2}\Reduces{1}n_{1}\mbox{\texttt{ + }}e_{2}'}{e_{2}\Reduces{1}e_{2}'
& n_{1}~\Value}
\end{equation}
\begin{equation}
\infer[\Rule{E-IntAddV}]{\quasiquote{n_{1}}\mbox{\texttt{ + }}\quasiquote{n_{2}}\Reduces{1}\quasiquote{n_{1}+n_{2}}}{%
\quasiquote{n_{1}}~\Value &
\quasiquote{n_{2}}~\Value}
\end{equation}
\end{subequations}
\begin{subequations}
\begin{equation}
\infer[\Rule{E-IntSub1}]{e_{1}\mbox{\texttt{ - }}e_{2}\Reduces{1}e_{1}'\mbox{\texttt{ - }}e_{2}}{e_{1}\Reduces{1}e_{1}'}
\end{equation}
\begin{equation}
\infer[\Rule{E-IntSub2}]{n_{1}\mbox{\texttt{ - }}e_{2}\Reduces{1}n_{1}\mbox{\texttt{ - }}e_{2}'}{e_{2}\Reduces{1}e_{2}'
& n_{1}~\Value}
\end{equation}
\begin{equation}
\infer[\Rule{E-IntSubV}]{\quasiquote{n_{1}}\mbox{\texttt{ - }}\quasiquote{n_{2}}\Reduces{1}\quasiquote{n_{1}-n_{2}}}{%
\quasiquote{n_{1}}~\Value &
\quasiquote{n_{2}}~\Value}
\end{equation}
\end{subequations}
\begin{subequations}
\begin{equation}
\infer[\Rule{E-IntMul1}]{e_{1}\mbox{\texttt{ * }}e_{2}\Reduces{1}e_{1}'\mbox{\texttt{ * }}e_{2}}{e_{1}\Reduces{1}e_{1}'}
\end{equation}
\begin{equation}
\infer[\Rule{E-IntMul2}]{n_{1}\mbox{\texttt{ * }}e_{2}\Reduces{1}n_{1}\mbox{\texttt{ * }}e_{2}'}{e_{2}\Reduces{1}e_{2}'
& n_{1}~\Value}
\end{equation}
\begin{equation}
\infer[\Rule{E-IntMulV}]{\quasiquote{n_{1}}\mbox{\texttt{ * }}\quasiquote{n_{2}}\Reduces{1}\quasiquote{n_{1}\cdot n_{2}}}{%
\quasiquote{n_{1}}~\Value &
\quasiquote{n_{2}}~\Value}
\end{equation}
\end{subequations}
\begin{subequations}
\begin{equation}
\infer[\Rule{E-IntDiv1}]{e_{1}\mbox{\texttt{ div }}e_{2}\Reduces{1}e_{1}'\mbox{\texttt{ div }}e_{2}}{e_{1}\Reduces{1}e_{1}'}
\end{equation}
\begin{equation}
\infer[\Rule{E-IntDiv2}]{n_{1}\mbox{\texttt{ div }}e_{2}\Reduces{1}n_{1}\mbox{\texttt{ div }}e_{2}'}{e_{2}\Reduces{1}e_{2}'
& n_{1}~\Value}
\end{equation}
When we take $n$ \texttt{div} $k$ for $k\neq0$, it returns the
greatest integer $m$ such that $m\cdot k\leq n$. For the ``normal''
division situation, we have (with premises stacked vertically for clarity):
\begin{equation}
\infer[\Rule{E-IntDivV}]{\quasiquote{n_{1}}\mbox{\texttt{ div }}\quasiquote{n_{2}}\Reduces{1}\quasiquote{m}}{%
\deduce{\quasiquote{n_{1}}~\Value}{\deduce{%
\quasiquote{n_{2}}~\Value}{%
\deduce{n_{2}\neq0}{%
m\cdot n_{2}\leq n_{1}<(m+1)\cdot n_{2}}}}}
\end{equation}
When we try to divide by zero, an exception is raised\footnote{This is
according to the \SML\ Basis specification \url{https://smlfamily.github.io/Basis/integer.html\#SIG:INTEGER.div:VAL}}.
\begin{equation}
\infer[\Rule{E-IntDivZ}]{\quasiquote{n_{1}}\mbox{\texttt{ div }}\quasiquote{0}\Reduces{1}\Raise{\texttt{Div}}}{%
\quasiquote{n_{1}}~\Value &
\quasiquote{0}~\Value}
\end{equation}
\end{subequations}
\begin{subequations}
\begin{equation}
\infer[\Rule{E-IntMod1}]{e_{1}\mbox{\texttt{ mod }}e_{2}\Reduces{1}e_{1}'\mbox{\texttt{ mod }}e_{2}}{e_{1}\Reduces{1}e_{1}'}
\end{equation}
\begin{equation}
\infer[\Rule{E-IntMod2}]{n_{1}\mbox{\texttt{ mod }}e_{2}\Reduces{1}n_{1}\mbox{\texttt{ mod }}e_{2}'}{e_{2}\Reduces{1}e_{2}'
& n_{1}~\Value}
\end{equation}
We have $(\quasiquote{i}$ \texttt{div} $\quasiquote{j})$ \texttt{*} $\quasiquote{j}$ \texttt{+}
$(\quasiquote{i}$ \texttt{mod} $\quasiquote{j}) = \quasiquote{i}$.
\begin{equation}
\infer[\Rule{E-IntModV}]{\quasiquote{n_{1}}\mbox{\texttt{ mod }}\quasiquote{n_{2}}\Reduces{1}\quasiquote{m}}{%
\deduce{\quasiquote{n_{1}}~\Value}{\deduce{\quasiquote{n_{2}}~\Value}{%
\deduce{n_{2}\neq0}{%
\lfloor n_{1}/n_{2}\rfloor\cdot n_{2} + m = n_{1}}}}}
\end{equation}
Following the specification\footnote{\url{https://smlfamily.github.io/Basis/integer.html\#SIG:INTEGER.mod:VAL}}, when trying to take any value modulo zero
should raise an exception:
\begin{equation}
\infer[\Rule{E-IntModZ}]{\quasiquote{n_{1}}\mbox{\texttt{ mod }}\quasiquote{0}\Reduces{1}\Raise{\texttt{Div}}}{%
\quasiquote{n_{1}}~\Value &
\quasiquote{0}~\Value}
\end{equation}
\end{subequations}
\end{node}
