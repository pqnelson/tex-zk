\begin{node}[Products]\label{sml-000H}%
\textbf{Type:} $T_{1}\mbox{ \texttt{*} }T_{2}$ for any types $T_{1}$ and $T_{2}$.
More generally, for any $n\geq 2$ and types $T_{1}$, \dots, $T_{n}$, we
form the product type $T_{1}\mbox{ \texttt{*} }\dots\mbox{ \texttt{*} }T_{n}$.

\textbf{Values:} For values $v_{1}\esti T_{1}$, \dots, $v_{n}\esti T_{n}$
we have the tuple $(v_{1},\dots,v_{n})$ be a value of type 
$T_{1}\mbox{ \texttt{*} }\dots\mbox{ \texttt{*} }T_{n}$.

\textbf{Operations:} We can have projections for product types, but
instead we use patterns to destructure tuples.

\textbf{Typing Rules:}
\begin{equation}
\infer[\Rule{T-Tuple}]{(t_{1},\dots,t_{n})\esti T_{1}\mbox{ \texttt{*} }\dots\mbox{ \texttt{*} }T_{n}}{%
t_{1}\esti T_{1},&\dots,&t_{n}\esti T_{n}
}
\end{equation}

\textbf{Evaluation:} We evaluate components of a tuple from left to
right, until all the components are values.
\begin{equation}
\infer[\Rule{E-Tuple}_{k}]{(v_{1},\dots,v_{k-1},t_{k},\dots,t_{n})\Reduces{1}(v_{1},\dots,v_{k-1},t_{k}',\dots,t_{n})}{%
v_{1}~\Value &\dots&v_{k-1}~\Value&t_{k}\Reduces{1}t_{k}'}
\end{equation}

\begin{node}[Remark]\label{sml-000J}%
Note that the record type is ``more primitive'' than the product type,
since $T_{1}\mbox{ \texttt{*} }\dots\mbox{ \texttt{*} }T_{n}$ is then
translated implicitly to
$\{\quasiquote{1}\esti T_{1},\dots,\quasiquote{n}\esti T_{n}\}$.
Tuples $(t_{1},\dots,t_{n})$ is translated to records
$\{\quasiquote{1}=t_{1},\dots,\quasiquote{n}=t_{n}\}$.
This is stated in Appendix A of the definition~\cite{milner1997definition}.
\end{node}
\end{node}
