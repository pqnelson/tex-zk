\begin{node}[Booleans]\label{sml-000G}%
\textbf{Type:} \texttt{bool}

\textbf{Values:} \texttt{true} and \texttt{false}

\textbf{Operations:} $e_{1}\mbox{ \texttt{andalso} }e_{2}$,
$e_{1}\mbox{ \texttt{orelse} }e_{2}$,
$\mbox{\texttt{if }}e_{1}\mbox{ \texttt{then} }e_{2}\mbox{ \texttt{else} }e_{3}$,
$e_{1}\mbox{ \texttt{<} }e_{2}$,
$e_{1}\mbox{ \texttt{<=} }e_{2}$,
$e_{1}\mbox{ \texttt{=} }e_{2}$ when $e_{1}$ and $e_{2}$ belong to an ``equality type''.
We will take $e_{1}>e_{2}$ as a synonym for $e_{2}<e_{1}$,
and $e_{1}\geq e_{2}$ as a synonym for $e_{2}\leq e_{1}$.

\textbf{Typing Rules:}
\begin{subequations}
\begin{equation}
\infer[\Rule{T-If}]{\mbox{\texttt{if }}e_{1}\mbox{ \texttt{then} }e_{2}\mbox{ \texttt{else} }e_{3}\esti T}{e_{1}\esti\mathtt{bool} & e_{2}\esti T & e_{3}\esti T}
\end{equation}
\begin{equation}
\infer[\Rule{T-And}]{e_{1}\mbox{ \texttt{andalso} }e_{2}}{e_{1}\esti\mathtt{bool}
& e_{2}\esti\mathtt{bool}}
\end{equation}
\begin{equation}
\infer[\Rule{T-Or}]{e_{1}\mbox{ \texttt{orelse} }e_{2}}{e_{1}\esti\mathtt{bool}
& e_{2}\esti\mathtt{bool}}
\end{equation}
\begin{equation}
\infer[\Rule{T-Lt}]{e_{1}\mbox{ \texttt{<} }e_{2}\esti\mathtt{bool}}{%
  e_{1}\esti T &
  e_{2}\esti T &
  T\in\{\mathtt{string},\mathtt{int},\mathtt{char}\}}
\end{equation}
\begin{equation}
\infer[\Rule{T-Leq}]{e_{1}\mbox{ \texttt{<=} }e_{2}\esti\mathtt{bool}}{%
  e_{1}\esti T &
  e_{2}\esti T &
  T\in\{\mathtt{string},\mathtt{int},\mathtt{char}\}}
\end{equation}
\end{subequations}

\textbf{Evaluation:}
\begin{subequations}
\begin{equation}
\infer[\Rule{E-If1}]{{\mbox{\texttt{if }}e_{1}\mbox{ \texttt{then} }e_{2}\mbox{ \texttt{else} }e_{3}}\Reduces{1}{\mbox{\texttt{if }}e_{1}'\mbox{ \texttt{then} }e_{2}\mbox{ \texttt{else} }e_{3}}}{e_{1}\Reduces{1}e_{1}'}
\end{equation}
\begin{equation}
\infer[\Rule{E-IfTrue}]{\mbox{\texttt{if }}\mathtt{true}\mbox{ \texttt{then} }e_{2}\mbox{ \texttt{else} }e_{3}\Reduces{1}e_{2}}{}
\end{equation}
\begin{equation}
\infer[\Rule{E-IfFalse}]{\mbox{\texttt{if }}\mathtt{false}\mbox{ \texttt{then} }e_{2}\mbox{ \texttt{else} }e_{3}\Reduces{1}e_{3}}{}
\end{equation}
\end{subequations}
We follow Appendix A of the Definition~\cite{milner1997definition} and
define as syntactic sugar [i.e., an abbreviation, which is automatically
translated by the compiler]:
\begin{equation}
e_{1}\mbox{ \texttt{andalso} }e_{2}\Reduces{0}\mathtt{if\ }e_{1}\mathtt{\  then\ }e_{2}\mathtt{\  else\  false}.
\end{equation}
As a consequence, we have the following derivable rules:
\begin{subequations}
\begin{equation}
\infer[\Rule{E-And1}]{e_{1}\mbox{ \texttt{andalso} }e_{2}\Reduces{1}e_{1}'\mbox{ \texttt{andalso} }e_{2}}{e_{1}\Reduces{1}e_{1}'}
\end{equation}
\begin{equation}
\infer[\Rule{E-AndT}]{\mathtt{true}\mbox{ \texttt{andalso} }e_{2}\Reduces{1}e_{2}}{}
\end{equation}
\begin{equation}
\infer[\Rule{E-AndF}]{\mathtt{false}\mbox{ \texttt{andalso} }e_{2}\Reduces{1}\mathtt{false}}{}
\end{equation}
\end{subequations}
Similarly, following Appendix A of the Definition~\cite{milner1997definition}, we have as syntactic sugar:
\begin{equation}
e_{1}\mbox{ \texttt{orelse} }e_{2}\Reduces{0}\mathtt{if\  }e_{1}\mathtt{\  then\ true\ }\mathtt{ else\  }e_{2}.
\end{equation}
We may obtain the following derivable rules:
\begin{subequations}
\begin{equation}
\infer[\Rule{E-Or1}]{e_{1}\mbox{ \texttt{orelse} }e_{2}\Reduces{1}e_{1}'\mbox{ \texttt{orelse} }e_{2}}{e_{1}\Reduces{1}e_{1}'}
\end{equation}
\begin{equation}
\infer[\Rule{E-OrT}]{\mathtt{true}\mbox{ \texttt{orelse} }e_{2}\Reduces{1}\mathtt{true}}{}
\end{equation}
\begin{equation}
\infer[\Rule{E-OrF}]{\mathtt{false}\mbox{ \texttt{orelse} }e_{2}\Reduces{1}e_{2}}{}
\end{equation}
\end{subequations}
\begin{subequations}
\begin{equation}
\infer[\Rule{E-Leq1}]{e_{1}\mbox{ \texttt{<=} }e_{2}\Reduces{1}e_{1}'\mbox{ \texttt{<=} }e_{2}}{%
e_{1}\Reduces{1}e_{1}'}
\end{equation}
\begin{equation}
\infer[\Rule{E-Leq2}]{n_{1}\mbox{ \texttt{<=} }e_{2}\Reduces{1}n_{1}\mbox{ \texttt{<=} }e_{2}'}{%
e_{2}\Reduces{1}e_{2}' & n_{1}~\Value}
\end{equation}
\begin{equation}
\infer[\Rule{E-LeqT}]{\quasiquote{n_{1}}\mbox{ \texttt{<=} }\quasiquote{n_{2}}\Reduces{1}\mathtt{true}}{n_{1}\leq n_{2}}
\end{equation}
\begin{equation}
\infer[\Rule{E-LeqF}]{\quasiquote{n_{1}}\mbox{ \texttt{<=} }\quasiquote{n_{2}}\Reduces{1}\mathtt{false}}{n_{1}>n_{2}}
\end{equation}
\end{subequations}
\begin{subequations}
\begin{equation}
\infer[\Rule{E-Lt1}]{e_{1}\mbox{ \texttt{<} }e_{2}\Reduces{1}e_{1}'\mbox{ \texttt{<} }e_{2}}{%
e_{1}\Reduces{1}e_{1}'}
\end{equation}
\begin{equation}
\infer[\Rule{E-Lt2}]{n_{1}\mbox{ \texttt{<} }e_{2}\Reduces{1}n_{1}\mbox{ \texttt{<} }e_{2}'}{%
e_{2}\Reduces{1}e_{2}' & n_{1}~\Value}
\end{equation}
\begin{equation}
\infer[\Rule{E-LtT}]{\quasiquote{n_{1}}\mbox{ \texttt{<} }\quasiquote{n_{2}}\Reduces{1}\mathtt{true}}{n_{1}<n_{2}}
\end{equation}
\begin{equation}
\infer[\Rule{E-LtF}]{\quasiquote{n_{1}}\mbox{ \texttt{<} }\quasiquote{n_{2}}\Reduces{1}\mathtt{false}}{n_{1}\geq n_{2}}
\end{equation}
\end{subequations}
\end{node}
