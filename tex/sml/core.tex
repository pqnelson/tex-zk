% 5
\section{Judgements and Inference Rules for Core Language}
\begin{node}\label{sml-0003}%
We need to introduce some judgement forms if we wish to view \SML\ as
``just another deductive system''. Then we will introduce the base types
(Booleans, integers, functions, and so on). We will follow the basic
approach found in Erdmann's lecture notes~\cite{erdmann2023evaluation}.
It appears to be inspired by Harper and Stone's earlier work~\cite{harper2000:smltt-final}.
\end{node}

\transclude{core/judgement-forms}
\transclude{core/extensional-equivalence}


\begin{node}[Primitive types]\label{sml:core-0004}%
\begin{node}\label{sml-000D}%
We can now discuss the base types for our fragment of \SML. The basic
template will be the same for all entries:
\begin{enumerate}
\item Type: the concrete syntax for the type in \SML;
\item Values: the set of subexpressions of this type which are values;
\item Operations: the set of, what the Definition~\cite{milner1997definition}
  calls, ``basic operations'' and adheres to rule (101) of the
  Definition (the goal is to minimize the number of basic operations,
  which will shorten the length of typing rules and evaluation rules);
\item Typing rules: inference rules for determining if an expression
  inhabits the type;
\item Evaluation: inference rules for describing the dynamics of the
  reduction relation;
\item Patterns: for certain types, the rules for pattern matching.
\end{enumerate}
\end{node}

\transclude{core/int}
\transclude{core/bool}
\transclude{core/product}
\transclude{core/fun-provisional}
\transclude{core/lists}
\transclude{core/char}
\transclude{core/string}
\transclude{core/record}
\end{node}


\begin{node}[Definitions, local environment]\label{sml:core-0000}%
The Definition~\cite{milner1997definition} treats bindings of
identifiers to values, types, and constructors as three finite partial
functions collectively referred to as the environment. The environment
is extended by \texttt{let}-expressions and local bindings. We treat the
definition portion of the judgement $\Delta$ similarly, but written in
reverse order: if we add $id=v$ to the list of definitions, then we
prepend it to $\Delta\mapsto(id=v),\Delta$. If we need to keep track of
local variables, I will partition $\Delta$ by a vertical bar
$(id_{loc}=v)|\Delta$\dots or something similar. 
\end{node}

\begin{node}\label{sml-000Q}%
The evaluation of a \SML\ program amounts to checking a declaration is
well-typed, then evaluating it, and adding it to the list of definitions
\begin{subequations}
\begin{equation}
\infer[\Rule{E-Val}]{\Delta\vdash (\mathtt{val}~id=e);prog\Reduces{k}id=v,\Delta\vdash prog}{\Delta\vdash e\Reduces{k}v
& \Delta\vdash e\esti T & \Delta\vdash v~\Value}
\end{equation}
Functions require a bit more care because we allow recursion. In this
case, we temporarily add a binding for the identifier of the function
when evaluating it:
\begin{equation}
\infer[\Rule{E-Fun}]{\Delta\vdash (\mathtt{fun}~id~p);prog\Reduces{k}id=v,\Delta\vdash prog}{id=\mathtt{fn}~p,\Delta\vdash (\mathtt{fn}~p)\Reduces{k}v
& \Delta\vdash (\mathtt{fn}~p)\esti T\to T' & \Delta\vdash v~\Value}
\end{equation}
\end{subequations}
These correspond to the rules found in the \SML\ definition for
evaluating top-level programs.

Care must be taken because, despite all this, each implementation of
\SML\ has its own way of determining the ``main function'' to be
executed as the program. MLton will allow you to run code wherever you
wish, but Poly/ML requires an entry point --- specifically a function
named \verb|main()|. Chris Cannam observed this difficulty in his blogpost\footnote{\url{https://thebreakfastpost.com/2015/06/10/standard-ml-and-how-im-compiling-it/}}.
\end{node}


%% \textbf{Type:}
%% \textbf{Values:}
%% \textbf{Operations:}
%% \textbf{Typing rules:}
%% \textbf{Evaluation:}
%% \textbf{Patterns:}
