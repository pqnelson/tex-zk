% 15
\begin{node}\label{nbg-0000}%
We can formalize a minimal notion of ``proper classes'' and treat
sets as a synonym for classes contained in another class.
\end{node}

\begin{puzzle}\label{nbg-0001}%
Can we formalize \NBG\ set theory using the Mizar language? Arguably, the
unique primitive notion would be a ``class'' instead of a ``set'', and
we could define ``\texttt{mode set} \verb|->| \texttt{class means ex C
  being class st it in C}''. The primitive notion ``\texttt{in}'' is a
predicate taking a mathematical object as its left argument and a class
as its right argument. We also have a primitive equality predicate.

\begin{node}[References]\label{nbg-0002}%
There are few good references on \NBG\ set
theory. Banakh~\cite{banakh2023classical} has some lecture
notes. Someone has done some heroic work on the English wikipedia page
for \NBG\ set theory. Most texts discussing the \NBG\ axioms follow
G\"{o}del's treatment~\cite{godel1940consistency}.
%% Pulgar\'{\i}n and Uribe-Zapata~\cite{pulgarin2023came} wrote a paper
%% trying to make finitistic metatheory rigorous.
\end{node}

\begin{node}[NBG is about sets]\label{nbg-0003}%
Remember, the whole point of \NBG\ set theory is to reason about sets. We
just get the ability to explicitly make statements concerning proper
classes as a side effect.
\end{node}
\end{puzzle}

\begin{node}[Axioms]\label{nbg-0004}%
The axioms of \NBG\ set theory have the advantage that it is finitely
axiomatizable. This lends itself to working with a metatheory which is
finitistic and recursively enumerable.

\begin{node}[Axiom of extensionality]\label{nbg-0005}%
For all classes $A$ and $B$ such that every object $x$ satisfies $x\in A$
iff $x\in B$, it holds that $A=B$.
\end{node}

\begin{node}[Axiom of pair]\label{nbg-0006}%
For all sets $x$ and $y$ there exists a set $z$ such that for all sets
$u$, $u\in z$ if and only if $u=x$ or $u=y$.
\end{node}

\begin{definition}\label{nbg-0007}%
Let $x$ and $y$ be mathematical objects. We define the term $\langle x,y\rangle$
to be the set equal to $\langle x,y\rangle := \{\{x,y\}, \{x\}\}$.
\end{definition}

\begin{node}[Axiom of membership]\label{nbg-0008}%
There exists a [proper] class $\properclass{E}$ such that for all sets $z$,
$z\in\properclass{E}$ if and only if there exists sets $x$ and $y$ such
that $x\in y$ and $z=\langle x,y\rangle$.
\end{node}

\begin{node}[Axiom of domain]\label{nbg-0009}%
For any class $X$ there exists a class $D$ such that for all sets $x$,
$x\in D$ if and only if there exists a set $y$ such that $\langle x,y\rangle\in X$.
\end{node}

\begin{definition}\label{nbg-000A}%
Let $X$ be a set. We define the term $\dom(X)$ (called the
\define{Domain} of $X$) to be the class equal to
$\dom(X):=\{x\mid\exists y,\langle x,y\rangle\in X\}$ where $x$ and $y$
are variables ranging over sets.
\end{definition}

\begin{node}\label{nbg-000B}%
We will denote by $\properclass{U}$ the proper class of the universe of sets.
In particular, $\dom(\properclass{E})=\properclass{U}$.
\end{node}

\begin{node}[Axiom of inversion]\label{nbg-000C}%
For any class $X$ there exists a class $Y$ such that for all sets $p$ we
have $p\in Y$ if and only if there exist sets $x$ and $y$ such that
$\langle x,y\rangle\in X$ and $p=\langle y,x\rangle$.
\end{node}
\end{node}

\begin{node}[Jech's axiomatization]\label{nbg-000D}%
Jech offers, at the end of chapter 6 of~\cite{jech2003set},
the following axiomatization of \NBG\ set theory:
\begin{enumerate}
  \item \begin{enumerate}
  \item Extensionality: $\forall u(u\in X\iff u\in Y)\implies X=Y$
  \item If $X\in Y$, then $X$ is a set
  \item Pairing: for any sets $x$ and $y$, $\{x,y\}$ is a set
  \end{enumerate}
  \item Comprehension $\forall X_{1}\dots\forall X_{n}\exists Y, Y=\{x\mid\varphi(x,X_{1},\dots,X_{n})\}$
    where $\varphi$ is a formula where only set variables are quantified.
  \item \begin{enumerate}
  \item Infinity: there exists an infinite set
  \item Union: for every set $x$, the set $\bigcup x$ exists
  \item Power set: for every set $x$ the powerset $\powerset(x)$ exists
  \item Replacement: If a class $F$ is a function and $x$ is a set, then
    $\{F(z)\mid z\in x\}$ is a set
  \end{enumerate}
  \item Regularity
  \item Choice: There is a function $F$ such that $F(x)\in x$ for every
    nonempty set $x$.
\end{enumerate}
\end{node}

\begin{node}\label{nbg-000E}%
When it comes to the benefits of finite axiomatizability,
Hester~\cite{hester2019automated} investigated whether there's any
benefits. It appears whatever benefits we might have gotten are lost in
the naive implementation, since we need to include extra predicates for
quantifiers over sets.
\end{node}
