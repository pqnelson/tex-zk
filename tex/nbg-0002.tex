\begin{node}[Jech's axiomatization]
  Jech offers, at the end of chapter 6 of \textit{Set Theory: The Third Millenium edition},
  the following axiomatization of NBG set theory:
  \begin{enumerate}
  \item \begin{enumerate}
  \item Extensionality: $\forall u(u\in X\iff u\in Y)\implies X=Y$
  \item If $X\in Y$, then $X$ is a set
  \item Pairing: for any sets $x$ and $y$, $\{x,y\}$ is a set
  \end{enumerate}
  \item Comprehension $\forall X_{1}\dots\forall X_{n}\exists Y, Y=\{x\mid\varphi(x,X_{1},\dots,X_{n})\}$
    where $\varphi$ is a formula where only set variables are quantified.
  \item \begin{enumerate}
  \item Infinity: there exists an infinite set
  \item Union: for every set $x$, the set $\bigcup x$ exists
  \item Power set: for every set $x$ the powerset $\powerset(x)$ exists
  \item Replacement: If a class $F$ is a function and $x$ is a set, then
    $\{F(z)\mid z\in x\}$ is a set
  \end{enumerate}
  \item Regularity
  \item Choice: There is a function $F$ such that $F(x)\in x$ for every
    nonempty set $x$.
  \end{enumerate}
\end{node}
