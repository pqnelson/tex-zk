% 35
\begin{node}\label{prop-0000}%
I'm following Mendelson's treatment~\cite{mendelson2015mathematical} of
propositional logic, with heavy correctives from
Harrison~\cite{harrison2009handbook}. Propositional logic deals with
only logical objects (``propositions''). Classical propositional logic
has propositions be either true or false, depending on the valuation of
the propositional variables and the logical connectives involved in the
proposition.
\end{node}

\begin{node}\label{prop-000H}%
We can divide the study of propositional logic into two parts: its
syntax and its semantics. The syntax of propositional logic amounts to
introducing its formal grammar and its informal interpretation, whereas
the semantics amounts to writing down truth tables.

This division into syntax and semantics survives when we study
first-order logic, and (ostensibly) any other formal system.
\end{node}

\transclude{prop/syntax}
\transclude{prop/semantics}

\begin{node}[Axiomatic System]\label{prop-000F}%
Mendelson~\cite[\S1.4]{mendelson2015mathematical} offers a so-called
``axiomatic system'' for propositional logic. It's a Hilbert proof
calculus, using implication $(\implies)$ and negation $(\neg)$ as the
primitive connectives, with a handful of axioms and only
\textit{modus ponens} as its rule of inference. More importantly, this
gives us a \emph{purely syntactic notion of proof} independent of
valuations and truth tables. We can prove that every theorem proven
by this system is a tautology, and vice-versa. In particular, Mendelson
proves the deduction theorem, the completeness theorem, and the consistency of
propositional logic.
\end{node}

\transclude{prop/normal-form}
