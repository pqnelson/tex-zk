\begin{node}[Connectives]\label{prop-0001}%
  The usual connectives are enumerated. We begin with negation.
\begin{enumerate}
\item If $A$ is
a proposition, then its \define{negation} is denoted $\neg A$. Its truth table is
\[\begin{array}{c|c}
A & \neg A\\ \hline
\verum & \falsum\\
\falsum & \verum\\
\end{array}\]
where $\verum$ and $\falsum$ are the truth values ``true'' and
``false'', respectively.
\item Let $A$ and $B$ be propositions. Their \define{conjunction} is the
  proposition denoted $A\land B$ with truth table
\[\begin{array}{cc|c}
A & B & A\land B\\\hline
\verum  & \verum  & \verum\\
\falsum & \verum  & \falsum\\
\verum  & \falsum & \falsum\\
\falsum & \falsum & \falsum\\
\end{array}\]
We refer to the subpropositions $A$ and $B$ as \define{Conjuncts} of
$A\land B$.
\item Let $A$ and $B$ be propositions. Their \define{Disjunction} is the
  proposition denoted $A\lor B$ and is interpreted as ``Either $A$ or
  $B$ or both''. Its truth table is
\[\begin{array}{cc|c}
A & B & A\lor B\\\hline
\verum  & \verum  & \verum\\
\falsum & \verum  & \verum\\
\verum  & \falsum & \verum\\
\falsum & \falsum & \falsum\\
\end{array}\]
Observe $A\lor B$ is false only when both $A$ and $B$ are false.
The subpropositions $A$ and $B$ are called the \define{Disjuncts} of
$A\lor B$.
\item The conditional ``If $A$, then $B$'' requires some
  justification. Surely ``If $\verum$, then $\falsum$'' should be
  false. But what of the other three possible cases? We should hope ``If
  $\verum$, then $\verum$'' is true. We will adopt the convention that
  ``If $\falsum$, then $B$'' is always true. The truth table may be
  written down as
\[\begin{array}{cc|c}
A & B & A\implies B\\\hline
\verum  & \verum  & \verum\\
\falsum & \verum  & \verum\\
\verum  & \falsum & \falsum\\
\falsum & \falsum & \verum\\
\end{array}\]
We call the subformula $A$ the \define{Antecedent} and the subformula
$B$ the \define{Consequent} (or sometimes \textit{Succedent}) of the
formula $A\implies B$.

\textsc{Caution:} The implication should not be used when considering
counterfactuals or causal laws. As odd as it sounds, neither
counterfactuals nor causal laws are needed in mathematics.
\item Let $A$ and $B$ be propositions. The logical equivalence of $A$
  and $B$, or \define{Biconditional} of $A$ and $B$, is the proposition
  denoted $A\iff B$ and is true only when $A$ and $B$ have the same
  truth values:
\[\begin{array}{cc|c}
A & B & A\iff B\\\hline
\verum  & \verum  & \verum\\
\falsum & \verum  & \falsum\\
\verum  & \falsum & \falsum\\
\falsum & \falsum & \verum\\
\end{array}\]
\end{enumerate}
\end{node}
