% 4
\section{First-order logic}

\begin{node}\label{isabelle:fol-0000}%
Isabelle implements Gentzen's natural deduction systems \textsc{nj} and
\textsc{nk}. Intuitionistic first-order logic is defined first, as the
theory \texttt{IFOL}. Classical logic (theory \texttt{FOL}) is
implemented by extending \texttt{IFOL} with the addition of the double
negation rule. Some basic proof provedures are provided, but it is not
as developed as \texttt{HOL}.
\end{node}

\begin{node}[Terms]\label{isabelle:fol-0003}%
Isabelle uses type classes for forming a hierarchy of classes, and
\texttt{IFOL} defines a ``top type'' (base class) for the
hierarchy: the \texttt{term} type class.
\end{node}

\begin{node}[Formulas]\label{isabelle:fol-0001}%
It appears to be idiomatic Isabelle to denote the type of formulas by
``\texttt{o}'' (I think this is following Church's practice as found in
his article ``A Formulation of the Simple Theory of Types'' which uses
omicron $o$ for the type of formulas). The BNF grammar for formulas is
\begin{center}
\begin{tabular}{rcl}
$\langle$\textit{formula\/}$\rangle$ & $::=$ & $\langle$\textit{term\/}$\rangle$ \texttt{=} $\langle$\textit{term\/}$\rangle$\\
  & $\mid$ & $\langle$\textit{term\/}$\rangle$ \verb|~=| $\langle$\textit{term\/}$\rangle$\\
  & $\mid$ & \verb|~| $\langle$\textit{formula\/}$\rangle$\\
  & $\mid$ & $\langle$\textit{formula\/}$\rangle$ \verb#&# $\langle$\textit{formula\/}$\rangle$\\
  & $\mid$ & $\langle$\textit{formula\/}$\rangle$ \verb#|# $\langle$\textit{formula\/}$\rangle$\\
  & $\mid$ & $\langle$\textit{formula\/}$\rangle$ \verb#--># $\langle$\textit{formula\/}$\rangle$\\
  & $\mid$ & $\langle$\textit{formula\/}$\rangle$ \verb#<-># $\langle$\textit{formula\/}$\rangle$\\
  & $\mid$ & \texttt{ALL} $\langle$\textit{id\/}$\rangle$ $\langle$\textit{id*\/}$\rangle$ \texttt{.} $\langle$\textit{formula\/}$\rangle$\\
  & $\mid$ & \texttt{EX} $\langle$\textit{id\/}$\rangle$ $\langle$\textit{id*\/}$\rangle$ \texttt{.} $\langle$\textit{formula\/}$\rangle$\\
  & $\mid$ & \texttt{EX!} $\langle$\textit{id\/}$\rangle$ $\langle$\textit{id*\/}$\rangle$ \texttt{.} $\langle$\textit{formula\/}$\rangle$\\*
$\langle$\textit{id*\/}$\rangle$ & $::=$ & $\langle$\textit{blank\/}$\rangle$\\
& $\mid$ & $\langle$\textit{id\/}$\rangle$ $\langle$\textit{id*\/}$\rangle$\\
\end{tabular}
\end{center}
\end{node}

\begin{node}[References]\label{isabelle:fol-0002}%
The earliest documentation concerning Isabelle/FOL seems to be Paulson's
tech report~\cite{paulson1993object}. Zhan~\cite{Zhan2017formalization}
applies his AUTO2 tool to \texttt{FOL} and Isabelle/ZF to formalize
results concerning the fundamental group.
\end{node}