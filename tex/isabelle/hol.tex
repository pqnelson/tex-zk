% 5
\section{Higher-Order Logic as Programming Language}

\begin{node}\label{isabelle:hol-0000}%
There are two ways to use \texttt{HOL} in Isabelle: it's a foundations
of mathematics, or it's a functional programming language on steroids.
For this latter perspective, which we will investigate in this section,
Nipkow and Klein~\cite{Nipkow2014concrete} is among the best references.
We have discussed in Chapter~\ref{chapter:sml} how to think of \SML\ as
a deductive system, but now we can actually use a deductive system for
functional programming.
\end{node}

\begin{puzzle}[\SML\ formalized in Isabelle]\label{isabelle:hol-0002}%
Can we actually formalize the semantics of \SML\ in Isabelle? How can we
prove adequacy?
\end{puzzle}

\begin{node}\label{isabelle:hol-0001}%
Unfortunately, confusingly, the type of formulas in Isabelle/HOL is
\texttt{bool} (compared to ``\texttt{o}'' in Isabelle/FOL).
\end{node}

\begin{node}[Getting started]\label{isabelle:hol-0003}%
The scratch theory can be used to do scratch work. We'd import the
``Main'' theory (which is just a wrapper to import all the useful HOL
theories, like lists, sets, etc.):
\begin{Isabelle}
theory Scratch
  imports Main
begin

end
\end{Isabelle}

\begin{node}[Unit testing]\label{isabelle:hol-0004}%
I am not sure there is an idiomatic way to do unit testing. Since
Isabelle can prove the correctness of the code, the only reason to do
unit tests would be to demonstrate some examples of the code (and for
regression testing purposes: to make sure changes don't break anything).
There has been some work done, e.g., by Brucker and
Wolff~\cite{brucker2013theorem}. 

In ``\texttt{HOL/List.thy}'', there is some code provided to check
results:
\begin{Isabelle}[xleftmargin=-6pc]
ML_val \<open>   (* HOL/List.thy *)
  let
    val read = Syntax.read_term \<^context> o Syntax.implode_input;
    fun check s1 s2 =
      read s1 aconv read s2 orelse
        error ("Check failed: " ^
          quote (#1 (Input.source_content s1)) ^ Position.here_list [Input.pos_of s1, Input.pos_of s2]);
  in
    check \<open>[(x,y,z). b]\<close> \<open>if b then [(x, y, z)] else []\<close>;
    check \<open>[(x,y,z). (x,_,y)\<leftarrow>xs]\<close> \<open>map (\<lambda>(x,_,y). (x, y, z)) xs\<close>;
    check \<open>[e x y. (x,_)\<leftarrow>xs, y\<leftarrow>ys]\<close> \<open>concat (map (\<lambda>(x,_). map (\<lambda>y. e x y) ys) xs)\<close>;
    check \<open>[(x,y,z). x<a, x>b]\<close> \<open>if x < a then if b < x then [(x, y, z)] else [] else []\<close>;
    check \<open>[(x,y,z). x\<leftarrow>xs, x>b]\<close> \<open>concat (map (\<lambda>x. if b < x then [(x, y, z)] else []) xs)\<close>;
    check \<open>[(x,y,z). x<a, x\<leftarrow>xs]\<close> \<open>if x < a then map (\<lambda>x. (x, y, z)) xs else []\<close>;
    (* ... *)
\<close>;
\end{Isabelle}
Basically the \verb|check| will read the two arguments its given, and
tests if they are $\alpha$-convertible or not (via the \verb|aconv|
predicate --- I only learned about this function from Appendix A of the
1990 tutorial~\cite{paulson1990tutorial}). 
\end{node}
\end{node}