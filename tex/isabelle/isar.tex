% 6
\section{Isar Proof Language}

\begin{node}\label{isabelle:isar-0000}%
Isar is a declarative proof language for Isabelle, heavily inspired by
Mizar~\zref{chapter:mizar}.
\end{node}

\begin{node}\label{isabelle:isar-0001}%
  The proof skeleton for a claim of the form $A\implies B$ looks like:
\begin{Isabelle}
proof
  assume "A"
  then have C1 by ...
  (* ... *)
  then have Cn by ...
  then show B by ...
qed
\end{Isabelle}

\begin{node}\label{isabelle:isar-0002}%
The justification (``\texttt{by ...}'') can also be a subproof. But
usually we use ``by simp (add: $\langle$\textit{labels\/}$\rangle$)'' or
``using $\langle$\textit{labels\/}$\rangle$ by $\langle$\textit{method\/}$\rangle$'' suffice. The
justification can refer to one of several external theorem provers when
supported by the object logic (Isabelle/HOL supports the largest
diversity of theorem provers).

\begin{node}\label{isabelle:isar-0004}%
The ``\texttt{simp}'' tactic is the ``fastest'' way to solve a problem
requiring rewriting.
\end{node}

\begin{node}[Metis]\label{isabelle:isar-0003}%
Metis\index{Metis}%
is an automated theorem prover based on
paramodulation.\footnote{Available: \url{https://www.gilith.com/software/metis/}}
It can be used as a justification in Isabelle for certain
situations. (Paramodulation generates all ``equal'' versions of clauses
modulo some conditions.)
Joe Hurd first implemented it for HOL4~\cite{hurd2003first}.
For more about it, see F{\"{a}}rber and Kaliszyk~\cite{farber2015metis}.
\end{node}
\end{node}

\begin{node}[Qed methods]\label{isabelle:isar-0005}%
We can optionally pass a method to ``\texttt{qed}'' which Isar will try
to use to solve all remaining goals. No facts may be passed to this method.
\end{node}
\end{node}