% 8
\section{Pure: Isabelle's Metalogic}
\begin{node}[Logical framework]\label{isabelle:metalogic-0000}%
In a logical framework, we need some way to encode judgements and
inference rules, then proofs in the ``object logic'' are leveraged by
using the ``meta-logic''. Isabelle encodes a judgement $\mathcal{J}$ in
the object logic as a proposition in the meta-logic denoted
$\llbracket\mathcal{J}\rrbracket$, and object-logic inference rules
\[\infer{\mathcal{J}}{\mathcal{J}_{1} & \dots & \mathcal{J}_{n}}\]
are encoded as meta-logic implications (written ``\texttt{==>}'' in
Isabelle code)
\[\llbracket\mathcal{J}_{1}; \dots; \mathcal{J}_{n}\rrbracket\implies\llbracket\mathcal{J}\rrbracket.\]
This coincides with how we reason about deductive systems.
\end{node}
\begin{definition}
The metalogic used by Isabelle is called \define{Pure} (or Isabelle/Pure).
\end{definition}
\begin{node}[Implies]\label{isabelle:metalogic-0001}%
We have a primitive constant
\[\Longrightarrow\colon prop\Rightarrow prop\Rightarrow prop\]
which coincides with implication. The ``shorter'' arrows $\Rightarrow$
coincide with the function type in Isabelle's metalogic. When writing
this down on paper, it is useful to use the \textsc{ascii} syntax for
$\Longrightarrow$ which Isabelle uses (``\texttt{==>}'').
\begin{node}[Inference rules]\label{isabelle:metalogic-0004}%
We have its introduction inference rule:
\[\infer[(\implies I)]{A\implies B}{\infer*{B}{[A]}}\]
and its elimination inference rule:
\[\infer[(\implies E)]{B}{A\implies B & A}\]
\end{node} % inference rules
\end{node}
\begin{node}\label{isabelle:metalogic-0002}%
We have the universal quantifier for the metalogic be denoted by
$\bigwedge x.\varphi$, and it's useful when we have an inference rule
involve schematic variables or side conditions. The type signature for
$\bigwedge_{\sigma}\mathbin{:}(\sigma\to prop)\to prop$ where $\sigma$
is a type variable. Specifically, $\bigwedge x.\varphi$ where $x$ has
type $\sigma$ abbreviates $\bigwedge_{\sigma}(\lambda x.\varphi)$.
\begin{node}[Inference rules]\label{isabelle:metalogic-0005}%
We have its introduction inference rule:
\[\infer[(\bigwedge I)]{\bigwedge x. B(x)}{\infer*{B(x)}{[x]}}\]
and its elimination inference rule:
\[\infer[(\bigwedge E)]{B~a}{\bigwedge x.B~x}\]
\end{node}
\end{node}
\begin{node}\label{isabelle:metalogic-0003}%
Equality in the meta-logic is traditionally denoted $\equiv$ in the
Isabelle literature. This is used to implement the underlying
$\lambda$-calculus. The type signature for this is
\[\equiv\colon\alpha\Rightarrow\alpha\Rightarrow prop\]
where $\alpha$ is a type variable.
\begin{node}[Inference rules]\label{isabelle:metalogic-0006}%
Equality is an equivalence relation,
\[\infer{A\equiv A}{},\qquad%
\infer{B\equiv A}{A\equiv B},\qquad%
\infer{A\equiv C}{A\equiv B&B\equiv C}\]
We also define $\lambda$-calculus's inference rules using it:
\[\infer[\alpha\mbox{-conv}]{(\lambda x.B)\equiv(\lambda y.B[y/x])}{}\]
provided $y$ is not free in $B$,
\[\infer[\eta\mbox{-conv}]{(\lambda x.f~x)\equiv f}{}\]
provided $x$ is not free in $f$,
\[\infer[\beta\mbox{-conv}]{(\lambda x.B)A\equiv B[A/x]}{}.\]
We have the abstraction rule,
\[\infer[\mbox{abs}]{(\lambda x.A)\equiv(\lambda x.B)}{A\equiv B}\]
provided $x$ is not free in any of the assumptions of the derivation of
$A\equiv B$. We also have the congruence rule,
\[\infer[\mbox{comb}]{f~a\equiv g~b}{f\equiv g,&a\equiv b}.\]
\end{node}
\end{node}
\begin{node}[References]\label{isabelle:metalogic-0007}%
Isabelle's metalogic seems to first be introduced in Paulson~\cite{paulson1988foundation}.
Chen reviews Isabelle's metalogic in chapter 2 of his master's thesis~\cite{chen2019implementation}.
Ro{\ss}kopf and Nipkow~\cite{Rosskopf2022formalization} formalize this
metalogic in Isabelle itself.
\end{node}

