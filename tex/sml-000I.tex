\begin{node}[Functions]\label{sml-000I}%
\textbf{Type:} $T_{1}\mbox{ \texttt{->} }T_{2}$ where $T_{1}$ and
$T_{2}$ are types. Note that $\texttt{->}$ is right associative, so
$T_{1}\texttt{ -> }T_{2}\texttt{ -> }T_{3}$ is parsed as 
$T_{1}\texttt{ -> }(T_{2}\texttt{ -> }T_{3})$.

\textbf{Values:} Function closures (see \S6.6 of the definition~\cite{milner1997definition}),
which is a tuple $(\mathtt{fn}~pat~\texttt{=>}~e_{b}, E)$ where $pat$ is
a pattern, $E$ is an environment.

\textbf{Operations:} The only operation is applying an argument to a
function, written $f~t$ for applying $t$ to the function $f$.

\textbf{Typing Rules:}
\begin{subequations}
\begin{equation}
\infer[\Rule{T-Abs}]{\mathtt{fn}~x~\texttt{=>}~e\esti T_{1}\to T_{2}}{x\esti T_{1}\vdash e\esti T_{2}}
\end{equation}
\begin{equation}
\infer[\Rule{T-App}]{e_{1,2}~e_{1}\esti T_{2}}{e_{1,2}\esti T_{1}\to T_{2}
& e_{1}\esti T_{1}}
\end{equation}
\end{subequations}

\textbf{Evaluation:} We evaluate the function expression before
evaluating the argument.
\begin{subequations}
\begin{equation}
\infer[\Rule{E-App1}]{e_{1}~e_{2}\Reduces{1}e_{1}'~e_{2}}{e_{1}\Reduces{1}e_{1}'}
\end{equation}
\begin{equation}
\infer[\Rule{E-App2}]{v_{1}~e_{2}\Reduces{1}v_{1}~e_{2}'}{v_{1}~\Value &
  e_{2}\Reduces{1}e_{2}'}
\end{equation}
Now, since we are using the conventions that we never redefine values or
functions, the bindings are determined once. Then we can define beta
reduction with the bindings $\Delta$ explicit in the judgement:
\begin{equation*}
\infer%[\Rule{E-Beta}]
{\Delta\vdash (\mathtt{fn}~x~\texttt{=>}~e_{b})~v_{1}\Reduces{1}\Delta,x=v_{1}\vdash e_{b}}{}
\end{equation*}
This is made more precise by rules (109--113) of the Definition~\cite{milner1997definition},
which is mildly complicated due to pattern matching.
\end{subequations}
\transclude{sml-000M}
\transclude{sml-000K}
\end{node}
