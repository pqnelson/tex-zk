% 5
\chapter{Set Theory}

\begin{node}\label{set-0000}%
Set theory is the ``microcode'' to the ``machine code'' of first-order
logic that mathematics runs on. There are various different ways to
axiomatize the same intuition of a set as a ``well-defined unordered
collection without duplicates''. Underlying all these different
formalisms is \ZFC\ axioms (give or take the axiom of choice). But
\NBG\ allows for finitely axiomatizable set theory by adding minimal
support for proper classes. For more about \ZFC\ set theory,
Jech~\cite{jech2003set} remains the standard reference.
\end{node}

\begin{node}[Classes]\label{set-0003}%
I was brought up believing in \ZFC\ and \NBG, at least nominally. Later
I ended up using Tarski--Grothendieck set theory when working in
category theory. But I still nominally believe there exists collections
``bigger'' than sets which we call ``proper classes'', even though we
don't need anything more than one Grothendieck universe to do all our
normal mathematics. Proper classes may be little more than a myth
(certainly it is not a \emph{convenient} fiction). But I still believe in
and fear them.
\end{node}

\begin{node}[Axiom of Limitation of Size]\label{set-0004}%
We usually denote $\properclass{V}$ to be the proper class of all
sets. Let $X$ be a class. Then $X$ is a proper class if and only if
there exists a surjective function $F\colon X\onto\properclass{V}$.
This works for ``virtual classes'' as well. There we have a scheme,
which looks like:

Let $\schematicvar{C}[-]$ be a unary predicate of objects [our ``virtual
class''], let $\schematicvar{F}(-)$ be a set parametrized by an
object. If for each set $y$ there exists an object such that
$\schematicvar{C}[x]$ and $y=\schematicvar{F}(x)$, then there does not
exist a set $c$ such that for each object $x$ we have $x\in c$ iff
$\schematicvar{C}[x]$.

This allows us to have the limitation of size heuristic in \ZFC\ and
related set theories lacking classes.
\end{node}

\transclude{set/zfc}

\transclude{set/nbg}

\transclude{set/tg}
