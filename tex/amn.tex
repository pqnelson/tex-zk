% 11
\chapter{Taking Notes}
\begin{node}\label{amn-0000}%
Jon Sterling argues that scientific thought requires arbitrary nesting
of trees and the ability to ``transclude'' trees left and right. We can
do this using the \verb|node| environment in my \verb|iso2145|
package. Does it really contribute to productivity? I have my doubts.
\end{node}

\begin{node}\label{amn-0001}%
I should probably give an earnest attempt to write some notes, to give a
fair assessment of how useful this is for writing notes. Towards that
end, I suppose I should give an earnest effort to write some notes on
the various topics I am interested in at the moment.
\end{node}

\begin{node}\label{amn-0002}%
I have written up some Emacs macros to expedite writing new nodes
(simply \verb|Ctrl-c Ctrl-n| will do this for me now) and
``transcluding'' (shudder) nodes using \verb|Ctrl-c t|. The basic macros
are reproduced below:
\lstinputlisting[showstringspaces=false,language=elisp,columns=flexible,basewidth={0.5em,0.5em},basicstyle=\ttfamily\small,upquote=true]{zk.el}
\end{node}

\begin{node}[Organization, IDs]\label{amn-0003}%
Finding anything is a nightmare in this approach. I have severe doubts
about this approach of ``writing the ID first, then the notes''. This
makes no sense at all. I'm following the convention that Jonathan
Sterling has outlined, that IDs should have the form
\begin{equation}
\langle\textit{identifier}\rangle ::= \langle\textit{namespace}\rangle\mbox{-}\langle\textit{number}\rangle,
\end{equation}
where the ``number'' is in base-36 and zero-padded to four places (which
means each namespace can have no more than 1,679,616 entries; adding an
extra digit allows 60,466,176 entries). Then ``namespace'' is a short
(usually 3-letter long) string like: ``amn'' (for metanotes), ``set''
(for set theory), ``fol'' (for first-order logic), etc.
\end{node}

\begin{node}[{Pseudo-hierarchies}]\label{amn-0004}%
Andy Matuschak has argued\footnote{\url{https://notes.andymatuschak.org/Evergreen_notes?stackedNotes=z8SU3r8xyZyvwRhyDdJasJ2&stackedNotes=zPpqLzE6VWKxXF6BpsAfr4C}}
that ``pseudo-hierarchies'' are the ``correct'' way to organize
notes using an ``outline note''. But he doesn't really address why an
``outline note'' is a \emph{pseudo}-hierarchy and not a ``real''
hierarchy. 
\end{node}

\begin{node}\label{amn-0005}%
The biggest problem is finding things. Linking nodes together is
insanely difficult, because I can see the node in the PDF, but I cannot
easily navigate Emacs to the file containing the node. I wonder if the
``big idea'' is not having ``a lot of files'', but to have the
``namespaces'' be a single file each, and to have ``lots of nodes'' and
link among each other? Perhaps I should refactor my notes to try to
organize these notes along such lines.

The argument that Sterling presents uses the Stacks project as an
example where immutable ``tags'' are the critical new feature. But the
Stacks project is organized by huge files for each chapter (or
section?), according to the Github repository\footnote{\url{https://github.com/stacks/stacks-project/tree/master}} for it.
This suggests to me that we should use a similar organization, with each
section (or chapter) being a huge file.

This would require rewriting the \verb|zk.el| file to manage the ``tag''
(i.e., \LaTeX\ label) for each node\dots somehow\dots

\begin{node}\label{amn-0006}%
I have decided to hack Yasnippets with some custom macros to expand
things using a comment at the first line of the file keeping track of
the node number. This will allow me to jump around using the ``search
within buffer'' Emacs provides. (I should add a YaSnippet to insert a
label for definitions, theorems, etc.) The Yasnippet for a node is reproduced
below (for \LaTeX-mode):
\lstinputlisting[breaklines=true,showstringspaces=false,language=elisp,columns=flexible,basewidth={0.5em,0.5em},basicstyle=\ttfamily\small,upquote=true]{n.yasnippet}
\end{node}
\end{node}

\begin{node}\label{amn-0007}%
I have added the ability to organize files with subdirectories in the
\verb|tex/| directory. The namespace for the nodes in files in these
subdirectories have their relative path prepended to it. For example, in
\verb|tex/prop/file.tex| the namespace would be \texttt{prop-file-nnnn}.
This allows me to do things like place all the notes about normal forms
in propositional logic in one easily identifiable place.

Better, this is more along the lines of Luhmann's philosophy about
organizing notes. I am creating ``clusters'' of notes, organized around
subjects as they are needed, and they emerge to clarify the organization
of thoughts. These are not ``tags'', nor are they ``pre-existing
categories'' (like the Dewey decimal system or Scott Schepper's
misunderstanding of Luhmann's system).
\end{node}

\begin{node}\label{amn-0008}%
I just found\footnote{\url{https://theory.stanford.edu/people/uribe/mail/qed.messages/33.html}} this quote from Lawrence Paulson
in an email dated April 20, 1993:
\begin{quote}
It may seem dangerous to speak of illusions when we are after certainty; a
fault in the parser/prettyprinter could be nearly as harmful as a fault in the
inference mechanisms themselves.  But illusions are a tool of our trade. 
Inside the computer there is nothing but bit patterns.  Every theorem prover
encodes its logic ultimately in terms of bits.  Even programmers rely upon the
illusion that terms and formulae are present.
\end{quote}
\end{node}

\begin{node}\label{amn-0009}%
I am still relatively confused by how Jon Sterling thinks we should be
writing notes. His only real instructions are:\footnote{\url{http://www.jonmsterling.com/jms-007L.xml}}
\begin{quote}
You may be used to writing \LaTeX\ documents, where you work from the top down: you create some section headings, put some text under those headings, make some deeper section headings, put more text, etc.
\emph{Forests work in the opposite way, from the bottom up: you start by writing independent, atomic notes/trees and then only later start to (sparingly) assemble these into a hierarchy in order to reify the emerging structure.}
(Emphasis mine)
\end{quote}
Again, the difficulty is that we end up with a lot of files which we
then manually have to keep track of, with increasing risk of duplication.
\end{node}

\begin{node}[Curl to get Bibtex entries]\label{amn-000A}%
  We can use curl to get \textsc{Bib}\TeX\ entries, for example:
\begin{Verbatim}
$ curl -LH "Accept: application/x-bibtex" https://doi.org/10.1006/jsco.2001.0456
  @article{Rudnicki_2001,
    title={Commutative Algebra in the Mizar System},
    volume={32}, ISSN={0747-7171},
    url={http://dx.doi.org/10.1006/jsco.2001.0456},
    DOI={10.1006/jsco.2001.0456}, number={1--2},
    journal={Journal of Symbolic Computation},
    publisher={Elsevier BV},
    author={Rudnicki, Piotr and Schwarzweller, Christoph and Trybulec, Andrzej},
    year={2001}, month=jul, pages={143--169} }
\end{Verbatim}
\end{node}