% 39

\begin{node}\label{untyped-lambda-0003}%
Lambda calculus studies a formal language which captures the basic
properties of variables and substitution we learn in primary
school. This is so basic that it underlies all type theoretic
foundations for mathematics, and so it is worth studying in its own
right. We can partition the study of lambda calculus into two segments:
\emph{syntax} studies how we form ``words'' and ``sentences'' in lambda
calculus (which strings are grammatically well-formed), and
\emph{semantics} which vaguely refers to how we understand the meaning
and usage of the syntactic parts of the language.

In other words, the \emph{syntax} studies the formal language underlying
the $\lambda$-calculus, the \emph{semantics} studies its ``meaning''
(how we manipulate these strings to ``compute'' things).
\end{node}

\begin{node}[Syntax]\label{untyped-lambda-0002}%

\begin{definition}\label{untyped-lambda-0000}%
Let $V$ be a (nonempty, countably infinite) set of variables. We define
the \define{$\lambda$-Terms} (or ``\textit{$\lambda$-Expressions}'', or
just ``terms'' and ``expressions'' if the context is clear) to be
inductively defined as:
\begin{enumerate}
\item Any variable $x\in V$ is a $\lambda$-expression;
\item Abstraction terms: If $x\in V$ and $e$ is a $\lambda$-expression,
  then we may form the expression ``$\lambda x.e$'';
\item Application terms: if $e_{1}$ and $e_{2}$ are
  $\lambda$-expressions (called ``rator'' and ``rand'' expressions,
  respectively, short for ``operator'' and ``operand''),
  then so is ``$e_{1}~e_{2}$''.
\end{enumerate}
In abstraction terms $\lambda x.e$, we call $e$ the ``body'' of the
abstraction, and the $\lambda x$ the ``head'' of the abstraction. The
``$\lambda$'' appearing in $\lambda x.e$ is called a ``binder'' or
``binding operator'' since it \emph{binds} a variable in the sense
discussed below \zref{untyped-lambda-0007}.

\begin{node}\label{untyped-lambda-0001}%
We can form the BNF grammar for lambda expressions as (allowing
arbitrary parentheses to clarify expressions, and variables ranging over
the allowed set):
\begin{center}  
\begin{tabular}{rcl}
$\langle$\textit{expression}$\rangle$ & ::= & $\langle$\textit{variable}$\rangle$\\
& $|$ & $\lambda$ $\langle$\textit{variable}$\rangle$ $.$ $\langle$\textit{expression}$\rangle$\\
& $|$ & $\langle$\textit{expression}$\rangle$ $\langle$\textit{expression}$\rangle$\\
& $|$ & \texttt{(}$\langle$\textit{expression}$\rangle$\texttt{)}
\end{tabular}
\end{center}
\end{node}

\begin{node}\label{untyped-lambda-0004}%
We could also define the set of $\lambda$-expressions using the
following sequence $\{S_{n}\}_{n\in\NN_{0}}$ of sets:
\begin{enumerate}
\item $S_{0}:=V$ is precisely the set of variables;
\item $S_{n+1}:=S_{n}\cup\{\lambda x.e\mid x\in V,e\in S_{n}\}\cup\{e_{1}~e_{2}\mid e_{1},e_{2}\in S_{n}\}$.
\end{enumerate}
Then the set $S$ of $\lambda$-expressions is
\begin{equation}
S = \bigcup^{\infty}_{n=0}S_{n}.
\end{equation}
This is equivalent to the previous presentations.
\end{node}
\end{definition}

\begin{node}[Precedence, associativity]\label{untyped-lambda-000T}%
We will interpret application as left associative, so
$e_{1}~e_{2}~e_{3}$ is read as $(e_{1}~e_{2})~e_{3}$. We will also take
abstract extending as far as possible, so we read ``$\lambda x.e_{1}~e_{2}~e_{3}$''
as ``$\lambda x.(e_{1}~e_{2}~e_{3})$''.
It is also common to ``collapse'' the $\lambda$ binder when multiple
appear together, writing $\lambda x.\lambda y.e$ as $\lambda x, y. e$.
\end{node}

\begin{node}[Structural Recursion and Induction]\label{untyped-lambda-0009}%
Given an inductive definition like this, we may use its inductive nature
to form for any $\lambda$-expression $e$ a set $f(e)$ (or any other
mathematical object $f(e)$) using a technique
called \define{Structural Recursion} by the following rules:
\begin{enumerate}
\item We have some way to form a set $f_{\text{var}}(x)$ from variables
  $x\in V$, and have $f(x):=f_{\text{var}}(x)$;
\item For an abstraction term $\lambda x.e$, the inductive hypothesis
  assumes we have some way to form the set $f(e)$, then we need some way
  to form $f(\lambda x.e)$ using $f_{\text{var}}(x)$ and $f(e)$;
\item For an application term $e_{1}~e_{2}$, the inductive hypothesis
  assumes we have formed sets $f(e_{1})$ and $f(e_{2})$, then we need
  some way to construct a set for $f(e_{1}~e_{2})$ using them.
\end{enumerate}
This suffices for constructing a set $f(e)$ for any $\lambda$-expression $e$.

If we want to prove a property $\mathcal{P}[-]$ for any
$\lambda$-expression $e$, we may use an analogous technique called
\define{Structural Induction} by the following steps:
\begin{enumerate}
\item Base cases: for any variable $x\in V$, we need to prove $\mathcal{P}[x]$;
\item Abstraction: for any abstraction expression $\lambda x.e$, the
  inductive hypothesis assumes $\mathcal{P}[e]$ has been proven, and we
  need to prove $\mathcal{P}[\lambda x.e]$ holds;
\item Application: for any application term $e_{1}~e_{2}$, the inductive
  hypotheses assumes both $\mathcal{P}[e_{1}]$ and $\mathcal{P}[e_{2}]$,
  then requires us to prove $\mathcal{P}[e_{1}~e_{2}]$.
\end{enumerate}
If we consider the construction \zref{untyped-lambda-0004}, this amounts
to proving $\mathcal{P}[-]$ holds for $S_{0}$ and then proving if it
holds for $S_{n}$ then it holds for $S_{n+1}$. In this manner, we see
how it relates to ``ordinary'' mathematical induction if we define
$\mathcal{Q}[n]:=\forall e\in S_{n},\mathcal{P}[e]$, then structural
induction amounts to ordinary induction applied to
$\mathcal{Q}[-]$.\footnote{This is accidentally true for untyped lambda
calculus, the reality is that this is a generalization known as Noetherian induction.}
\end{node}

\begin{definition}\label{untyped-lambda-0008}%
We can define the set of \define{Subexpressions} of a $\lambda$-expression
$e$ to be the set $\Sub(e)$ defined recursively as:
\begin{enumerate}
\item $\Sub(x)=\{x\}$;
\item $\Sub(\lambda x.e)=\{\lambda x.e\}\cup\{x\}\cup\Sub(e)$;
\item $\Sub(e_{1}~e_{2})=\{e_{1}~e_{2}\}\cup\Sub(e_{1})\cup\Sub(e_{2})$.
\end{enumerate}
We shall call a $\lambda$-expression $e'$ a \define{Subexpression} of
$e$ if $e'\in\Sub(e)$. Furthermore, if $e'\neq e$, we shall call it a
\define{Proper} subexpression.
\end{definition}

\begin{definition}\label{untyped-lambda-000F}%
Let $e$ be any $\lambda$-expression. We will call a variable $x$
\define{Fresh} in $e$ if it does not appear in the expression $x\notin\Sub(e)$.
We can extend this notion, saying a variable $x$ to be free in a
family of $\lambda$-expressions iff it is free in each $\lambda$-expression.

If we want to take a fresh variable relative to an expression $e$ (or
several expressions), then we will adopt the convention of writing
$\fresh{x}$ for the fresh variable. (This is taken from the Clojure
language's conventions.)
\end{definition}

\begin{node}[Free and bound variables]\label{untyped-lambda-0007}%
The variable appearing in the head of an abstraction term (i.e., the
``$x$'' appearing in ``$\lambda x.e$'') plays the role of keeping track
of a ``parameter'' to a function expression which will be substituted
for an expression upon application. The appearance of such variables in
the body of its abstraction (i.e., the appearances of $x$ in the
subexpression $e$ of $\lambda x.e$) are \emph{bound} to the $\lambda$
``binder''. Compare this to variables which do not have a ``binding
$\lambda$'', which are considered \emph{free}.

In general, in almost all formal systems where ``free'' and ``bound''
variables can be distinguished, free and bound variables are
distinguished by if their values affect the value of the overall
expression. Free variables are distinguished by the property that the
value of the expression depends on the value of free variables. Bound
variables are shorthand ``stage instructions'' for the semantical
meaning of expressions.

\begin{definition}\label{untyped-lambda-0005}%
We inductively define the set of \define{Free Variables} in a
$\lambda$-expression $e$ as the set $FV(e)$ formed by the following rules:
\begin{enumerate}
\item $FV(x)=\{x\}$;
\item $FV(\lambda x.e)=FV(e)\setminus\{x\}$ --- all the free variables
  in the body $e$, except for $x$;
\item $FV(e_{1}~e_{2})=FV(e_{1})\cup FV(e_{2})$.
\end{enumerate}
\end{definition}

\begin{definition}\label{untyped-lambda-0012}%
When a $\lambda$-expression $e$ has no free variables $FV(e)=\emptyset$,
then we say $e$ is a \define{Closed} term.
\end{definition}

\begin{definition}\label{untyped-lambda-0006}%
We inductively define the set of \define{Bound Variables} in a
$\lambda$-expression as the set $BV(e)$ formed by the following rules:
\begin{enumerate}
\item $BV(x)=\emptyset$;
\item $BV(\lambda x.e)=\{x\}\cup BV(e)$;
\item $BV(e_{1}~e_{2})=BV(e_{1})\cup BV(e_{2})$.
\end{enumerate}
\end{definition}

\begin{node}[Example]\label{untyped-lambda-000A}%
Consider the expression $\lambda x.(x~y~z)$. There is one bound
variable, $x$, and two free variables $y$ and $z$.
\end{node}

\begin{node}[Example]\label{untyped-lambda-000B}%
Consider the expression $(x~\lambda y.z)(\lambda x.y)$. We can compute
the set of free variables:
\begin{subequations}
\begin{align}
FV((x~\lambda y.z)(\lambda x.y))
&=FV(x~\lambda y.z)\cup FV(\lambda x.y)\\
&=\bigl(FV(x)\cup FV(\lambda y.z)\bigr)\cup FV(\lambda x.y)\\
\intertext{then applying the rule for determining the $FV$ of abstraction terms}
&=\bigl(FV(x)\cup\bigl(FV(z)\setminus\{y\}\bigr)\bigr)\cup\bigl(FV(y)\setminus\{x\}\bigr)\\
\intertext{and the rule for $FV$ of free variables}
&=\bigl(\{x\}\cup\bigl(\{z\}\setminus\{y\})\bigr)\bigr)\cup\{y\}\\
&=\bigl(\{x\}\cup(\{z\})\bigr)\cup\{y\}\\
&=\bigl(\{x, z\})\bigr)\bigr)\cup\{y\}\\
&=\{x, z, y\}.
\end{align}
\end{subequations}
Similarly, the set of bound variables:
\begin{subequations}
\begin{align}
BV((x~\lambda y.z)(\lambda x.y))
&=BV(x~\lambda y.z)\cup BV(\lambda x.y)\\
&=\bigl(BV(x)\cup BV(\lambda y.z)\bigr)\cup\bigl(BV(\lambda x.y)\bigr)\\
&=\bigl(BV(x)\cup\bigl(\{y\}\cup BV(z)\bigr)\bigr)\cup\bigl(\{x\}\cup BV(y)\bigr)\\
\intertext{then applying the rule that $BV$ of free variables is the
  empty set:}
&=\bigl(\emptyset\cup\bigl(\{y\}\cup\emptyset\bigr)\bigr)\cup\bigl(\{x\}\cup\emptyset\bigr)\\
&=\{y\}\cup\{x\}=\{x,y\}.
\end{align}
\end{subequations}
Observe, then, that $x$ and $y$ are both free \emph{and} bound variables.
\end{node}
\end{node}

\begin{definition}\label{untyped-lambda-000E}%
Let $x$ be a variable, let $e$ and $e_{0}$ be $\lambda$-expressions. We
define \define{Capture-free substitution} replacing $x$ by $e_{0}$ in $e$
to be the $\lambda$-expression denoted $e[e_{0}/x]$ or $e[x:=e_{0}]$ by
structural recursion
\begin{enumerate}
\item For variables: $\displaystyle y[e_{0}/x]=\begin{cases}e_{0} & \mbox{if }x=y\\ y & \mbox{if }x\neq y \end{cases}$
\item Abstraction: $(\lambda y.e)[e_{0}/x]$ boils down to several cases
\begin{enumerate}
\item If $x=y$: $(\lambda x.e)[e_{0}/x]=\lambda x.e$ does nothing;
\item If $x\neq y$ and $y\notin FV(e_{0})$: $(\lambda y.e)[e_{0}/x]=\lambda y.(e[e_{0}/x])$;
\item If $x\neq y$ and $y\in FV(e_{0})$: take a fresh variable \zref{untyped-lambda-0005} $\fresh{y}$
  such that $\fresh{y}\notin\Sub(e_{0})$, then we replace the bound $y$
  in $e$ by $\fresh{y}$ and use the previous rule, i.e.,
  $(\lambda y.e)[e_{0}/x]=(\lambda \fresh{y}.(e[\fresh{y}/y]))[e_{0}/x]$
  (i.e., use $\alpha$-congruent~\zref{untyped-lambda-000V} rator
  expressions to avoid accidentally binding any free variable in the
  rand expression)
\end{enumerate}
\item Application: $(e_{1}~e_{2})[e_{0}/x] = (e_{1}[e_{0}/x])~(e_{2}[e_{0}/x])$
  we just perform capture-free substitution on the rator and rand subexpressions.
\end{enumerate}
\end{definition}

\begin{node}[Example]\label{untyped-lambda-000M}%
The difficulty with abstraction terms may be seen easily with the
attempt to make sense of $(\lambda y.x)[y/x]$. If we just ``naively''
try to replace $x$ by $y$, then we get the incorrect result $\lambda y.y$
instead of the correct solution $\lambda z.y$ where $z$ is the fresh
variable chosen to avoid ``capturing'' $y$ as a newly bound variable.
\end{node}

\begin{definition}\label{untyped-lambda-000V}%
Let $e$ be a $\lambda$-expression. We define
\define{$\alpha$-Congruence} to be the smallest equivalence relation
generated by, for all variables $x$ and all ``fresh''
variables~\zref{untyped-lambda-0005} $\fresh{y}$, $\lambda x.e
\cong_{\alpha}\lambda\fresh{y}.(e[\fresh{y}/x])$.

\begin{node}\label{untyped-lambda-000G}%
Alpha congruence amounts to saying terms with renamed bound variables
are equivalent to each other. This is keeping with our earlier
remarks~\zref{untyped-lambda-0007} about bound variables being used for
book-keeping purposes.
\end{node}

\begin{node}[Convention]\label{untyped-lambda-000H}%
We will take the convention that $\alpha$-congruent expressions may be
identified as equivalent, and freely replaced by each other wherever
they appear as subexpressions \emph{provided there are no captured variables}. 
That is to say, we do not accidentally bind variables by the replacement
of $\alpha$-congruent subexpressions.
\end{node}
\end{definition}


\end{node}

\begin{node}[Semantics]\label{untyped-lambda-000C}%

\begin{node}\label{untyped-lambda-000D}%
There are various ways to study ``semantics'' of a formal system. If we
wish to view it as a \emph{mathematical theory} in its own right, then
we can rely on the familiar classical model theory (as found in, e.g.,
Chang and Keisler~\cite{chang2012model}). If we wish to use it \emph{as}
a deductive system (e.g., as a foundations for mathematics, or as an
example of a programming language), then we should prefer to examine its
so-called \emph{operational semantics}.

The models of untyped $\lambda$-calculus are really intriguing, because
it requires a field of mathematics called ``domain theory'' which allows
constructing mathematical gadgets $D$ which look like $D=D^{D}$. (This
clearly cannot be a set, but if we consider topological spaces or
posets, then we may have a chance for a model --- this was Dana Scott's
\textit{coup d'\oe{}il}.)
We are not interested in $\lambda$-calculus as ``just another mathematical theory''.
Readers wishing to learn more about this, Barendregt~\cite{barendregt2012lambda}
remains the invaluable reference.
\end{node}

\begin{definition}\label{untyped-lambda-000I}%
We define \define{Beta Reduction} to be the relation generated by (for
any variable $x$ and $\lambda$-expressions $e$ and $e'$)
\[(\lambda x.e)e'\betareduce e[e'/x]. \]
This is the heart of semantics for untyped $\lambda$-calculus. It tells
us that $\lambda x.e$ acts like a function $f(x)=e$, and application
$(\lambda x.e)~e'$ results in an expression analogous to substitution $f(e')$.
Specifically, it results in the \emph{capture-free substitution}~\zref{untyped-lambda-000E}
replacing $x$ by $e'$ in $e$.
\end{definition}

\begin{node}[Inference rules]\label{untyped-lambda-000J}%
We can specify the behaviour of beta reduction by treating $e\betareduce e'$
as a judgement, then the inference rules tell us how to use it. The
difficulty is that there are multiple different choices for inference
rules describing $\beta$-reduction. We have three representative
different ``operational semantics'', or inference rules defining the
behaviour of $\beta$-reduction.

\begin{node}[Full $\beta$-reduction]\label{untyped-lambda-000K}%
Full $\beta$-reduction (or ``full semantics'') allows us to apply
$\beta$ reduction to any subexpression. This is the most generous
strategy, defined by the following inference rules:
\begin{subequations}
\begin{equation}
\infer[\Rule{E-App1}]{e_{1}~e_{2}\betareduce e_{1}'~e_{2}}{e_{1}\betareduce e_{1}'}
\end{equation}
\begin{equation}
\infer[\Rule{E-App2}]{e_{1}~e_{2}\betareduce e_{1}~e_{2}'}{e_{2}\betareduce e_{2}'}
\end{equation}
\begin{equation}
\infer[\Rule{E-Beta}]{(\lambda x.e_{1})~e_{2}\betareduce e_{1}[e_{2}/x]}{}
\end{equation}
\begin{equation}
\infer[\Rule{E-Body}]{\lambda x.e\betareduce \lambda x.e'}{e\betareduce e'}
\end{equation}
\end{subequations}
Note that \Rule{E-Body} is sometimes called the $\xi$-rule.
\end{node}

\begin{node}[Call-by-value rules]\label{untyped-lambda-000L}%
We may alternatively use rules resembling ``eager evaluation'' where we
do not $\beta$-reduce the bodies of abstractions, and only
$\beta$-reduce the rand subexpression in an application when the rator
is an abstraction term. Here, ``values'' means any abstraction term
(since they are not reduced any further). This is the ``call-by value
semantics''. Formally, this is defined by the rules
\begin{subequations}
\begin{equation}
\infer[\Rule{E-Cbv-App1}]{e_{1}~e_{2}\betareduce e_{1}'~e_{2}}{e_{1}\betareduce e_{1}'}
\end{equation}
\begin{equation}
\infer[\Rule{E-Cbv-App2}]{(\lambda x.e_{1})~e_{2}\betareduce (\lambda x.e_{1})~e_{2}'}{e_{2}\betareduce e_{2}'}
\end{equation}
\begin{equation}
\infer[\Rule{E-Cbv-Beta}]{(\lambda x.e_{1})~v_{2}\betareduce e_{1}[v_{2}/x]}{}
\end{equation}
where $v_{2}$ is a value (i.e., an abstraction term).
\end{subequations}
\end{node}

\begin{node}[Call-by-name rules]\label{untyped-lambda-000N}%
We can formalize a ``lazy'' semantics for $\beta$-reduction. That is to
say, when doing $\beta$-reduction, we transform rator expressions until
they are abstractions, then substitute the rand expression freely. The
rand is not evaluated (or, more precisely, it is only evaluated when
needed, i.e., \emph{lazily}), unlike the call-by-value semantics.
This is the ``call-by-name semantics'', consisting of the rules: 
\begin{subequations}
\begin{equation}
\infer[\Rule{E-Cbn-App1}]{e_{1}~e_{2}\betareduce e_{1}'~e_{2}}{e_{1}\betareduce e_{1}'}
\end{equation}
\begin{equation}
\infer[\Rule{E-Cbn-Beta}]{(\lambda x.e_{1})~e_{2}\betareduce e_{1}[e_{2}/x]}{}
\end{equation}
\end{subequations}
\end{node}

\begin{node}[Does it ``stop''?]\label{untyped-lambda-000R}%
We see that full $\beta$-reduction is nondeterministic (we have multiple
different possible choices of which rule we could apply). Are there
instances where it will ``settle down'' and we have no more rules we
could apply? Or will it always be possible to apply one or more
inference rule to expressions?
\end{node}
\end{node}

\begin{node}[Example]\label{untyped-lambda-000S}%
Consider the $\lambda$-expression $\Omega:=\lambda x.(x~x~x)$.
Evaluate $\Omega~\Omega$ using full $\beta$-reduction.

We see that $\Omega$ is a value, and that all three semantic strategies
apply to $\Omega~\Omega$. But we see that, after one step by any of the
strategies,
\begin{equation}
\Omega~\Omega\betareduce\Omega~\Omega~\Omega.
\end{equation}
If we keep applying $\beta$-reduction, and write
$\Omega^{n+1}=\Omega^{n}~\Omega$ with $\Omega^{2}=\Omega~\Omega$, we get
the divergent sequence of expressions
$\Omega^{2}\betareduce\Omega^{3}\betareduce\dots\betareduce\Omega^{n}\betareduce\dots$.
\end{node}

\begin{definition}\label{untyped-lambda-000Q}%
We define the \define{Multistep $\beta$ Reduction} in zero or more steps
to be the relation $\multistep\betareduce$ such that
\begin{enumerate}
\item For any expression $e$, we have $e\multistep\betareduce e$;
\item If $e\betareduce e'$, then $e\multistep\betareduce e'$;
\item If $e\multistep\betareduce e'$ and $e'\betareduce e''$,
  then $e\multistep\betareduce e''$.
\end{enumerate}
That is to say, $\multistep\betareduce$ is the reflexive, transitive closure
of $\beta$-reduction. If we just want the transitive closure of
$\beta$-reduction, we refer to it by $\betareduce^{+}$.
\end{definition}

\begin{definition}\label{untyped-lambda-000O}%
We say a $\lambda$-expression $e$ is in \define{Normal Form} if there is
no $\lambda$-expression $e'$ such that $e\betareduce e'$. We can
generically refer to \emph{normal forms} as expressions in normal form.

Note: the notion of a ``normal form'' depends on our choice of semantics
for $\beta$-reduction.
\end{definition}

\begin{node}[Exercise]\label{untyped-lambda-000U}%
Prove or find a counter-example: in call-by-value semantics, normal
forms are values (i.e., abstraction expressions).
\end{node}

\begin{theorem}[Church--Rosser Confluence]\label{untyped-lambda-000P}%
When working with $\lambda$-expressions modulo $\alpha$-congruence
(i.e., treating $\alpha$-congruent expressions as equivalent), 
if $e\multistep\betareduce e_{1}$ and $e\multistep\betareduce e_{2}$,
then there exists an expression $e'$ such that
$e_{1}\multistep\betareduce e'$ and $e_{2}\multistep\betareduce e'$.
\end{theorem}

\begin{node}\label{untyped-lambda-000W}%
As an immediate consequence, if an expression has a normal form, then it
is unique and will eventually be reached by finitely many steps of
$\beta$-reduction.
\end{node}

\end{node}

\begin{node}[Church Encodings]\label{untyped-lambda-000X}%
We can use clever encodings of datatypes to do some programming with
untyped $\lambda$-calculus.

\begin{node}[Booleans]\label{untyped-lambda-000Y}%
We can define the terms
\begin{subequations}
\begin{align}
\mathtt{true} &:= \lambda x,y.x\\
\mathtt{false} &:= \lambda x,y.y
\end{align}
The boolean connectives may be formed, e.g., conjunction
\begin{equation}
\mathtt{and} := \lambda a,b.a~b~\mathtt{false}.
\end{equation}
This satisfies the usual rules
\begin{equation}
\mathtt{and}~\mathtt{true}~\mathtt{true}\multistep\betareduce\mathtt{true}
\end{equation}
\begin{equation}
\mathtt{and}~\mathtt{true}~\mathtt{false}\multistep\betareduce\mathtt{false}
\end{equation}
\begin{equation}
\mathtt{and}~\mathtt{false}~\mathtt{true}\multistep\betareduce\mathtt{false}
\end{equation}
\begin{equation}
\mathtt{and}~\mathtt{false}~\mathtt{false}\multistep\betareduce\mathtt{false}
\end{equation}
\end{subequations}
Similarly, we can define $\mathtt{if}~b~\mathtt{then}~e_{1}~\mathtt{else}~e_{2}$
to be an abbreviation of $b~e_{1}~e_{2}$.
\end{node}

\begin{node}[Arithmetic]\label{untyped-lambda-000Z}%
Recall Peano arithmetic introduces a ``number'' $Z$ (for ``zero'') and
an operation $S(n)$ for any number $n$ ($S$ stands for
``successor''). We can encode this using $\lambda$-calculus as:
\begin{subequations}
\begin{equation}
0 := \lambda s.\lambda z.z
\end{equation}
\begin{equation}
1 := \lambda s.\lambda z.s~z
\end{equation}
\begin{equation}
2 := \lambda s.\lambda z.s~(s~z)
\end{equation}
\begin{equation}
3 := \lambda s.\lambda z.s~(s~(s~z))
\end{equation}
and so on. A number that tells us how many times to add the successor to
zero, just as Peano axiomatized. Then adding two numbers together may be
formed by 
\begin{equation}
\mathtt{PLUS} := \lambda m.\lambda n.\lambda s.\lambda z.m~s~(n~s~z).
\end{equation}
Intuitively this takes two numbers $m$ and $n$, their sum is obtained by
taking $m$ successors of $n$ (since $n~s~z$ applies $n$ successor
operations $s$ to the zero constant $z$, and $m~s$ takes a number and
applies $m$ successor operations $s$ to it). Multiplication may be taken
as
\begin{equation}
\mathtt{MULT} :=\lambda m.\lambda n.m~(\mathtt{PLUS}~n)~0,
\end{equation}
which instead of working one successor at a time, works $n$ successors
at a time, and takes $m$ of these operations applied to the zero constant.
\end{subequations}
\end{node}

\begin{node}[Lists]\label{untyped-lambda-0010}%
We can encode lists using $\lambda$-expressions. We would have the
following setup:
\begin{subequations}
\begin{equation}
\mathtt{PAIR} := \lambda x.\lambda y.\lambda f.f~x~y
\end{equation}
\begin{equation}
\mathtt{FIRST} := \lambda p.p~\mathtt{true}
\end{equation}
\begin{equation}
\mathtt{SECOND} := \lambda p.p~\mathtt{false}
\end{equation}
\begin{equation}
\mathtt{NIL} := \lambda x.\mathtt{true}
\end{equation}
\begin{equation}
\mathtt{NULL} := \lambda p.p~(\lambda x.\lambda y.\mathtt{false}).
\end{equation}
Then a linked list is either \texttt{NIL} for the empty list, or a
``node'' (``cons'') \texttt{PAIR} of an element and a tail sublist. The
predicate \texttt{NULL} tests for the value \texttt{NIL}.
\end{subequations}
\end{node}

\begin{node}[Problem]\label{untyped-lambda-0011}%
A difficulty arises that $\lambda$-calculus allows us to concatenate
expressions together which should be illegal, but attempts to evaluate
it regardless. For example $\mathtt{PLUS}~\mathtt{NIL}~\mathtt{false}$
makes no sense, since we expect $\mathtt{PLUS}$ to add two numbers
together, but it is supplied with a list (\texttt{NIL}) and a Boolean
value (\texttt{false}). An error should be raised. But $\lambda$-calculus
cheerfully tries to evaluate this. This leads to introducing a notion of
types to avoid this problem.
\end{node}
  
\end{node}
