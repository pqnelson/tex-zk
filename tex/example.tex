\begin{node} %1
  Blah
\end{node}
\begin{node} % 2
  Node 2
\end{node}
\begin{node}
  Node 3
  \begin{node}
    Node 3.1
    \begin{node}
      Node 3.1.1
      \begin{node}[Proposition]%
Node 3.1.1.1
      \end{node}
      \begin{theorem}[Euclid]
There exists infinitely many primes.
      \end{theorem}
      \begin{node}
        Node 3.1.1.2
      \end{node}
      \begin{node}
        Node 3.1.1.3
      \end{node}
      \begin{theorem}
There exists no real number $x$ such that $x^{2}=-1$.
      \end{theorem}
      \begin{node}
        Node 3.1.1.4
      \end{node}
    \end{node}
    \begin{node}
      Node 3.1.2
    \end{node}
    \begin{node}
      Node 3.1.3
    \end{node}
    \begin{node}
      Node 3.1.4
    \end{node}
  \end{node}
  \begin{node}
    3.2
  \end{node}
  \begin{node}
    3.3
  \end{node}
\end{node}


\M
Perhaps if we have a \verb#node# environment, which acts like a
chunk. And if we have nested nodes, it amounts to ``growing'' depth. 

\M
This is a test to see if this works. \lipsum[2-4]

\M I hope it works. \lipsum[1-3]

\M[1]
And this should be \S{2.1}, I hope. \lipsum[4-5]

\N[1]{Claim:} This should be \S{2.1.1}.

\M[-1]
This should be \S{2.2}.

\M[3]
This should be \S{2.2.0.0.1}, I hope?

\N[-3]{Theorem.}%
This is \S3, yes?

\M[-1]
No, this is \S3.

\M[0] This should be \S4.

\N[0]{Corollary.} This is \S5?



